\documentclass[sigconf]{acmart}

% Recommended, but optional, packages for figures and better typesetting:
\usepackage{microtype}
\usepackage{graphicx}
\usepackage{subfigure}
\usepackage{booktabs} % for professional tables
\usepackage{hyperref}

% Attempt to make hyperref and algorithmic work together better:
\newcommand{\theHalgorithm}{\arabic{algorithm}}

% additional packages
%\usepackage[numbers,compress]{natbib}
\usepackage{mathtools}
\usepackage{amsfonts}
\usepackage{amsmath,amssymb}%,amsthm}
\usepackage{mathrsfs}
\usepackage{nicefrac}  % compact symbols for \nicefrac[]{1}{2}, etc.
%\usepackage[table]{xcolor} % colour table

\usepackage{bm}
\usepackage{bbm}
\usepackage{stmaryrd}
\usepackage{enumerate}
\usepackage{algorithm}
\usepackage{algorithmic}
\usepackage{tablefootnote}    % footnote in table and tabular env
\usepackage{footmisc}   % \footref, refer the same footnote at different places
%\usepackage{subcaption} % sub-figures, cannot be used with subfigure package
\usepackage{setspace}   % set space between lines
\usepackage[english]{babel}
\usepackage{multirow} % for tables
\usepackage{colortbl} % cellcolor
\graphicspath{{fig/}}   % Location of the graphics files

%\newtheorem{theorem}{Theorem}
%\newtheorem{corollary}{Corollary}
%\newtheorem{lemma}{Lemma}
%\newtheorem{proposition}{Proposition}
%\newtheorem{remark}{Remark}

\DeclareMathOperator*{\argmin}{argmin}
\DeclareMathOperator*{\argmax}{argmax}
\newcommand{\eat}[1]{}
\newcommand{\given}{\mid}
\newcommand{\llb}{\llbracket}
\newcommand{\rrb}{\rrbracket}
\newcommand{\bu}{\mathbf{u}}
\newcommand{\bv}{\mathbf{v}}
\newcommand{\bb}{\mathbf{b}}
\newcommand{\bc}{\mathbf{c}}
\newcommand{\f}{\mathbf{f}}
\newcommand{\h}{\mathbf{h}}
\newcommand{\x}{\mathbf{x}}
\newcommand{\y}{\mathbf{y}}
\newcommand{\z}{\mathbf{z}}
%\newcommand{\1}{\mathbf{1}}
\newcommand{\w}{\mathbf{w}}
\newcommand{\A}{\mathbf{A}}
\newcommand{\B}{\mathbf{B}}
%\newcommand{\C}{\mathbf{C}}
\newcommand{\D}{\mathbf{D}}
%\newcommand{\G}{\mathbf{G}}
\newcommand{\I}{\mathbf{I}}
\newcommand{\Hb}{\mathbf{H}}
\newcommand{\M}{\mathbf{M}}
\newcommand{\Pb}{\mathbf{P}}
\newcommand{\Q}{\mathbf{Q}}
\newcommand{\W}{\mathbf{W}}
\newcommand{\V}{\mathbf{V}}
\newcommand{\X}{\mathbf{X}}
\newcommand{\Y}{\mathbf{Y}}
\newcommand{\Z}{\mathbf{Z}}
\newcommand{\p}{\mathbb{P}}
\newcommand{\E}{\mathbb{E}}
\newcommand{\R}{\mathbb{R}}
\newcommand{\q}{\mathbf{q}}
\newcommand{\DCal}{\mathcal{D}}
\newcommand{\LCal}{\mathcal{L}}
\newcommand{\RCal}{\mathcal{R}}
\newcommand{\SCal}{\mathcal{S}}
\newcommand{\XCal}{\mathcal{X}}
\newcommand{\YCal}{\mathcal{Y}}
\newcommand{\WCal}{\mathcal{W}}
\newcommand{\alphat}{\widetilde{\alpha}}
\newcommand{\betat}{\widetilde{\beta}}
\newcommand{\gammat}{\widetilde{\gamma}}
\newcommand{\phit}{\widetilde{\phi}}
\newcommand{\bt}{\widetilde{b}}
\newcommand{\alphabm}{{\bm{\alpha}}}
\newcommand{\betabm}{{\bm{\beta}}}
\newcommand{\gammabm}{{\bm{\gamma}}}
\newcommand{\deltabm}{{\bm{\delta}}}
\newcommand{\bmu}{\bm{\mu}}
\newcommand{\nubm}{\bm{\nu}}
\newcommand{\phibm}{\bm{\phi}}
\newcommand{\xibm}{\bm{\xi}}
\newcommand{\thetabm}{\bm{\theta}}
\newcommand{\Thetabm}{\mathbf{\Theta}}
\newcommand{\lambdabm}{\bm{\lambda}}
\newcommand{\Lambdabm}{\mathbf{\Lambda}}
\newcommand{\Omegabm}{\bm{\Omega}}
\newcommand{\Phibm}{\mathbf{\Phi}}
\newcommand{\one}{\mathbf{1}}
\newcommand{\zero}{\mathbf{0}}
\newcommand{\equiva}{\Leftrightarrow}
%\newcommand{\widebar}[1]{\mkern 1.5mu\overline{\mkern-1.5mu#1\mkern-1.5mu}\mkern 1.5mu}
%\newcommand{\widebar}[1]{\mkern 1.5mu\overline{\mkern-1.5mu#1\mkern-2.5mu}\mkern 3mu}
\newcommand{\widebar}[1]{\mkern 1.5mu\overline{\mkern-2.5mu#1\mkern-1.5mu}\mkern 1.5mu}

% madeness: suPer-script in Brackets
\newcommand{\pb}[1]{^{({#1})}}
\newcommand{\eg}{e.g.\ }
\newcommand{\ie}{i.e.\ }
\newcommand{\diag}{\text{diag}}
\newcommand{\downto}{\,\textbf{downto}\,}
\newcommand{\blue}[1]{{\color{blue}{#1}}}
\newcommand{\green}[1]{{\color{green}{#1}}}
\newcommand{\firstBest}[1]{\cellcolor{gray!50}{#1}}
\newcommand{\secondBest}[1]{\cellcolor{yellow!20}{#1}}

\newcommand{\TODO}{\blue{\bf{TODO:\ }}}
\newcommand{\DONE}{\green{\bf{DONE\ }}}
\newcommand{\cheng}[1]{\color{magenta}{\bf{Cheng:\ #1}}\color{black}}


% Copyright
%\setcopyright{none}
%\setcopyright{acmcopyright}
%\setcopyright{acmlicensed}
\setcopyright{rightsretained}
%\setcopyright{usgov}
%\setcopyright{usgovmixed}
%\setcopyright{cagov}
%\setcopyright{cagovmixed}


% DOI
%\acmDOI{10.475/123_4}

% ISBN
%\acmISBN{123-4567-24-567/08/06}

%Conference
\acmConference[RecSys'18]{ACM Conference on Recommender Systems}{October 2-7, 2018}{Vancouver, Canada}
\acmYear{2018}
\copyrightyear{2018}
%\acmArticle{4}
%\acmPrice{15.00}

% Remove the non-mandatory "ACM Reference format."
\fancyhead{}
\settopmatter{printacmref=false, printfolios=false}

\allowdisplaybreaks


\begin{document}
%\title{Music recommendation and playlist augmentation using multitask classification and ranking}
%\title{Augment Music Playlist via Multitask Classification and Ranking}
%\title{Music Recommendation via Multitask Learning}
%\title{Multitask Learning for Cold-Start Music Recommendation}
\title{Cold-start Music Recommendation via Multitask Learning}

%\author{Author Name}
%\affiliation{\institution{Institution Name}}
%\email{Email Address}

% The default list of authors is too long for headers.
%\renewcommand{\shortauthors}{B. Trovato et al.}

%
% The code below should be generated by the tool at
% http://dl.acm.org/ccs.cfm
% Please copy and paste the code instead of the example below.
%

\keywords{Cold start, Music recommendation, Multitask learning}

\begin{abstract}
% !TEX root=./main.tex

%Playlists are a core feature of music streaming services.
Playlist recommendation concerns producing a sequence of songs that a user might enjoy.
We investigate this problem in three different cold-start scenarios:
%Specifically, we investigate three settings with different cold items:
(i) \emph{cold songs}, where we recommend newly released songs to extend existing playlists;
(ii) \emph{cold playlists}, where we recommend a set of songs to form a new playlist for an existing user; %without additional context except the user;
(iii) \emph{cold users}, where we recommend a set of songs to form a new playlist for a new user. %, without any other context.

We propose a flexible multitask learning method to deal with all three settings.
The method learns from user-curated playlists,
%the %multitask learning
%method
and encourages songs in the playlist 
to be ranked higher than those are not
by minimising a %the Bottom-Push
bipartite ranking loss.
We formulate the objective as a constrained convex optimisation problem,
and show how this may be approximated by an unconstrained objective
%then address the difficulty of a large number of constraints by approximating the %Bottom-Push loss
%bipartite ranking loss
%with a classification loss
inspired by an equivalence relationship between bipartite ranking and binary classification.
Empirical results on two real music playlist datasets show the proposed approach has good performance for playlist recommendation
in cold-start settings.
%in three cold-start settings.

\end{abstract}

\maketitle

% !TEX root=main.tex

In recommendation tasks such as trajectory recommendation, it is desirable to avoid revisiting
a state or location that has already been visited before.

\begin{itemize}
	\item connect to workshop
	\item distinguish between next location vs whole trajectory
	\item define word usage: trajectory, path, walk, sequence, tour, etc.
	\item describe relation to travelling salesman, and say why different
	\item contributions of this paper
\end{itemize}

%\documentclass[10pt,a4paper]{article}
%\documentclass[twocolumn,10pt,a4paper]{article}
%\documentclass[twocolumn,a4wide,9pt]{extarticle}
\documentclass[twocolumn,9pt]{extarticle}
%\usepackage[a4paper,top=0.85in,left=0.75in,bottom=1in,right=0.52in]{geometry} % A4 paper margins
\usepackage[a4paper,top=0.75in,left=0.55in,bottom=0.9in,right=0.5in]{geometry} % A4 paper margins
\usepackage{hyperref}
\usepackage{amsmath}
\usepackage{amsfonts}
\usepackage{bm}
\usepackage{bbm}
\usepackage{algorithm}
\usepackage{algorithmic}
\usepackage[sc]{mathpazo}
\linespread{1.05}         % Palladio needs more leading (space between lines)
\usepackage[T1]{fontenc}

\DeclareMathOperator*{\argmin}{argmin}
\DeclareMathOperator*{\argmax}{argmax}
\newcommand{\eat}[1]{}
\setlength{\columnsep}{1.5em} % spacing between columns

\title{The Trajectory Recommendation Problem}

\author{Dawei Chen}

\date{\today}

\begin{document}

\maketitle


\section{Problem formulation}
\label{sec:formulation}

Given a set of point of interest (POI) $\mathcal{P}$ and a trajectory query $\mathbf{x} = (s, k)$,
where $s \in \mathcal{P}$ is the desired start POI and $k$ is the number of POIs in the desired trajectory (including the start location $s$).
We want to recommend a sequence of POIs $\mathbf{y}^*$ that maximises utility, i.e.,
\begin{equation*}
\mathbf{y}^* = \argmax_{\mathbf{y}}~f(\mathbf{x}, \mathbf{y}),
\end{equation*}
where $\mathbf{y} = (y_1 = s,~ y_2, \dots, y_k),~ y_i \in \mathcal{P},~ i=1,\dots,k$, and $y_i \ne y_j$ if $i \ne j$.



\section{Related problems}
\label{sec:related}

This problem is related to automatic playlist generation, 
where we recommend a sequence of songs given a specified song (a.k.a. seed) and the number of new songs.

Another similar problem is choosing a small set of photos from a large photo library and compiling them into a slideshow or movie.



\section{Proposed methods}
\label{sec:methods}

We can solve the problem using simple models such as linear regression, logistic regression and learning to rank,
moreover, sophisticated approaches such as structured models (probabilistic or non-probabilistic) can also be employed.
In this section, we describe these methods briefly.

The training set contains $N$ trajectories 
$\{ \mathbf{x}^{(i)}, \mathbf{y}^{(i)} \}_{i=1}^N$,
where $\mathbf{y}^{(i)}$ is the $i$-th trajectory and $\mathbf{x}^{(i)} = (y_1,~ \mid \mathbf{y}^{(i)} \mid)$ is the query 
with respect to trajectory $\mathbf{y}^{(i)}$.



\subsection{Simple models}
\label{sec:simple}

We can model each POI in the desired trajectory independently, which leads to a number of straightforward methods.



\subsubsection{Linear regression}
\label{sec:linear}

We can use linear regression to directly model the rank of a POI $p \in \mathcal{P}$ w.r.t. a query.
First, we construct the feature vectors $\Psi$ and labels (i.e., ranks) $R$ for trajectories in training set, 
\begin{align*}
\Psi &= \left( \Psi(\mathbf{x}^{(i)}, p) \right)_{i=1,\dots,N,~p \in \mathcal{P}} \in \mathbb{R}^{(N \cdot \mid \mathcal{P} \mid) \times D}, \\
   R &= \left( r(\mathbf{y}^{(i)}, p \right)_{i=1,\dots,N,~p \in \mathcal{P}}     \in \mathbb{R}^{(N \cdot \mid \mathcal{P} \mid) \times 1},
\end{align*}
where $D$ is the dimension of feature vector and $r(\mathbf{y}, p)$ is the (normalised) location or rank of POI $p$ in a trajectory $\mathbf{y}$,
\begin{align*}
r(\mathbf{y}, p) &= \sum_{j=1}^{\mid \mathbf{y} \mid} j \cdot \mathbbm{1}(y_j = p) ~~~\text{or} \\
r(\mathbf{y}, p) &= \frac{1}{\mid \mathbf{y} \mid} \sum_{j=1}^{\mid \mathbf{y} \mid} j \cdot \mathbbm{1}(y_j = p),
\end{align*}
and $\mathbbm{1}(\cdot)$ is the indicator function.

To learn the parameters $\mathbf{w}$, we need to solve linear equations $\Psi \cdot \mathbf{w} = R$,
which is straightforward, i.e., $\mathbf{w} = \Psi^{-1} R$ or $\mathbf{w} = \Psi \backslash R$ by using matrix left division.

To recommend a trajectory given a query, we simply choose the top (first) $k-1$ POIs from $\mathcal{P} \setminus s$.
We note that when two different trajectories satisfy the same query, the feature vectors for all POIs will be the same (see Section~\ref{sec:feature})
but the labels (i.e., ranks) can be different, which may confuse this model.



\subsubsection{Logistic regression}
\label{sec:logistic}

In addition, if we consider whether a POI appeared in a trajectory but ignore its location, i.e., construct the labels as
\begin{equation*}
l_p^i = \begin{cases}
+1,~p \in \mathbf{y}^{(i)}, \\
-1,~p \notin \mathbf{y}^{(i)}.
\end{cases}
\end{equation*}
and train a logistic regression model 
%\begin{equation*}
%h_\mathbf{w}(p \mid \mathbf{x}) = \frac{1}{1 + \exp \left(- \mathbf{w}^\top \Psi(\mathbf{x}, p) \right)}.
%\end{equation*}
%To learn the parameters, we solve the following optimisation problem
\begin{equation*}
\min_{\mathbf{w}} \frac{1}{2} \mathbf{w}^\top \mathbf{w} + 
C \sum_{i=1}^N \sum_{p \in \mathcal{P}} \log \left(1 + \exp \left(- l_p^i \cdot \mathbf{w}^\top \Psi_p^i \right) \right),
\end{equation*}
where $C>0$ is a regularisation constant and we denote $\Psi(\mathbf{x}^{(i)}, p)$ as $\Psi_p^i$ for brevity.

The probability of a POI given a query is
$\mathbb{P}(p \mid \mathbf{x}; \mathbf{w}) = \sigma \left( \mathbf{w}^\top \Psi(\mathbf{x}, p) \right)$
where $\sigma(\cdot)$ is the logistic function, we choose the $k-1$ mostly likely POIs from $\mathcal{P} \setminus s$ 
to form the recommendation.



\subsubsection{RankSVM}
\label{sec:rank}

On the other hand, we can learn a ranking model.
Let $c_p^i$ denote the number of times POI $p$ was observed in trajectories satisfying query $\mathbf{x}^{(i)}$ (except the start POI),
and define 
\begin{equation*}
l_p^{ij} = \begin{cases}
+1,~ c_p^i > c_p^j, \\
-1,~ c_p^i < c_p^j.
\end{cases}
\end{equation*}

We learn the parameters by training a rankSVM,
\begin{equation*}
\min_{\mathbf{w}} \frac{1}{2} \mathbf{w}^\top \mathbf{w} +  
C \sum_{i=1}^N \sum_{p_i, p_j \in \mathcal{P}} \max\left(0,~ 1 - l_p^{ij} \cdot \mathbf{w}^\top \left( \Psi_p^i - \Psi_p^j \right) \right).
\end{equation*}

The rank of a POI given a query is $R(p \mid \mathbf{x}; \mathbf{w}) = \mathbf{w}^\top \Psi(\mathbf{x}, p)$,
and we simply choose the top $k-1$ POIs from $\mathcal{P} \setminus s$ to form a recommendation.
RankSVM models the rank of a POI given a query, but ignores the transition preference between different POIs.



\subsection{Structured models}
\label{sec:structured}

We can model the dependences between different POIs in a trajectory by employing structured prediction models,
either probabilistic models such as maximum-entropy Markov models (MEMM) and conditional random fields (CRF),
or non-probabilistic model such as structured SVM.
We model the desired trajectory with respect to query $\mathbf{x}$ as a sequence of discrete variables, 
with the first variable being observed, and each variable has $|\mathcal{P}|$ states.
To make a recommendation, we find a trajectory that achieves the highest score
\begin{equation*}
\mathbf{y}^* = \argmax_{\mathbf{y} \in \mathcal{Y}}~ f(\mathbf{x}, \mathbf{y}),
\end{equation*}
where $\mathcal{Y}$ is the set of all possible trajectory with POIs in $\mathcal{P}$ and satisfies query $\mathbf{x}$,
$f(\mathbf{x}, \mathbf{y})$ is a function that scores the compatibility between query $\mathbf{x}$ and a specific trajectory $\mathbf{y}$.



\subsubsection{Maximum-entropy Markov models}
\label{sec:memm}

For MEMM, the compatibility function $f(\mathbf{x}, \mathbf{y})$ is the probability of trajectory $\mathbf{y}$ given query $\mathbf{x}$,
\begin{equation*}
f(\mathbf{x}, \mathbf{y}) = \mathbb{P}(\mathbf{y} \mid \mathbf{x}; \mathbf{w}) 
                          = \prod_{j=2}^{\mid \mathbf{y} \mid}~
                            \frac{\exp \left(\mathbf{w}^\top \Psi_j(\mathbf{x}, y_{j-1}, y_j) \right)}
                                 {1 + \exp \left(\mathbf{w}^\top \Psi_j(\mathbf{x}, y_{j-1}, y_j) \right)}.
\end{equation*}

The negative log-likelihood is 
\begin{equation*}
\ell(\mathbf{w}) = -\sum_{i=1}^N \log \mathbb{P}(\mathbf{y}^{(i)} \mid \mathbf{x}^{(i)}; \mathbf{w}).
\end{equation*}

To learn the parameters, we minimise the negative log-likelihood with L2 regularisation (maximum likelihood estimation (MLE))
\begin{equation*}
\min_{\mathbf{w}} \frac{1}{2} \mathbf{w}^\top \mathbf{w} + C \ell(\mathbf{w}).
\end{equation*}


\subsubsection{Conditional random fields}
\label{sec:crf}

For CRF, the compatibility function $f(\mathbf{x}, \mathbf{y})$ is also the probability of trajectory $\mathbf{y}$ given query $\mathbf{x}$,
\begin{align*}
f(\mathbf{x}, \mathbf{y}) = \mathbb{P}(\mathbf{y} \mid \mathbf{x}; \mathbf{w}) 
&= \frac{\exp \left( \mathbf{w}^\top \Psi(\mathbf{x}, \mathbf{y}) \right)}
        {\sum_{\mathbf{y}'} \exp \left( \mathbf{w}^\top \Psi(\mathbf{x}, \mathbf{y}') \right)} \\
&= \frac{\prod_{j=2}^{\mid \mathbf{y} \mid} \exp \left( \mathbf{w}^\top \Psi_j(\mathbf{x}, y_{j-1}, y_j) \right)}
       {\sum_{\mathbf{y}'} \prod_{j=2}^{\mid \mathbf{y}' \mid} \exp \left( \mathbf{w}^\top \Psi_j(\mathbf{x}, y_{j-1}', y_j') \right)},
\end{align*}
assuming decomposition and parameter tying
\begin{equation*}
\mathbf{w}^\top \Psi(\mathbf{x}, \mathbf{y}) = \sum_{j=2}^{\mid \mathbf{y} \mid} \mathbf{w}^\top \Psi_j(\mathbf{x}, y_{j-1}, y_j).
\end{equation*}

The negative log-likelihood is 
\begin{equation*}
\ell(\mathbf{w}) = -\sum_{i=1}^N \log \mathbb{P}(\mathbf{y}^{(i)} \mid \mathbf{x}^{(i)}; \mathbf{w}).
\end{equation*}

To learn the parameters, we minimise the negative log-likelihood with L2 regularisation (MLE)
\begin{equation*}
\min_{\mathbf{w}} \frac{1}{2} \mathbf{w}^\top \mathbf{w} + C \ell(\mathbf{w}).
\end{equation*}



\subsubsection{Structured SVM}
\label{sec:ssvm}

For structured SVM, the compatibility function $f(\mathbf{x}, \mathbf{y})$ is this linear form,
\begin{equation*}
f(\mathbf{x}, \mathbf{y}) = \mathbf{w}^\top \Psi(\mathbf{x}, \mathbf{y}),
\end{equation*}
where the $\Psi(\mathbf{x}, \mathbf{y})$ is a \emph{joint feature map} 
which captures features extracted from both query $\mathbf{x}$ and trajectory $\mathbf{y}$.

The design of joint feature $\Psi(\cdot)$ is problem specific, 
in the setting of trajectory recommendation,
assuming decomposition and parameter tying, we have
\begin{equation*}
\label{eq:jointfeature}
\mathbf{w}^\top \Psi(\mathbf{x}, \mathbf{y}) = \sum_{j=2}^{\mid \mathbf{y} \mid} 
                                               \left( \mathbf{w}_1^\top \Psi_j(\mathbf{x}, y_j) + 
                                                      \mathbf{w}_2^\top \Psi_{j-1, j}(\mathbf{x}, y_{j-1}, y_j) \right),
\end{equation*}
where $\mathbf{w}_1$ and $\mathbf{w}_2$ are parameter vectors,
$\Psi_j$ is a feature vector of POI $y_j$ with respect to query $\mathbf{x}$ (Table~\ref{tab:poifeature}),
$\Psi_{j-1,j}$ is a pairwise feature vector that captures the affinity of transition from POI $y_{j-1}$ to POI $y_j$ and
here we use the transition probabilities between individual POI properties as described in Table~\ref{tab:tranfeature}.
This joint feature design shares parameters among POIs/transitions in a trajectory.

To learn the parameters, we train the structured SVM by optimising a quadratic program (QP),
\begin{equation}
\label{eq:nslackform}
\begin{aligned}
\min_{\mathbf{w}, ~\bm{\xi} \ge 0} ~& \frac{1}{2} \mathbf{w}^\top \mathbf{w} + \frac{C}{n} \sum_{i=1}^n \xi_i \\
s.t.~~ ~& \mathbf{w}^\top \Psi(\mathbf{x}^{(i)}, \mathbf{y}^{(i)}) - \mathbf{w}^\top \Psi(\mathbf{x}^{(i)}, \bar{\mathbf{y}}) \ge 
       \Delta(\mathbf{y}^{(i)}, \bar{\mathbf{y}}) - \xi_i, ~\forall i
\end{aligned}
\end{equation}
where $\mathbf{w} = [\mathbf{w}_1, \mathbf{w}_2]^\top$ is the parameter vector, $C > 0$ is a regularisation constant, 
and $\Delta(\mathbf{y}, \bar{\mathbf{y}})$ is a discrepancy function that measures the loss 
for prediction $\bar{\mathbf{y}}$ given ground truth $\mathbf{y}$, and $\xi_i$
is a slack variable that represents the \emph{hinge loss} associated with the prediction for the $i$-th example~\cite{tsochantaridis2005large},
\begin{equation*}
\label{eq:nslackloss}
\xi_i = \max \left( 0,~ 
        \max_{\bar{\mathbf{y}} \in \mathcal{Y}} 
        \left\{ \Delta(\mathbf{y}_i, \bar{\mathbf{y}}) + \mathbf{w}^\top \Psi(\mathbf{x}^{(i)}, \bar{\mathbf{y}}) \right\} -
        \mathbf{w}^\top \Psi(\mathbf{x}^{(i)}, \mathbf{y}^{(i)}) \right).
\end{equation*}
%This formulation is called "$n$-slack" as we have one slack variable for each example in training set.

We can rewrite the constraint of problem (\ref{eq:nslackform}) as
\begin{equation*}
\mathbf{w}^\top \Psi(\mathbf{x}^{(i)}, \mathbf{y}^{(i)}) + \xi_i \ge
          \max_{\bar{\mathbf{y}} \in \mathcal{Y}} 
          \left\{\mathbf{w}^\top \Psi(\mathbf{x}^{(i)}, \bar{\mathbf{y}}) + \Delta(\mathbf{y}^{(i)}, \bar{\mathbf{y}}) \right\},~ \forall i.
\end{equation*}
where the right hand side is the \emph{loss-augmented inference}.

To solve problem (\ref{eq:nslackform}), one option is simply enumerating all constraints, feeding the problem into a standard QP solver.
However, this approach is impractical as there is a constraint for every possible label $\bar{\mathbf{y}}$.
Instead, a cutting-plane algorithm is employed which repeatedly solves QP (\ref{eq:nslackform}) with respect to different set of constraints, 
and each iteration solves the loss-augmented inference and generates a new constraint that helps shrink the feasible region of the problem, 
until a specified precision $\varepsilon$ is achieved~\cite{joachims2009predicting}.

The loss-augmented inference for trajectory recommendation is equivalent to 
find a maximum weighted loop-less path with exactly $k$ edges in a complete weighted (both nodes and edges) graph, which is NP-hard (need proof).
To solve the loss-augmented inference, we can formulated it as an integer linear program (ILP) and solve it using a ILP solver,
or use lazy constraint generation/cutting plane technique with an LP solver.
Moreover, we can use list Viterbi algorithm~\cite{nill1995list} or 
employ heuristics such as the Christofides algorithm~\cite{christofides1976} when the problem has the triangle inequality property 
(which is indeed for trajectories).


\eat{
\subsection{Other models}
\label{sec:other}
Label ranking model,
Plackett-Luce probabilistic ranking


\section{Evaluation metrics}
\label{sec:evaluation}
F1 score on points
F1 score on pairs
Kendall's tau, all POIs not appeared in trajectory are ranked last (and share the same rank).
}


\begin{table*}[ht]
\caption{Features of POI $p$ with respect to query $(s,k)$}
\label{tab:poifeature}
\centering
\setlength{\tabcolsep}{10pt} % tweak the space between columns
\begin{tabular}{l|l} \hline
\textbf{Feature}  & \textbf{Description} \\ \hline
\texttt{category}               & one-hot encoding of the category of $p$ \\
\texttt{neighbourhood}          & one-hot encoding of the POI cluster that $p$ resides in \\
\texttt{popularity}             & logarithm of POI popularity of $p$ \\
\texttt{nVisit}                 & logarithm of the total number of visit by all users at $p$ \\
\texttt{avgDuration}            & logarithm of the average duration at $p$ \\ \hline
\texttt{trajLen}                & trajectory length $k$, i.e., the number of POIs required \\
\texttt{sameCatStart}           & $1$ if the category of $p$ is the same as that of $s$, $-1$ otherwise \\
\texttt{sameNeighbourhoodStart} & $1$ if $p$ resides in the same POI cluster as $s$, $-1$ otherwise \\
\texttt{distStart}              & distance between $p$ and $s$, calculated using the Haversine formula \\
\texttt{diffPopStart}           & real-valued difference in POI popularity of $p$ from that of $s$ \\
\texttt{diffNVisitStart}        & real-valued difference in the total number of visit at $p$ from that at $s$ \\
\texttt{diffDurationStart}      & real-valued difference in average duration at $p$ from that at $s$ \\
\hline
\end{tabular}
\end{table*}


\section{Features}
\label{sec:feature}

The POI and query specific features we extracted from trajectories are shown in Table~\ref{tab:poifeature},
features that describe the transition preference between different POIs are shown in Table~\ref{tab:tranfeature}.


\begin{table}[ht]
\caption{POI features used to estimate the (feature-wise) transition probabilities}
\label{tab:tranfeature}
\centering
%\setlength{\tabcolsep}{28pt} % tweak the space between columns
\begin{tabular}{l|l} \hline
\textbf{Feature}       & \textbf{Description} \\ \hline
\texttt{category}      & category of POI \\
\texttt{neighbourhood} & the cluster that a POI resides in \\
\texttt{popularity}    & (discretised) popularity of POI \\
\texttt{nVisit}        & (discretised) total number of visit at POI \\
\texttt{avgDuration}   & (discretised) average duration at POI \\ \hline
\end{tabular}
\end{table}



\bibliographystyle{ieeetr}
\bibliography{ref}

\end{document}

\section{Multitask learning for music recommendation}
\label{sec:method}

In this section, we first describe a multitask objective that supports the cold-start recommendation tasks
described in Section~(\ref{sec:problem}). We propose a ranking approach to optimise the multitask objective 
by solving an convex constrained optimisation problem.
We analyse the challenge to deal with a large number of constraints and then propose an classification approach
that approximates the multitask objective without any constraints, which results in an efficient optimisation of
the multitask objective.


\subsection{Multitask learning objective}

We decompose the representation $\w_i^u$ of a playlist $i$ from user $u$ in (\ref{eq:scorefunc}) into three components, \ie
$$
\w_i^u = \bv_i + \bmu_u + \widebar\bmu,
$$
where $\bv_i$ is a parameter vector specific for playlist $i$,
$\bmu_u$ is the component for user $u$,
and $\widebar\bmu$ is a representation shared by all users.

The learning task is to minimise the empirical risk of affinity function $f$ on dataset $\DCal$ over all parameters.
Let $\uptheta$ which represents all parameters in $\bv_i, \bmu_u, \widebar\bmu, \, u \in \{1,\dots,U\}, \, i \in P_u$.
Formally, we solve an optimisation problem
\begin{equation}
\label{eq:obj}
\min_{\uptheta} \, R(\uptheta) + \RCal(f, \DCal),
\end{equation}
where $R(\uptheta)$ is the regularisation term and $\RCal(f, \DCal)$ denotes the empirical risk.
We call the objective in problem~(\ref{eq:obj}) a multitask learning objective,
since jointly learn from multiple tasks where each task is to recommend a set of songs given either a user or a playlist.

We further assume that playlists of the \emph{same} user have \emph{similar} representations,
%and the difference between playlists $i$ and $i'$ of the same user $j$ is reflected by the playlist component $\bv_i$
%and $\bv_{i'}$, respectively.
%This assumption requires that parameters specific to playlist should be sparse,
which can be achieved by imposing an L1 regularisation, \ie $\sum_{u=1}^U \sum_{i \in P_u} \| \bv_i \|_1$
that encourages sparse representations.
Lastly, although the representations of different users are different,
we assume they nonetheless share a small number of features in their representations.
This can be formalised similarly by imposing an additional L1 regularisation term $\| \widebar\bmu \|_1$.

The regularisation terms in our multitask learning objective can be summarised as
\begin{equation}
\label{eq:reg}
R(\uptheta) = \lambda_1 \sum_{u=1}^U \sum_{i \in P_u} \|\bv_i\|_1 + \lambda_2 \sum_{u=1}^U \|\bmu_u\|_2^2 + \lambda_3 \|\widebar\bmu\|_1,
\end{equation}
where we add an L2 regularisation term to penalise large values in user representations,
and $\lambda_1, \lambda_2, \lambda_3 \in \R_+$ are regularisation constants.

Once all parameters have been learned, in the task of recommending new songs to extend an existing playlist $i$ from user $u$, 
we can score a new song $m$ by
$$
\hat f(u, i, m) = (\bv_i + \bmu_u + \widebar\bmu)^\top \x_m.
$$

Further, in the task of recommending a set of songs for an existing user $u$ (to form a playlist), 
we score song $m$ by
$$
\hat f(u, \cdot, m) = (\bmu_u + \widebar\bmu)^\top \x_m.
$$

Finally, to recommend a set of songs for a new user, each song $m$ can be scored by
$$
\hat f(\cdot, \cdot, m) = \widebar\bmu^\top \x_m.
$$


\subsection{Recommending the most probable songs}
\label{ssec:bploss}

In this section, we describe a ranking approach that learns to rank songs in playlist above
those not in it, which aims to rank the most probable songs above those unlikely when making a recommendation.
The corresponding loss function of this approach is known as the Bottom-Push loss~\cite{rudin2009p} in bipartite ranking literature.
%Formally, for playlist $i$, we would like
%\begin{equation*}
%\begin{aligned}
%\min_{m: y_m^i = 1} f(i, m) \ge f(i, n), \ \forall n \in \{1,\dots,M\} \ \text{and} \ y_n^i = 0,
%\end{aligned}
%\end{equation*}
%where $y_m^i = 1$ denotes that song $m$ is in playlist $i$,
%and $y_n^i = 0$ represents song $n$ does not appear in playlist $i$.
Formally, we minimise the number of songs not in playlist but ranked above the lowest ranked song in playlist,
%but have a higher score than the lowest ranked song in playlist, 
\ie, the (normalised) empirical risk is
\begin{equation}
\label{eq:bprisk}
\resizebox{.9\linewidth}{!}{$\displaystyle
\RCal_\textsc{rank}(\uptheta) = \frac{1}{N} \sum_{u=1}^U \sum_{i \in P_u} \frac{1}{M_-^i} \sum_{m': y_{m'}^i = 0} 
\llb \min_{m: y_m^i = 1} f(u, i, m) \le f(u, i, m') \rrb,
$}
\end{equation}
where $M_-^i$ denotes the number of songs that are not in playlist $i$,
and $\llb \cdot \rrb$ is the indicator function that represents the 0-1 loss.

As a remark, it is easy to see that the order of songs in a specific playlist does not affect the empirical 
risk~(\ref{eq:bprisk}) as long as they are scored higher than the lowest scored song in that playlist.
The optimisation problem~(\ref{eq:obj}) now become the following regularised risk minimisation:
\begin{equation}
\label{eq:obj_rank}
\min_\uptheta \ R(\uptheta) + \RCal_\textsc{rank}(\uptheta).
\end{equation}
%where the regularisation term $R(\uptheta)$ is defined in (\ref{eq:reg}).

There are two challenges to optimise the above objective,
namely, the non-differentiable 0-1 loss and the \emph{min} operator in $\RCal_\textsc{rank}(\uptheta)$.
To address these challenges, we first replace the 0-1 loss with one of its convex surrogate $\ell(f, y)$,
(\eg, the exponential loss $\ell(f, y) = e^{-fy}$, the logistic loss $\ell(f, y) = \log(1 + e^{-fy})$,
or the squared hinge loss $\ell(f, y) = [\max(0, \, 1 - fy)]^2$).
%which results in
%\begin{equation}
%\label{eq:obj}
%\min_\uptheta \ R(\uptheta) + \frac{1}{N} \sum_{i=1}^U \sum_{j \in P_i} \frac{1}{M_-^j} \sum_{m': y_{m'}^j = 0} 
%                \ell \left( \min_{m: y_m^j = 1} f(m, j) - f(m', j) \right).
%\end{equation}



\subsection{Solving a constrained optimisation problem}

Suppose we use the exponential loss to upper-bound the 0-1 loss in $\RCal_\textsc{rank}$,
the multitask learning objective~(\ref{eq:obj_rank}) can be optimised by
\begin{equation}
\label{eq:expobj_rank}
\resizebox{.9\linewidth}{!}{$\displaystyle
\min_{\uptheta} \ R(\uptheta) + \frac{1}{N} \sum_{u=1}^U \sum_{i \in P_u} \frac{1}{M_-^i} 
                  \sum_{m': y_{m'}^i = 0} \exp \left(f(u, i, m') - \min_{m: y_m^i = 1} f(u, i, m) \right).
$}
\end{equation}

%Directly solving problem (\ref{eq:expobj}) is challenging due to the \emph{min} operator.
To deal with the challenge imposed by the \emph{min} operator in the exponential function, 
we reformulate problem (\ref{eq:expobj_rank}) into a constrained optimisation problem by 
introducing slack variables $\xi_i, \, i \in P_u, \, u \in \{1,\dots,U\}$,
\begin{equation}
\label{eq:expobj_cons}
\begin{aligned}
\min_{\uptheta} \ \, & R(\uptheta) + \frac{1}{N} \sum_{u=1}^U \sum_{i \in P_u} \frac{1}{M_-^i} \sum_{m': y_{m'}^i = 0} e^{f(u, i, m') - \xi_i} \\
s.t. \ \, & \xi_i \le f(u, i, m), \ \forall m \in \{1,\dots,M\} \ \text{and} \ y_m^i = 1.
\end{aligned}
\end{equation}

One may observe that the objective of problem~(\ref{eq:expobj_cons}) is convex but not differentiable due to 
the L1 regularisation terms in $R(\uptheta)$, nonetheless, we can use its sub-gradient.
Further, the number of constraints in problem~(\ref{eq:expobj_cons}) is
$$
\sum_{u=1}^U \sum_{i \in P_u} \sum_{m=1}^M \llb y_m^i = 1 \rrb,
$$
in other words, the number of constraints equals the total number of songs played in all playlists.
Asymptotically, it is of order $O(\widebar{L} N)$ where $\widebar{L}$ is the average number of songs in playlists.
Although $\widebar{L}$ is dataset dependent, and is typically less than $100$,
the total number of playlists $N$ can be very large in production systems 
(\eg, Spotify hosts more than $2$ billion playlists~\cite{recsysch2018}),
which imposes significant challenge when optimising problem~(\ref{eq:expobj_cons}).

We want to mention two techniques that can be employed to alleviate this issue:
the cutting-plane~\cite{avriel2003nonlinear} method and the sub-gradient method.
We have found both approaches converge extremely slow in practice for this problem, 
in particular, the cutting plane method is required to deal with constrained optimisation problems 
with at least $N$ constraints, which is still challenging when $N$ is large.
%but the minimum number of constraints is still in the order of $O(N)$, and we also found these techniques do not work extremely well in practice.

In this paper, we address this issue by formulating an unconstrained optimisation problem which approximates 
the objective in problem~(\ref{eq:expobj_cons}), by leveraging an equivalence relationship between bipartite 
ranking and binary classification~\cite{ertekin2011equivalence}, we describe this approach in the following section.

%\cheng{This still sounds hard. So what do we do next? Unclear why we suddenly talk about binary classification right after this.}



%\subsection{From ranking to classification}
\subsection{Unconstrained optimisation with classification loss}

It has been known that binary classification and bipartite ranking are
closely related~\cite{ertekin2011equivalence,menon2016bipartite}.
In particular, \citet{ertekin2011equivalence} have shown that the P-Norm Push bipartite ranking loss~\cite{rudin2009p}
is equivalent to the P-Classification loss~\cite{ertekin2011equivalence} when using the exponential surrogate.
Further, the P-Norm Push loss is an approximation of the Infinite-Push loss~\cite{agarwal2011infinite},
or equivalently, the Top-Push loss~\cite{li2014top}, which focuses on the highest ranked negative example instead of
of the lowest ranked positive example as in~(\ref{eq:bprisk}).

Inspired by these connections, we first seek a bipartite ranking loss that approximates the Bottom-Push loss in~(\ref{eq:bprisk}),
then propose a classification loss that is equivalent to this bipartite ranking loss.
The reason not to directly optimise the bipartite ranking loss is due to computation cost,
and classification loss can be optimised more efficiently in general~\cite{ertekin2011equivalence}.

We can approximate the \emph{min} operator by utilising the well known Log-sum-exp approximation 
of the \emph{max} operator~\cite[p. 72]{boyd2004convex},
\begin{equation}
\label{eq:minappox}
\min_i z_i = -\max_i (-z_i) \approx -\frac{1}{p} \log \sum_i e^{-p z_i},
\end{equation}
where $p > 0$ is a parameter that trades off the approximation precision.
This approximation becomes precise when $p \to \infty$.

By Eq.~(\ref{eq:bprisk}) and (\ref{eq:minappox}), the empirical risk $\RCal_\textsc{rank}$ can be approximated
(with the exponential surrogate) by $\widetilde\RCal_\textsc{rank}$ defined as
\begin{equation}
\label{eq:rankapprox}
\resizebox{.9\linewidth}{!}{$
\begin{aligned}
\widetilde\RCal_\textsc{rank}(\uptheta)
= \frac{1}{N} \sum_{u=1}^U \sum_{i \in P_u} \frac{1}{M_-^i} \left( \sum_{m: y_m^i = 1} \left( \sum_{m': y_{m'}^i = 0} 
  e^{-(f(u, i, m) - f(u, i, m'))} \right)^p \right)^\frac{1}{p}.
\end{aligned}
$}
\end{equation}


One may note that $\widetilde\RCal_\textsc{rank}$ can be transformed into the standard P-Norm Push loss by swapping the
positives ($m: y_m^i = 1$) and negatives ($m': y_{m'}^i = 0$). % define P-Norm Push first?
Inspired by this observation, we swap the positives and negatives in the P-Classification loss (by taking care of signs),
which results in the following classification risk:
\begin{equation}
\label{eq:clfrisk}
\resizebox{.9\linewidth}{!}{$\displaystyle
\RCal_\textsc{clf}(\uptheta)
= \frac{1}{N} \sum_{u=1}^U \sum_{i \in P_u} \left(
  \frac{1}{p M_+^i} \sum_{m: y_m^i = 1} e^{-p f(u, m, i)}
  + \frac{1}{M_-^i} \sum_{m': y_{m'}^i = 0} e^{f(u, m', i)} \right).
$}
\end{equation}

We have the following lemma:
\begin{lemma}
\label{lm:rank2clf}
Let $\uptheta^* \in \argmin_{\uptheta} \RCal_\textsc{clf}$ (assuming minimisers exist),
then $\uptheta^* \in \argmin_{\uptheta} \widetilde\RCal_\textsc{rank}$.
\end{lemma}

\begin{proof}
This theorem can be proved by swapping the positives and negatives in the proof of 
the equivalence between P-Norm Push loss and P-Classification loss~\cite{ertekin2011equivalence}.
A complete proof from first principles is described in Appendix.
\end{proof}

We can therefore create an unconstrained optimisation problem using the classification risk $\RCal_\textsc{clf}$:
\begin{equation}
\label{eq:expobj_clf}
\min_\uptheta \quad R(\uptheta) + \RCal_\textsc{clf}(\uptheta).
\end{equation}

The objective in problem~(\ref{eq:expobj_clf}) is convex but not differentiable due to the L1 regularisation terms in $R(\uptheta)$,
but it can be efficiently solved using the Orthant-Wise Limited-memory Quasi-Newton (OWL-QN)~\cite{andrew2007scalable} L-BFGS variant.

We refer the approach that solves problem~(\ref{eq:expobj_clf}) as \emph{Multitask Classification} in experiment.
As a remark, the optimal solutions of problem (\ref{eq:expobj_clf}) are not necessarily the optimal solutions 
of problem $\min_\uptheta \ R(\uptheta) + \widetilde\RCal_\textsc{rank}(\uptheta)$ due to the regularisation terms,
however, when parameters $\uptheta$ are small (which is generally the case when using regularisation), the two objectives 
can nonetheless approximate each other in acceptable level.


%\cheng{Explain in words why one is ranking, and the other one classification.}

%\cheng{New subsection: Describe the difference between existing and new user.}

%\subsection{Discussion}

%In this section, we want to discuss the advantages and limitations of the
%two approaches that solve problem~(\ref{eq:expobj}),
%as well as a few practical strategies for the proposed approaches.
%
%The constrained optimisation problem~(\ref{eq:expobj_cons}) can be solved using interior-point method,
%although the large number of constraints requires a large amount of computation resources.
%Cutting-plane~\cite{avriel2003nonlinear} techniques can be employed to deal with this challenge,
%but we found it does not work extremely well in practice.
%However, this approach allows us to use other surrogate losses besides the exponential loss,
%such as squared hinge loss, which may provide a few other desired properties. % detailing this
% solvers: IPOPT, CVX etc.

%To solve the unconstrained optimisation problem~(\ref{eq:expobj_clf}),
%the Orthant-Wise Limited-memory Quasi-Newton (OWL-QN)~\cite{andrew2007scalable} L-BFGS variant can be employed to
%efficiently deal with the L1 regularisation term without imposing huge memory burdens.
% solver: pylbfgs

%\cheng{Justify why we need to compare two methods empirically: multitask classification, multitask ranking.}

%\section{Related work}
%\label{sec:related}

%\cheng{Make this a subsection of the multitask learning section.}

We first summarise recent work of recommending music to form playlists
according the problem settings, the recommendation methods,
as well as the information being utilised,
and then music recommendation techniques in cold-start scenarios.

%\cheng{Reduce related work subsection to 1 column}

{\bf Problem settings}.
There are three typical settings:
playlist generation, next song recommendation, and playlist continuation.
%
There is rich collection of recent literature on the playlist generation or prediction
problem~\cite{platt2002learning,mcfee2011natural,mcfee2012hypergraph,chen2012playlist,ben2017groove}.
%
Typical work is to generate a complete playlist given some seed,
for example, the AutoDJ system~\cite{platt2002learning} generates playlists given one or more seed songs,
a playlist for a specific user can be generated by Groove Radio given a seed artist~\cite{ben2017groove},
or a seed location in hidden space, where all songs are embedded,
should be specified in order to generate a complete playlist~\cite{chen2012playlist}.
%
There are also work that focus on evaluating the learned playlist model,
without concretely generate playlist~\cite{mcfee2011natural,mcfee2012hypergraph}.
More details of playlist generation can be found in this recent survey~\cite{bonnin2015automated}.


Next song recommendation~\cite{hariri2012context,bonnin2013evaluating,jannach2015beyond}
is to predict the next song a user might play after observing some context,
for example, the most recent sequence of songs a user interacted with the system was used to
infer the contextual information, which was further used to rank the next possible song
with regards to a topic-based sequential patterns learned from user playlists~\cite{hariri2012context}.
%
The artists appeared in user's listening history can be used as context,
which, together with the popularity of song or frequency of artists collocations,
were further used to score the next song~\cite{mcfee2012million,bonnin2013evaluating}.
%
It is obvious that next song recommendation techniques can also be used to generate a
complete playlist by picking the next song sequentially~\cite{bonnin2013evaluating,ben2017groove}.


Playlist continuation is to add one or more songs to a playlist,
given the added songs serve the same purpose of the original playlist~\cite{schedl2017,recsysch2018}.
One can immediately notice that playlist generation and next song recommendation are special cases
of playlist continuation, which means techniques developed for playlist generation and
next song recommendation can both be used for playlist continuation.

In the collaborative filtering literature,
the cold-start setting has primarily been addressed through
suitable regularisation of matrix factorisation parameters
based on exogenous user- or item-features~\cite{Ma:2008,Agarwal:2009,Cao:2010}.
Another popular approach involves explicitly mapping such features to the latent embeddings~\cite{Gantner:2010}.

{\bf Methods}.
Ranking approaches such as popularity-based ranking have been shown to
work surprisingly well~\cite{mcfee2012million,bonnin2013evaluating,bonnin2015automated}.
The reason is believed to be the long-tail distribution of songs in
playlists~\cite{cremonesi2010performance,bonnin2013evaluating}.
%
This approach has been further improved by taking into account artist information in addition to
song popularity, which creates a comparably strong baseline that outperforms many sophisticated
approaches such as Bayesian Personalised Ranking based approach, neighbourhood methods, and approaches
making use of association rules and sequential patterns~\cite{mcfee2012million,bonnin2013evaluating}.
%
Ranking method has also been used as a post-processing component in more sophisticated approaches,
where a subset of songs were selected by scoring~\cite{jannach2015beyond} or matching~\cite{hariri2012context}
before ranking which optimises specific characteristics of the generated playlists.


Markov chains and related approaches were also widely used for playlist generation by casting the task
as language modelling problem~\cite{mcfee2011natural},
random walks in a hyper-graph~\cite{mcfee2012hypergraph} where songs were first grouped (by genre,
year of release, or social tags etc.) to serve as edges in the hyper-graph, or samples of Markov chains
in latent space where songs are embedded as (pairs of) points~\cite{chen2012playlist}.
%
Other techniques including playlist generation as sequential classification based on context~\cite{ben2017groove},
Gaussian process regression for user preference learning~\cite{platt2002learning},
topic models for sequential pattern mining and the well-known matrix factorisation techniques for learning
latent representations of songs, artists and users~\cite{mcfee2012hypergraph,chen2012playlist,ben2017groove}.


{\bf A diverse set of information} has been used for music recommendation,
such as song metadata (\eg song title, artist name, era, genre, albums etc.)~\cite{hariri2012context,platt2002learning},
content data (\eg lyrics, low level audio features etc.)~\cite{mcfee2011natural,mcfee2012hypergraph,jannach2015beyond,ben2017groove},
artists information (\eg artist popularity, collocation of artists etc.)~\cite{bonnin2013evaluating,ben2017groove},
user listening history and social interactions (\eg social tags) as well as usage statistics (\eg song popularity,
song co-occurrence etc.)~\cite{mcfee2012hypergraph,hariri2012context,bonnin2013evaluating,jannach2015beyond,ben2017groove}.
There are a few work that make use of latent representations of song, user and artist which are learned from existing playlists,
or user-song and user-artist interactions~\cite{chen2012playlist,ben2017groove}.

The sequential order of songs in playlist has not been well understood~\cite{schedl2017},
some work suggest that song order and song-to-song transitions are important
for playlist quality~\cite{mcfee2012hypergraph,kamehkhosh2018automated},
while other work have shown that the ensemble of songs in playlist do matter,
but the song order seems to be negligible~\cite{tintarev2017sequences,vall2017importance}.
In this work, we treat a playlist as a set of songs by discarding the sequential order,
and leave the investigation of using song order to assist playlist generation as future work.

%\cheng{In 2-3 sentences, re-define the problem we are attacking in this paper.}



%It has been known that binary classification and bipartite ranking are
%closely related~\cite{ertekin2011equivalence,menon2016bipartite}.
%In particular, \citet{ertekin2011equivalence} have shown that the P-Norm Push bipartite ranking loss~\cite{rudin2009p}
%is equivalent to the P-Classification loss~\cite{ertekin2011equivalence} when using the exponential surrogate.
%Further, the P-Norm Push loss is an approximation of the Infinite-Push loss~\cite{agarwal2011infinite},
%or equivalently, the Top-Push loss~\cite{li2014top}, which focuses on the highest ranked negative example instead of
%of the lowest ranked positive example as in~(\ref{eq:bprisk}).
%
%Inspired by these connections, we seek a classification loss that is equivalent to a bipartite ranking loss (under a few assumptions),
%which can approximate the risk with Bottom-Push loss in~(\ref{eq:bprisk}).
%This will make it possible to formulate an unconstrained objective that approximates the empirical loss $\RCal_\textsc{rank}$.


{\bf Cold-start music recommendation}.
Content based recommendation approaches~\cite[Chapter~4]{aggarwal2016recommender}
can be adopted to recommend {\it cold songs} (\ie new songs),
typically by making use of content features of songs extracted either automatically~\cite{seyerlehner2010automatic,eghbal2015vectors}
or manually by musical experts~\cite{john2006pandora}.
Further, content features can also be combined with other approaches, such as those based on 
collaborative filtering~\cite{yoshii2006hybrid,donaldson2007hybrid,shao2009music}.
This is known as the hybrid recommendation approaches~\cite{burke2002hybrid}, 
see~\cite[Chapter~6]{aggarwal2016recommender} for a general description besides music recommendation.
The problem of recommending music for {\it cold users} (\ie new users) 
has also been tackled by a number of approaches, such as transferring user preferences learned 
from related domains~\cite{hu2010study,aizenberg2012build},
or techniques that balance the exploration-exploitation trade-off~\cite{wang2014exploration,liebman2015dj}.

\clearpage
\newpage

\section{Experiment}
\label{sec:experiment}

We first evaluate the proposed method on the task of tag recommendation from text data,
which was formulated as multi-label classification problem~\cite{katakis2008multilabel}.

\subsection{Tag recommendation as multi-label classification}

We experiment on two dataset, \texttt{bibtex} and \texttt{bookmarks}~\cite{katakis2008multilabel}.
The performance are evaluated on classification metrics, \ie F$_1$ scores averaged over either examples or labels 
(which are also known as instance-F$_1$ and macro-F$_1$, respectively),
as well as ranking metric, \ie R-Precision (averaged over either examples or labels).

\paragraph{Baselines}
We compare our method with four baselines.
\begin{itemize}
\item Logistic regression: independently learn a logistic regression classifier for each label, (a.k.a binary relevance).
\item PRLR~\cite{lin2014multi}: a multi-label classifier with a regulariser which encourages sparse and low-rank predictions.
\item SPEN~\cite{belanger2016structured}: a structured prediction framework which employs a deep network to represent the energy function,
      and predictions are produced by minimising the energy.
\item DVN~\cite{gygli2017deep}: a structured prediction method which uses a deep value network to distill the knowledge of a given loss function,
      which is the F$_1$ score (averaged over examples) in this task.
\end{itemize}

We implemented Logistic regression using scikit-learn~\cite{}.
The results of SPEN and DVN are reproduced using the coded released the authors,
and the results of PRLR are taken from \cite{lin2014multi}.

%$\RCal_\textsc{example}$ 
For the proposed method, we used a linear score function $f(\x) = \w_k^\top \x + b$ for the $k$-th label,
and the empirical risk was minimised with L2 regularisation using LBFGS provided by Scipy~\cite{},
and hyper-parameters were tuned using 5-fold cross validation.

\paragraph{Result analysis}
The results on test set are summarised in Table~\ref{tab:perf_mlc},
\begin{itemize}
\item $\RCal_\textsc{example}$ outperform the independent logistic regression baseline by a large margin.
\item $\RCal_\textsc{example}$ also achieves better performance than PRLR~\cite{lin2014multi} which regularisation specific to multi-label classification.
\item Finally, it is encouraging that our method performs better than (\cite{belanger2016structured}) and (\cite{gygli2017deep}),
both work learn complex non-linear functions using deep neural networks to achieve state-of-the-art performance, while our method uses a linear function.
\end{itemize}

\TODO
\begin{itemize}
\item evaluation metric: add R-Precision averaged over both examples and labels, remove AUC.
\item results of more variants: $\RCal_\textsc{label}$ and $\RCal_\textsc{both}$?
\end{itemize}


\begin{table}[!h]
\centering
\caption{Performance on multi-label dataset}
\label{tab:perf_mlc}
%\resizebox{\linewidth}{!}{
\setlength{\tabcolsep}{2pt} % tweak the space between columns
%\begin{tabular}{l*{6}{c}}
\begin{tabular}{l|ccc|ccc}
\toprule
{} & \multicolumn{3}{c|}{\textbf{bibtex}} & \multicolumn{3}{c}{\textbf{bookmarks}} \\
{} &   F$_1$ Example & F$_1$ Label &    AUC &      F$_1$ Example & F$_1$ Label &    AUC \\
\midrule
Binary Relevance~\cite{}           &          $37.9$ &      $30.1$ & $85.3$ &             $29.5$ &      $21.0$ & $87.2$ \\
PRLR~\cite{lin2014multi}           &          $44.2$ &      $37.2$ &    N/A &             $34.9$ &      $23.0$ &    N/A \\
SPEN~\cite{belanger2016structured} &          $41.3$ &      $33.7$ & $92.6$ &             $35.5$ &      $24.1$ & $90.8$ \\
DVN~\cite{gygli2017deep}           &          $44.7$ &      $32.4$ & $86.7$ &             $37.2$ &      $23.7$ & $76.9$ \\
MLR (Ours)                         &          ${\bf 47.0}$ & ${\bf 38.8}$ & ${\bf 93.3}$ & ${\bf 37.7}$ & ${\bf 28.4}$ & ${\bf 91.8}$ \\
\bottomrule
\end{tabular}
%}
\end{table}



\subsection{New song recommendation}
\label{ssec:newsongrec}

\paragraph{Task:} 
We are interested in the task of recommending newly released songs to users,
in particular, to augment users' existing playlists with these songs,
which is a cold-start problem.

\paragraph{Problem formulation:}
We formulate the task of recommending newly released songs to augment existing playlists
as a multi-label classification problem, where we predict, for each song, 
whether it will be included in a given playlist.
This formulation is illustrated in Figure~\ref{fig:mlr},
where rows represent songs (from top to bottom, sorted by the release date in ascending order)
and columns represent playlists (no specific order).
Further, rows with white colour represent songs in training set, and rows with grey colour represent songs in test set.
If entry $(i, j)$ is \texttt{1} (or \texttt{0}), it means the $i$-th song is (or not) found in the $j$-th playlist,
otherwise, we do not know whether the $i$-th song is found in the $j$-th playlist (\ie entry $(i, j)$ is a question mark \texttt{?}).


\TODO
{\it Formulas for each variant.
The objective is the same as $\RCal_\textsc{row}$ and $\RCal_\textsc{col}$ except that,
for the playlists that we choose to hold the later half, all we observed is the first half, 
all other songs for these playlists are unobserved (they can be positive/negative examples),
this is different from the case that we explicitly observed that songs are not in playlists (they are negative examples).
}


\input{fig_mlr}

\paragraph{Dataset:}
We make use of publicly available playlist dataset: the AotM-2011~\cite{mcfee2012hypergraph} and 30Music~\cite{30music2015} playlist dataset. \\
%
{\bf AotM-2011 Dataset} is a collection of playlists shared by users\footnote{\url{http://www.artofthemix.org}} ranging from 1998 to 2011, 
songs in the dataset had been matched to those in the Million Song Dataset (MSD)~\cite{msd2011}.
We filtered out playlists with less than 5 songs, which results in roughly 84K playlists over 114K songs from 14K users. \\
%
{\bf 30Music Dataset} is a collection of listening events and playlists retrieved from Last.fm\footnote{\url{https://www.last.fm}}.
We utilise the playlists data by first intersecting with the MSD, leveraging the Last.fm dataset~\cite{lastfmdataset} 
which matched songs from Last.fm with those in MSD, then filtering out playlists with less than 5 songs, 
which results in roughly 17K playlists over 45K songs from 8K users.

We make use of the audio features of songs provided by MSD, 
and genre data from the Top-MAGD genre dataset~\cite{schindler2012facilitating} and tagtraum genre annotations for MSD~\cite{schreiber2015improving},
which results in 202 audio features and 15 one-hot encoding for genres.

%% details for compute song features.
%% - temporal audio features: use 5-number (percentiles) summary: min, max, median, Q1 and Q3.
%% - missing genre: imputed using the mean values of the genre

Table~\ref{tab:stats_newsongrec} summarises the statistics of the two dataset used for this task.

\begin{table}[!h]
\centering
\caption{Statistics of dataset for new song recommendation}
\label{tab:stats_newsongrec}
%\resizebox{\linewidth}{!}{
\begin{tabular}{ccccccc}
\toprule
Dataset & \#Users & \#Songs (train/dev/test) & \#Playlists & \#Song Features \\
\midrule
AotM-2011 & 14,182  & 68,657 / 22,885 / 22,886 & 84,710 & 217 \\
30Music   & 8,070   & 27,281 / 9,093 / 9,094   & 17,457 & 217 \\
\bottomrule
\end{tabular}
%}
\end{table}


\paragraph{Experimental design:}
In each dataset, we hold (a random) half of the 40\% latest released songs for test,
and other half as validation set, the remaining 60\% of songs are used for training.
All playlists in the dataset are used for this task.


\subsubsection{A few conclusions}

\paragraph{Which type of loss is most helpful?}
\begin{table}[!h]
\centering
\caption{Empirical results (AUC)}
%\resizebox{\linewidth}{!}{
\begin{tabular}{l|cccc}
\toprule
{}            & $\RCal_\textsc{example}$ & $\RCal_\textsc{label}$ & $\RCal_\textsc{both}$ & Independent L.R. \\
\midrule
AotM-2011     & 0.64792  & 0.67782 & 0.59602  & 0.62226 \\
30Music       & 0.6768   & 0.70917 & 0.70914  & 0.6654 \\
%30Music       & 0.54413  & 0.55894 & 0.55864  & 0.53698 \\
\bottomrule
\end{tabular}
%}
\end{table}

\paragraph{Experimental design:}
C: 1, 1, 1, p: 1, no multi-task regularisation.

\paragraph{Is multi-task regularisation helpful?}

\begin{table}[!h]
\centering
\caption{Empirical results}
%\resizebox{\linewidth}{!}{
\begin{tabular}{l|ccc}
\toprule
{}            & Multi-task Reg. + $\RCal_\textsc{example}$ & Multi-task reg. + $\RCal_\textsc{label}$ \\
\midrule
AotM-2011     & 0.65778     & 0.6884618 \\
30Music       & 0.68179     & 0.7149 \\
%30Music       & 0.549567    & 0.557156 \\
\bottomrule
\end{tabular}
%}
\end{table}



\subsection{Playlist augmentation}
\label{ssec:pla}

\paragraph{Task:}
We are interested in augmenting user created playlists with songs from a music library,
in particular, for a partial playlist, we would like to add more songs from an existing collection of songs.
The difference between this task from the task in Section~\ref{ssec:newsongrec} is that,
it is possible to choose any songs from the entire collection of songs, which is not a cold-start problem,
while the task described in Section~\ref{ssec:newsongrec} restricts that 
we choose songs from a subset of the entire collection (the newly released songs), 
which is a cold-start problem.

\paragraph{Problem formulation:}
We formulate the task of augmenting existing playlist as a multi-label classification problem,
that is, for each song that is not in the given playlist, 
we predict whether it will be added to augment the given playlist.
This formulation is illustrated in Figure~\ref{fig:pla},
where rows represent songs (no specific order) and columns represent playlists (no specific order).
Further, columns with white colour represent playlists in training set, 
and columns with grey colour represent playlists that should be augmented (\ie test set).
Similar to the formulation in Section~\ref{ssec:newsongrec}, if entry $(i, j)$ is \texttt{1} (or \texttt{0}), 
it means the $i$-th song is (or not) found in the $j$-th playlist, 
and a question mark \texttt{?} means that we do not know whether the $i$-th song is found in the $j$-th playlist.
As a remark, columns represent playlists in test set contain only \texttt{1} and \texttt{?} entries.

\begin{figure}[!h]
\centering
\setlength{\tabcolsep}{1pt} % tweak the space between columns
\begin{tabular}{|*{7}{c}|ccccc|} \hline
%\rule{.3em}{0pt} 
\rule{0em}{10pt}
& \texttt{0} & \texttt{1} & \texttt{0} & $\cdots$ & \texttt{0} & & & \texttt{?} & $\cdots$ & \texttt{?} & \\
& \texttt{0} & \texttt{0} & \texttt{0} & $\cdots$ & \texttt{0} & & & \texttt{1} & $\cdots$ & \texttt{?} & \\
& \texttt{1} & \texttt{0} & \texttt{0} & $\cdots$ & \texttt{0} & & & \texttt{?} & $\cdots$ & \texttt{?} & \\
\vspace{-5pt}
& \texttt{0} & \texttt{0} & \texttt{0} & $\cdots$ & \texttt{1} & & & \texttt{?} & $\cdots$ & \texttt{1} & \\
& $\vdots$ & $\vdots$ & $\vdots$ & $\vdots$ & $\vdots$ & & & $\vdots$ & $\vdots$ & $\vdots$ & \\
& \texttt{0} & \texttt{0} & \texttt{1} & $\cdots$ & \texttt{0} & & & \texttt{?} & $\cdots$ & \texttt{?} & \\ \hline
\end{tabular}
\caption{Illustration of augmenting playlists as multi-label classification.}
\label{fig:pla}
\end{figure}


\paragraph{Dataset:}
We again use the AotM-2011~\cite{mcfee2012hypergraph} and 30Music~\cite{30music2015} playlist dataset,
the pre-process of the two dataset is the same as that in Section~\ref{ssec:newsongrec}.
Besides making use of the audio features and genres of songs,
we also use the popularity of a given song as a feature, which is defined as the number of occurrence in all playlists,
including the partial playlists in test set.

Table~\ref{tab:stats_pla} summarises the statistics of the two dataset used for this task.

\begin{table}[!h]
\centering
\caption{Statistics of dataset for playlist augmentation}
\label{tab:stats_pla}
%\resizebox{\linewidth}{!}{
\begin{tabular}{ccccccc}
\toprule
Dataset & \#Users & \#Songs & \#Playlists (train/dev/test)  & \#Song Features \\
\midrule
AotM-2011 & 14,182 & 114,428 & 60,260 / 12,225 / 12,225 & 218 \\
30Music   & 8,070  & 45,468  & 15,591 / 933 / 933       & 218 \\
\bottomrule
\end{tabular}
%}
\end{table}

\paragraph{Experimental design:}
In each dataset, we create the test set such that it contains 20\% of each user's playlists (chosen uniformly at random),
if the user has 5 or more playlists, the validation set is constructed the same as the test test.
All remaining playlists are used for training.
As a remark, we observed all songs during training.


\subsubsection{A few conclusions}

\paragraph{Which type of loss is most helpful?}

{\bf Row-wise loss}: weighting by the number of positive/negative labels for each example.
\ie we perform a classification/bipartite ranking task on each multilabel example 
which forms a dataset of examples with binary labels: $\{(\x_n, y_k\}_{k=1}^K$ for the $n$-th multilabel example.

\begin{equation*}
%\resizebox{\linewidth}{!}{$
\RCal_\textsc{row} 
= \displaystyle \sum_s 
  \frac{1}{K_+^s} \sum_{s \in pl} e^{-(\w_{pl}^\top \phi(s) + b)} +
  \frac{1}{K_-^s} \sum_{s \notin pl} \frac{1}{p} e^{p \w_{pl}^\top \phi(s)}.
%$}
\end{equation*}
where normalising factor $K_+^s$ is the number of playlists that include song $s$,
and $K_-^s$ is the number of playlists that do not include song $s$.


{\bf Column-wise loss}: weighting by the number of positive/negative examples for each label.
\ie we perform a classification/bipartite ranking task on each label which forms a binary dataset:
$\{\x_n, y_k\}_{n=1}^N$ for the $k$-th label.

\begin{equation*}
%\resizebox{\linewidth}{!}{$
\RCal_\textsc{col} 
= \displaystyle \sum_{pl}
  \frac{1}{N_+^{pl}} \sum_{s \in pl} e^{-(\w_{pl}^\top \phi(s) + b)} +
  \frac{1}{N_-^{pl}} \sum_{s \notin pl} \frac{1}{p} e^{p \w_{pl}^\top \phi(s)}.
%$}
\end{equation*}
where normalising factor $N_+^{pl}$ is the number of songs in playlist $pl$,
and $N_-^{pl}$ is the number of songs in a music library that playlist $pl$ does not include.


{\bf Row-wise + column-wise loss}: the summation of both: $\RCal_\textsc{row} + C \RCal_\textsc{col}$ 
where $C$ is a trade-off parameter.

The binary relevance baseline is learning a logistic regression for each playlist independently.


\begin{table}[!h]
\centering
\caption{Empirical results}
%\resizebox{\linewidth}{!}{
\begin{tabular}{l|ccccc}
\toprule
{}            & $\RCal_\textsc{example}$ & $\RCal_\textsc{label}$ & $\RCal_\textsc{both}$ & Independent L.R. & Pop-rank \\
\midrule
%AotM-2011     & 0.6827396 & 0.743770 & 0.7385298 & 0.6924 & 0.80199 \\
AotM-2011     & 0.68459 & 0.747755 & 0.7429  & 0.6924 & 0.80199 \\
30Music       & 0.7168  & 0.76867  & 0.76917 & 0.7225 & 0.7165 \\
%30Music       & 0.56766179 & 0.6350941 & 0.63555 & 0.575567 & 0.80558 \\
\bottomrule
\end{tabular}
%}
\end{table}

\paragraph{Experimental design:}
C: 1, 1, 1, p: 1, no multi-task regularisation

\paragraph{Is multi-task regularisation helpful?}

multi-task regularisation: we regularise the difference of playlist parameters 
such that $\|\w_j - \w_k\|_2$ is small if playlist $j$ and $k$ belong to the same user.

\begin{equation*}
\RCal_\textsc{reg} = \frac{1}{\sum_u N_u (N_u - 1)} \sum_u \sum_{j, k \in u} (\w_j - \w_k)^\top (\w_j - \w_k)
\end{equation*}
where $N_u$ is the number of playlist user $u$ has.

\begin{table}[!h]
\centering
\caption{Empirical results}
%\resizebox{\linewidth}{!}{
\begin{tabular}{l|ccc}
\toprule
{}            & Multi-task Reg. + $\RCal_\textsc{example}$ & Multi-task reg. + $\RCal_\textsc{label}$ \\
\midrule
%AotM-2011     & 0.6882099 & 0.7615590 \\
AotM-2011     & 0.69167 & 0.7819 \\
30Music       & 0.7177  & 0.7840 \\
%30Music       & 0.581426 & 0.6597  \\
%30Music       & 0.574175 & 0.667804  \\
%AUC           & 0.66583  & 0.68517   \\
\bottomrule
\end{tabular}
%}
\end{table}



\TODO
measure performance by AUC and HitRate@K,
compare with baselines such as independent logistic regression (\ie binary relevance), popularity based recommendation,
and matrix factorisation.



\subsection{Discussion}

{\it the choice of playlist dataset?}

\section{Conclusion}

We investigate the problem of recommending playlists to users in cold-start settings.
In the cold songs setting, we recommend newly released songs to extend existing playlists;
in the cold playlists setting, we recommend a set of songs to form a new playlist for an existing user;
and in the cold users setting, we recommend a set of songs to form a new playlist for a new user.
We deal with all three settings using a multitask learning method which encourages songs in playlist 
to be ranked higher than those are not by minimising a bipartite ranking loss. 
We formulate the objective as a constrained convex optimisation problem, and further approximates it 
by an unconstrained objective inspired by an equivalence relationship between bipartite ranking and
binary classification. 
Empirical results on two real music playlist datasets show the proposed approach 
has good performance for playlist recommendation in cold-start settings.

We are aware of a few limitations of the proposed approach, which we leave as future work.
Specifically, additional data sources (\eg music information shared on social media) or song/user 
features (\eg lyrics, user profile), as well as the sequential order of songs which could provide 
additional information to help make better recommendations.
Further, non-linear models such as deep neural networks have shown strong performance in a wide arrange of tasks,
and the linear model with sparse parameters in this work could potentially be more compact if non-linear objective were employed.

Finally, as a remark, we want to mention the challenge of evaluating the recommended results.
While metrics in information retrieval are commonly used, recommender system is more like a generative process
than a information retrieval task. Fortunately, this challenge has been noticed and been attacked in many 
ways~\cite{mcfee2011natural,mcfee2012hypergraph,schedl2017}, 
we believe that promising automatic evaluation methods that accepted by the (majority of) 
community is one premise of significant progress in music recommendation.


\bibliographystyle{ACM-Reference-Format}
\bibliography{ref}

\clearpage
\newpage
\onecolumn
\appendix
\section{Notations}

We introduce notations in Table~\ref{tab:symbol_tpush}.
\begin{table}[!h]
\caption{Glossary of commonly used symbols}
\label{tab:symbol_tpush}
\renewcommand{\arraystretch}{1.5} % tweak the space between rows
\setlength{\tabcolsep}{1pt} % tweak the space between columns
\centering
\begin{tabular}{llll}
\toprule
\multicolumn{3}{l}{\textbf{Symbol}} & \textbf{Quantity} \\ \midrule
$D$        &  $\in$  &  $\Z^+$            & The number of features for each song \\
$M$        &  $\in$  &  $\Z^+$            & The number of songs, indexed by $m, n \in \{1,\dots,M\}$ \\
$N$        &  $\in$  &  $\Z^+$            & The number of playlists, indexed by $k \in \{1,\dots,N\}$ \\
$U$        &  $\in$  &  $\Z^+$            & The number of users, indexed by $i \in \{1,\dots,U\}$ \\
$\bu_i$    &  $\in$  &  $\R^D$            & The weights of user $i$ \\
$\bv_k$    &  $\in$  &  $\R^D$            & The weights of playlist $k$ \\
$\mubm$    &  $\in$  &  $\R^D$            & The weights shared by all playlists \\
$\w_{i,k}$ &  $\in$  &  $\R^D$            & The combined weights for user $i$ and playlist $k$, $\w_{i,k} = \bu_i + \bv_k + \mubm$ \\
$\bu_{i(k)}$  &  $\in$  &  $\R^D$         & The weights of user that owns playlist $k$ \\
$\Y$       &  $\in$  &  $\R^{M \times N}$ & The matrix of binary labels that indicating if a song is in a playlist \\
$y_m^k$    &  $\in$  &  $\R^D$            & The positive binary label $y_m^k = 1$, \ie song $m$ in playlist $k$ \\
$y_n^k$    &  $\in$  &  $\R^D$            & The negative binary label $y_n^k = 0$, \ie song $n$ not in playlist $k$ \\
$\X$       &  $\in$  &  $\R^{M \times D}$ & The matrix of features of all songs \\
$\x_m$     &  $\in$  &  $\R^D$            & The feature vector of song $m$ \\
$\x_n$     &  $\in$  &  $\R^D$            & The feature vector of song $n$ \\
\bottomrule
\end{tabular}
\end{table}

\section{Proof of Lemma~\ref{lm:rank2clf}}

First, we can approximate the empirical risk $R_{\uptheta}^{\textsc{rank}}$ (with the exponential surrogate) as follows:
\begin{equation*}
\begin{aligned}
R_{\uptheta}^{\textsc{rank}}(f, \DCal)
&= \frac{1}{N} \sum_{u=1}^U \sum_{i \in P_u} \frac{1}{M_-^i} \sum_{m': y_{m'}^i = 0} \exp \left( -\min_{m: y_m^i = 1} f(m, u, i) + f(m', u, i) \right) \\
&= \frac{1}{N} \sum_{u=1}^U \sum_{i \in P_u} \frac{1}{M_-^i} \exp \left( -\min_{m: y_m^i = 1} f(m, u, i) \right) 
   \sum_{m': y_{m'}^i = 0} \exp \left( f(m', u, i) \right) \\
&\approx \frac{1}{N} \sum_{u=1}^U \sum_{i \in P_u} \frac{1}{M_-^i} \exp \left( \frac{1}{p} \log \sum_{m: y_m^i = 1} e^{-p f(m, u, i)} \right)
   \sum_{m': y_{m'}^i = 0} \exp \left( f(m', u, i) \right) \\
&= \frac{1}{N} \sum_{u=1}^U \sum_{i \in P_u} \frac{1}{M_-^i} \left( \sum_{m: y_m^i = 1} e^{-p f(m, u, i)} \right)^\frac{1}{p} 
   \sum_{m': y_{m'}^i = 0} e^{f(m', u, i)} \\
&= \frac{1}{N} \sum_{u=1}^U \sum_{i \in P_u} \frac{1}{M_-^i} \left( \left( \sum_{m': y_{m'}^i = 0} e^{f(m', u, i)} \right)^p 
   \sum_{m: y_m^i = 1} e^{-p f(m, u, i)} \right)^\frac{1}{p} \\
&= \frac{1}{N} \sum_{u=1}^U \sum_{i \in P_u} \frac{1}{M_-^i} \left( 
   \sum_{m: y_m^i = 1} e^{-p f(m, u, i)} \left( \sum_{m': y_{m'}^i = 0} e^{f(m', u, i)} \right)^p \right)^\frac{1}{p} \\
&= \frac{1}{N} \sum_{u=1}^U \sum_{i \in P_u} \frac{1}{M_-^i} \left( 
   \sum_{m: y_m^i = 1} \left( \sum_{m': y_{m'}^i = 0} e^{- \left( f(m, u, i) - f(m', u, i) \right)} \right)^p \right)^\frac{1}{p} \\
&= \widetilde R_{\uptheta}^{\textsc{rank}}(f, \DCal).
\end{aligned}
\end{equation*}

Recall that $R_{\uptheta}^\textsc{mtc}$ is the following classification risk:
\begin{equation*}
R_{\uptheta}^{\textsc{mtc}}(f, \DCal)
= \frac{1}{N} \sum_{u=1}^U \sum_{i \in P_u} \left( 
  \frac{1}{p M_+^i} \sum_{m: y_m^i = 1} e^{-p f(m, u, i)} 
  + \frac{1}{M_-^i} \sum_{m': y_{m'}^i = 0} e^{f(m', u, i)} \right),
\end{equation*}

Let $\uptheta^* \in \argmin_{\uptheta} R_{\uptheta}^{\textsc{mtc}}$ (assuming minimisers exist),
we want to prove that $\uptheta^* \in \argmin_{\uptheta} \widetilde R_{\uptheta}^{\textsc{rank}}$.

\begin{proof}
We follow the proof technique in~\cite{ertekin2011equivalence}
by first introducing a constant feature $1$ for each song,
without loss of generality, let the first feature of $\x_m, \, m \in \{1,\dots,M\}$ be the constant feature, \ie $x_m^0 = 1$.
We can show that
$\frac{\partial \, R_{\uptheta}^{\textsc{mtc}}} {\partial \, \uptheta} = 0$ implies
$\frac{\partial \, \widetilde R_{\uptheta}^{\textsc{rank}}} {\partial \, \uptheta} = 0$,
which means minimisers of $R_{\uptheta}^{\textsc{mtc}}$ also minimise $\widetilde R_{\uptheta}^{\textsc{rank}}$.

Let 
%\begin{equation*}
%\begin{aligned}
$
\displaystyle
0 
= \frac{\partial \, R_{\uptheta}^{\textsc{mtc}}} {\partial \, \beta_i^0}
= \frac{1}{N} \left( 
   \frac{1}{p M_+^i} \sum_{m: y_m^i = 1} e^{-p f(m, u, i)} (-p)
   + \frac{1}{M_-^i} \sum_{m': y_{m'}^i = 0} e^{f(m', u, i)} \right),
\ \forall i \in P_u, \, u \in \{1,\dots,U\},
$
%\end{aligned}
%\end{equation*}

we have
\begin{equation}
\label{eq:eq1}
\frac{1}{M_+^i} \sum_{m: y_m^i = 1} e^{-p f(m, u, i)} \Bigg|_{\uptheta = \uptheta^*}
= \frac{1}{M_-^i} \sum_{m': y_{m'}^i = 0} e^{f(m', u, i)} \Bigg|_{\uptheta = \uptheta^*}, 
\ \forall i \in P_u, \, u \in \{1,\dots,U\},
\end{equation}

Further, let
\begin{equation*}
\zero 
= \frac{\partial \, R_{\uptheta}^{\textsc{mtc}}} {\partial \, \betabm_i} 
= \frac{1}{N} \left( 
   \frac{1}{p M_+^i} \sum_{m: y_m^i = 1} e^{-p f(m, u, i)} (-p \x_m)
   + \frac{1}{M_-^i} \sum_{m': y_{m'}^i = 0} e^{f(m', u, i)} \x_{m'} \right),
\ \forall i \in P_u, \, u \in \{1,\dots,U\},
\end{equation*}
we have
\begin{equation}
\label{eq:eq2}
\frac{1}{M_+^i} \sum_{m: y_m^i = 1} e^{-p f(m, u, i)} \x_m \Bigg|_{\uptheta = \uptheta^*}
= \frac{1}{M_-^i} \sum_{m': y_{m'}^i = 0} e^{f(m', u, i)} \x_{m'} \Bigg|_{\uptheta = \uptheta^*},
\ \forall i \in P_u, \, u \in \{1,\dots,U\}.
\end{equation}

Note that $\forall i \in P_u, \, u \in \{1,\dots,U\}$,
\begin{equation}
\label{eq:eq3}
\resizebox{\textwidth}{!}{$
\begin{aligned}
&\frac{\partial \, \widetilde R_{\uptheta}^{\textsc{rank}}} {\partial \, \betabm_i} \Bigg|_{\uptheta = \uptheta^*} \\
&= \frac{1}{N M_-^i} \left[
   \frac{1}{p} \left( \sum_{m: y_m^i = 1} e^{-p f(m, u, i)} \right)^{\frac{1}{p} - 1} 
   \sum_{m: y_m^i = 1} e^{-p f(m, u, i)} (-p \x_m) 
   \sum_{m': y_{m'}^i = 0} e^{f(m', u, i)} + 
   \left( \sum_{m: y_m^i = 1} e^{-p f(m, u, i)} \right)^\frac{1}{p}
   \sum_{m': y_{m'}^i = 0} e^{f(m', u, i)} \x_{m'} \right] \\
&= \frac{-1}{N M_-^i}
   \left( \sum_{m: y_m^i = 1} e^{-p f(m, u, i)} \right)^{\frac{1}{p} - 1} 
   \left[
   \sum_{m: y_m^i = 1} e^{-p f(m, u, i)} \x_m 
   \sum_{m': y_{m'}^i = 0} e^{f(m', u, i)} -
   \sum_{m: y_m^i = 1} e^{-p f(m, u, i)}
   \sum_{m': y_{m'}^i = 0} e^{f(m', u, i)} \x_{m'} \right] \\
&= \frac{-1}{N M_-^i}
   \left( \sum_{m: y_m^i = 1} e^{-p f(m, u, i)} \right)^{\frac{1}{p} - 1} 
   \left[
   \left( \sum_{m: y_m^i = 1} e^{-p f(m, u, i)} \x_m \right)
   \left( \frac{M_-^i}{M_+^i} \sum_{m: y_m^i = 1} e^{-p f(m, u, i)} \right) -
   \sum_{m: y_m^i = 1} e^{-p f(m, u, i)}
   \sum_{m': y_{m'}^i = 0} e^{f(m', u, i)} \x_{m'} \right] \\
&  \hspace{1.5em} (\text{by Eq.~\ref{eq:eq1}}) \\
&= \frac{-1}{N M_-^i}
   \left( \sum_{m: y_m^i = 1} e^{-p f(m, u, i)} \right)^\frac{1}{p}
   \left[
   \frac{M_-^i}{M_+^i}
   \sum_{m: y_m^i = 1} e^{-p f(m, u, i)} \x_m -
   \sum_{m': y_{m'}^i = 0} e^{f(m', u, i)} \x_{m'} \right] \\
&= \zero \ (\text{by Eq.~\ref{eq:eq2}}).
\end{aligned}
$}
\end{equation}

Let 
\begin{equation*}
h(u, i) 
= \frac{1}{N M_-^i} \left( \sum_{m: y_m^i = 1} e^{-p f(m, u, i)} \right)^\frac{1}{p} 
  \sum_{m': y_{m'}^i = 0} e^{f(m', u, i)},
\ \forall i \in P_u, \, u \in \{1,\dots,U\}.
\end{equation*}

Similar to Eq.~(\ref{eq:eq3}), we have
\begin{equation}
\label{eq:eq4}
\frac{\partial \, h(u, i)}{\partial \, \betabm_i} \Bigg|_{\uptheta = \uptheta^*} = \zero,
\ \forall i \in P_u, \, u \in \{1,\dots,U\}.
\end{equation}

Note that $\forall u \in \{1,\dots,U\}$,
by Eq.~(\ref{eq:eq4})
\begin{equation}
\label{eq:eq5}
\frac{\partial \, \widetilde R_{\uptheta}^{\textsc{rank}}} {\partial \, \alphabm_u} \Bigg|_{\uptheta = \uptheta^*} 
= \sum_{i \in P_u} \frac{\partial \, h(u, i)}{\partial \, \alphabm_u}  \Bigg|_{\uptheta = \uptheta^*}
= \sum_{i \in P_u} \frac{\partial \, h(u, i)}{\partial \, \betabm_i} \Bigg|_{\uptheta = \uptheta^*}
= \zero,
\end{equation}
and
\begin{equation}
\label{eq:eq6}
\frac{\partial \, \widetilde R_{\uptheta}^{\textsc{rank}}} {\partial \, \bmu} \Bigg|_{\uptheta = \uptheta^*} 
= \sum_{u=1}^U \sum_{i \in P_u} \frac{\partial \, h(u, i)}{\partial \, \bmu} \Bigg|_{\uptheta = \uptheta^*}
= \sum_{u=1}^U \sum_{i \in P_u} \frac{\partial \, h(u, i)}{\partial \, \betabm_i} \Bigg|_{\uptheta = \uptheta^*}
= \zero.
\end{equation}

Finally, by Eq.~(\ref{eq:eq3}), Eq.~(\ref{eq:eq5}), and Eq.~(\ref{eq:eq6}), $\uptheta^* \in \argmin_{\uptheta} \widetilde R_{\uptheta}^{\textsc{rank}}$.

\end{proof}

\end{document}
