\section{Related work}
\label{sec:related}

We summarise recent work most related to recommending music for playlist
according the problem settings, the recommendation method,
as well as the information being utilised.


{\bf Problem settings} for music recommendation including three typical settings:
playlist generation, next song recommendation, and playlist continuation.
%
There is rich collection of recent literature on the playlist generation 
or prediction~\cite{platt2002learning,mcfee2011natural,mcfee2012hypergraph,chen2012playlist,ben2017groove}.
%
Typical work is to generate a complete playlist given some seed,
for example, the AutoDJ system~\cite{platt2002learning} generate playlists given one or more seed songs,
a playlist for a specific user can be generated by Groove Radio given a seed artist~\cite{ben2017groove},
or a seed location in hidden space, which all songs are embedded, 
should to be specified to generate a complete playlist~\cite{chen2012playlist}.
%
There are also work that focus on evaluating the learned playlist model, 
without concretely generate playlist~\cite{mcfee2011natural,mcfee2012hypergraph}.
More details of playlist generation can be found in this recent survey~\cite{bonnin2015automated}.


Next song recommendation~\cite{hariri2012context,bonnin2013evaluating,jannach2015beyond}
is to predict the next song a user might play after observing some context,
for example, the most recent sequence of songs a user interacted with the system was used to 
infer the contextual information, which was further used to rank the next possible song 
with regards to a topic-based sequential patterns learned from user playlists~\cite{hariri2012context}.
%
The artists appeared in user's listening history can be used as context, 
which, together with the popularity of song or frequency of artists collocations,
were further used to score the next song~\cite{mcfee2012million,bonnin2013evaluating}.
%
It is obvious that next song recommendation techniques can also be used to generate a 
complete playlist by picking the next song sequentially~\cite{bonnin2013evaluating,ben2017groove}.


Playlist continuation is to add one or more songs to a playlist, 
given the added songs serve the same purpose of the original playlist~\cite{schedl2017,recsysch2018}.
One can immediately notice that playlist generation and next song recommendation are special cases
of playlist continuation, which means techniques developed for playlist generation and 
next song recommendation can both be used for playlist continuation.


{\bf Methods} for music recommendation.
Ranking approaches such as popularity-based ranking have been shown to
work surprisingly well~\cite{mcfee2012million,bonnin2013evaluating},
the reason is believed to be the long-tail distribution of songs in 
playlists~\cite{cremonesi2010performance,bonnin2013evaluating}.
%
This approach has been further improved by taking into account artist information in addition to
song popularity, which creates a comparably strong baseline that outperforms many sophisticated 
approaches such as Bayesian Personalised Ranking based approach, neighbourhood methods, and approaches 
making use of association rules and sequential patterns~\cite{mcfee2012million,bonnin2013evaluating}.
%
Ranking method has also been used as a post-processing component in more sophisticated approaches,
where a subset of songs were selected by scoring~\cite{jannach2015beyond} or matching~\cite{hariri2012context}
before ranking which optimises specific characteristics of playlist.


Markov chains and related approaches were also widely used for playlist generation,
which was cast as a language modelling problem~\cite{mcfee2011natural},
a random walk in hyper-graph~\cite{mcfee2012hypergraph}, where songs were first grouped (by genre, 
year of release, or social tags etc.) to serve as edges in the hyper-graph, or sample of Markov chains 
in latent space where songs are embedded as (pairs of) points~\cite{chen2012playlist}.
%
Other techniques such as playlist generation as sequential classification based on context~\cite{ben2017groove},
Gaussian process regression for user preference learning~\cite{platt2002learning},
topic models for sequential pattern mining and the well-known matrix factorisation techniques for learning 
latent representations of songs, artists and users~\cite{mcfee2012hypergraph,chen2012playlist,ben2017groove}.


{\bf A diverse set of information}, such as 
has been used for music recommendation.
