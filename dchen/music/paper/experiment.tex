\section{Experiment}
\label{sec:experiment}

\subsection{Music playlist dataset}
We make use of publicly available playlist dataset: the AotM-2011~\cite{mcfee2012hypergraph} and 30Music~\cite{30music2015} playlist dataset. \\
%
{\bf AotM-2011 Dataset} is a collection of playlists shared by users\footnote{\url{http://www.artofthemix.org}} ranging from 1998 to 2011, 
songs in the dataset had been matched to those in the Million Song Dataset (MSD)~\cite{msd2011}.
We filtered out playlists with less than 5 songs, which results in roughly 84K playlists over 114K songs from 14K users. \\
%
{\bf 30Music Dataset} is a collection of listening events and playlists retrieved from Last.fm\footnote{\url{https://www.last.fm}}.
We utilise the playlists data by first intersecting with the MSD, leveraging the Last.fm dataset~\cite{lastfmdataset} 
which matched songs from Last.fm with those in MSD, then filtering out playlists with less than 5 songs, 
which results in roughly 17K playlists over 45K songs from 8K users.

We make use of the audio features of songs provided by MSD, 
and genre data from the Top-MAGD genre dataset~\cite{schindler2012facilitating} and tagtraum genre annotations for MSD~\cite{schreiber2015improving},
which results in 202 audio features and 15 one-hot encoding for genres,
further, for the task of playlist augmentation, an additional feature we utilise is the popularity of songs,
that is, the number of occurrence of the song (\ie a listening event of the particular song) in all playlists,
including the partial playlists in test set, which will be described in Section~\ref{ssec:pla}.
%
%% details for compute song features.
%% - temporal audio features: use 5-number (percentiles) summary: min, max, median, Q1 and Q3.
%% - missing genre: imputed using the mean values of the genre
%
Table~\ref{tab:stats_pldata} summarises the two playlist dataset used in this work.
%
\begin{table}[!h]
\centering
\caption{Statistics of music playlist dataset}
\label{tab:stats_pldata}
%\resizebox{\linewidth}{!}{
\small
\begin{tabular}{l|rrrr}
\toprule
Dataset   & \#Songs & \#Playlists & \#Users & Avg. \#Songs per Playlist \\
\midrule
AotM-2011 & 114,428 & 84,710      & 14,182  & 10.1 \\
30Music   & 45,468  & 17,457      & 8,070   & 16.3 \\
\bottomrule
\end{tabular}
%}
\end{table}


\subsection{Evaluation}
We evaluate performance using metrics that are commonly used for music and playlist recommendation 
tasks~\cite{schedl2017,hariri2012context,jannach2015beyond}:
\begin{itemize}
\item R-Precision~\cite{manning2008introIR}, which computes the ratio of correctly recommended songs in top-$n$ recommendation, 
      where $n$ is the number of songs held for a given playlist.
\item Hit-Rate@K~\cite{hariri2012context}, which computes the ratio of correctly recommended songs among top-$K$ recommendation, 
      where $K$ is a given number, \eg $10$, $100$. 
      This metric is also known as Precision@K in information retrieval literature~\cite{manning2008introIR}.
\item NDCG and AUC.
\end{itemize}
We break down the results according to either user or the playlist length (\ie the number of songs in playlist).


\subsection{New song recommendation}
\label{ssec:newsongrec}

We hold 5000 of the latest released songs for test in both the AotM-2011 and 30Music dataset,
and we predict a song in test will be in a given playlist.
As a remark, the set of users are the same for training and test set,
so does the number of playlists; 
further, we remove playlists in which all songs are in test set are removed.
Table~\ref{tab:stats_newsongrec} summarises the statistics of the two dataset used for this task.

To tune hyper-parameters, we hold a subset of songs in training set as the validation set, 
which is constructed using the same approach as the test test.

\begin{table}[!h]
\centering
\caption{Statistics of dataset for new song recommendation}
\label{tab:stats_newsongrec}
%\resizebox{\linewidth}{!}{
\small
\begin{tabular}{l|rrr}
\toprule
Dataset   & \#Playlists \#Songs (train) & \#Songs (test) \\
\midrule
AotM-2011 &           &          & \\
30Music   &           &          & \\
\bottomrule
\end{tabular}
%}
\end{table}

%\begin{table}[!h]
\centering
\caption{Empirical results (R-Precision $\times 10^3$)}
%\resizebox{\linewidth}{!}{
\small
\begin{tabular}{l|cccc}
\toprule
{}            & & & & BR \\
\midrule
AotM-2011     &  &  &  & 0.92 \\
30Music       &  &  &  & 7.24 \\
\bottomrule
\end{tabular}
%}
\end{table}



\subsection{Playlist augmentation}
\label{ssec:pla}

We use the AotM-2011~\cite{mcfee2012hypergraph} and 30Music~\cite{30music2015} playlist dataset,
In each dataset, we create a test set by holding 20\% of each user's playlists (chosen uniformly at random) 
if the user has 5 or more playlists. 
The we hold all songs except the first $K, \, K \in \{1,2,3,4\}$ ones for every playlist in test set.
As a remark, we observed the entire collection of songs during training, 
and all users in test set have playlists in training set.
Table~\ref{tab:stats_pla} summarises the statistics of the two dataset used for this task.

To tune hyper-parameters, we hold a subset of playlists in training set as the validation set, 
which is constructed using the same approach as the test test.

\begin{table}[!h]
\centering
\caption{Statistics of dataset for playlist augmentation}
\label{tab:stats_pla}
%\resizebox{\linewidth}{!}{
\small
\begin{tabular}{l|rrr}
\toprule
Dataset   & \#Songs      & \#Playlists (train) & \#Playlists (test) \\
\midrule
AotM-2011 &              &             &           
30Music   &              &             &             
\bottomrule
\end{tabular}
%}
\end{table}

%\begin{table}[!h]
\centering
\caption{Empirical results (R-Precision $\times 10^3$)}
%\resizebox{\linewidth}{!}{
\small
\begin{tabular}{l|ccccc}
\toprule
{}            & $\RCal_\textsc{example}$ & $\RCal_\textsc{label}$ & $\RCal_\textsc{both}$ & BR & \textsc{PopRank} \\
\midrule
AotM-2011     &  &  &  & 2.05 & 3.67 \\
30Music       &  &  &  & 6.88 & 4.30 \\
%AotM-2011     &  &  &  & 2.69 & 3.69 \\
%30Music       & 5.53 & 9.02 &  & 9.44 & 4.49 \\
%AotM-2011     & 0.68459 & 0.747755 & 0.7429  & 0.6924 & 0.80199 \\
%30Music       & 0.7168  & 0.76867  & 0.76917 & 0.7225 & 0.7165 \\
%AotM-2011     & 0.6827396 & 0.743770 & 0.7385298 & 0.6924 & 0.80199 \\
%30Music       & 0.56766179 & 0.6350941 & 0.63555 & 0.575567 & 0.80558 \\
\bottomrule
\end{tabular}
%}
\end{table}

\begin{table}[!h]
\centering
\caption{Empirical results}
%\resizebox{\linewidth}{!}{
\begin{tabular}{l|ccc}
\toprule
{}            & Multi-task Reg. + $\RCal_\textsc{example}$ & Multi-task reg. + $\RCal_\textsc{label}$ \\
\midrule
%AotM-2011     & 0.69167 & 0.7819 \\
%30Music       & 0.7177  & 0.7840 \\
%AotM-2011     & 0.6882099 & 0.7615590 \\
%30Music       & 0.581426 & 0.6597  \\
%30Music       & 0.574175 & 0.667804  \\
%AUC           & 0.66583  & 0.68517   \\
\bottomrule
\end{tabular}
%}
\end{table}

