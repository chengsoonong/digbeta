\section{Introduction}
\label{sec:intro}
Online music streaming services (\eg Spotify, Pandora, Apple Music, Google Play Music, Amazon Music etc.) 
are playing an increasingly important role in the digital music industry.
A key ingredient of these services is the ability to automatically recommend songs for playlist
to help user explore a large collection of music and which also creates an uninterrupted listening experience.

% one paragraph about recommender system
Conventional recommender systems for books or movies~\citep{Sarwar:2001,Netflix},
which typically learns a score function via matrix factorisation~\citep{Koren:2009},
and recommend the item that achieves the highest score.
%
% explain what is different about recommending playlist
In contrast, when recommending musics for playlist, 
it is expected to recommend a subset from a large collection of music
instead of only one top ranked song.
In addition to the intractability of enumerating all possible subsets of a large collection,
it is highly likely that more than one subsets of songs meet users' requirements,
\eg a user generally has more than one playlist when using a music streaming service.

We are motivated by the problem of automatically recommending music to form a playlist for a given user.
If the user is new to the system, we would like to make a decent default recommendation by learning from 
all available playlists of existing users.
On the other hand, if the user already has a few playlists in the system, it is expected the recommender 
system can also learn from this information and hopefully make high quality recommendations.

In this paper, we propose a novel approach which ranks songs that could end up being in playlist for the 
given user higher than those unlikely, by using playlists of all users to optimise a multitask learning 
objective with respect a number of constraints.
Inspired by an equivalence relationship between bipartite ranking and binary classification,
we propose a classification loss which approximately minimises the multitask objective without 
any constraints.
Experiments on two real playlist datasets show that our proposed approaches work effectively,
and the classification based approach in particular significantly improves the learning efficiency 
while maintaining the same of high performance as the ranking approach.

Our paper is organised as follows:
Section 2 discusses previous work that are most relevant to our work.
Section 3 describes motivations of the multitask learning objective, and propose a ranking based approach 
that optimises the objective by solving a constrained optimisation problem.
We further describe a classification loss which enables us to approximately optimise the objective by
solving an unconstrained optimisation problem.
Section 4 details our experiment on two real playlist datasets.
Lastly, we summarises the paper in section 5 and describe future works.
