\section{Introduction}
\label{sec:intro}

{\it
%Binary classification and bipartite ranking are the simplest form among their own kind, during the decades (or even tens of decades) of research,
%quite a few loss functions have been proposed for these two tasks. 
%Yet, the relationships between these losses are not fully clear.

%There are many works that explored this topic, 
%JMLR'11 and JMLR'16 are typically ones,
%JMLR'11 showed that P-Norm push loss and P-Classification loss enjoys a nice equivalence relationships under exponential surrogate,
%moreover, the RankBoost and AdaBoost ...

%JMLR'16 showed that four of ... losses are equivalent. 

Briefly describe bipartite ranking, binary classification, multi-label classification.
the equivalence between P-Classification loss and P-Norm push loss under exponential surrogate.

top-push vs. bottom push, top-push vs P-Classification vs p-norm.
}

Contributions:
\begin{itemize}
\item We generalise the equivalence relationships between P-Norm Push loss and P-Classification loss, 
      with the equivalence between several pairs of bipartite ranking loss and classification loss as special cases,
      such as the (approximated) Top Push and P-Classification loss, the (approximated) Bottom Push and another classification loss,
      and the RankBoost and Cost-Sensitive AdaBoot loss.
\item We generalise the equivalence relationships from binary to multilabel setting, where we described two variants -- 
      by label and by example, we show that the former holds the equivalence claims similar to the binary setting, 
      the latter one, however, holds those claims under a few strict assumptions.
\item We empirically show that one can get good performance on both ranking and classification task by optimising only one loss function,
      on a standard multilabel classification benchmark as well as a novel application on music playlist augmentation and 
      new song recommendation.
\end{itemize}
