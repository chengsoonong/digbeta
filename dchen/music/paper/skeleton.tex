% 1. Recommend songs for playlists
% Problems description
% - playlist augmentation
% - new song recommendation
% - related work: the same, the different
%
% 2. Playlist augmentation and bipartite ranking
% Method:
% - augment one playlist can be achieved by bipartite ranking (assume order of tracks is irrelevant)
% - known: bipartite ranking == binary classification (under assumptions)
% - Theorem 1
% 
% 3. Playlist augmentation and multi-label classification
% Method:
% - augment multiple playlists -> multiple bipartite ranking -> multi-label classification
% - Theorem 2
% - one user has multiple playlists -> multi-task regularisation
%
% 4. New song recommendation: an extension of playlist augmentation
% Method:
% - the same as previous work (cold start)
% - the different (?)
%
% 5. Experiment
% - multi-label classification
% - playlist augmentation
% - new song recommendation
% 1. Recommend songs for playlists
% Problems description
% - playlist augmentation
% - new song recommendation
% - related work: the same, the different
%

\section{Problem}

% brief description of both recommendation tasks
We are motivated by the problem of automatically augmenting music playlist with a collection of songs $\SCal$,
in particular, given a partial playlist with the first $K$ songs\footnote{$K$ can be different for different partial playlists.},
we would like to recommend a subset of $\SCal$ by learning from user created playlist dataset.
This task is also known as Automatic Playlist Continuation~\cite{schedl2017,recsysch2018}.


\section{Related work}
% related work
% describe each task with related work: the same, the different
% describe that we don't use the order of songs, conflicting findings in literature
% mainly two pieces of work:
% - music recommendation
% - playlist generation
{\it Some related work.}

In this paper, we describe two variants of the playlist recommendation problem,
one is augmenting a playlist by recommending a subset of songs from a collection of music $\SCal$,
given the first $K$ seed songs, where $K$ can be any positive integer from 1 to the total number of songs in playlist minus 1.
% in contrast to settings where all songs except the last one are observed\cite{}, or giving a fixed number of seed songs\cite{}.
Another variant is restricting that all songs to recommend are not observed during learning,
\ie in the setting of recommending newly released songs to augment a given playlist, which is an instance of the cold-start problem.
We call the first variant \emph{playlist augmentation} and the second \emph{new song recommendation}.


\section{Machine learning tasks for playlist recommendation}

We formulate the task of \emph{playlist augmentation} as a multi-label classification problem,
that is, for each song that is not in the given playlist, 
we predict whether it will be added to the given playlist.
This formulation is illustrated in Figure~\ref{fig:pla},
where rows represent songs (no specific order) and columns represent playlists (no specific order).
Further, columns with white colour represent playlists in training set, 
and columns with grey colour represent playlists that should be augmented (\ie test set).
If entry $(i, j)$ is \texttt{1} (or \texttt{0}), 
it means the $i$-th song is (or not) found in the $j$-th playlist, 
and a question mark \texttt{?} means that we do not know whether the $i$-th song is found in the $j$-th playlist.
As a remark, columns represent playlists in test set contain only \texttt{1} and \texttt{?} entries.

\begin{figure}[!h]
\centering
\setlength{\tabcolsep}{1pt} % tweak the space between columns
\begin{tabular}{|*{7}{c}|ccccc|} \hline
%\rule{.3em}{0pt} 
\rule{0em}{10pt}
& \texttt{0} & \texttt{1} & \texttt{0} & $\cdots$ & \texttt{0} & & & \texttt{?} & $\cdots$ & \texttt{?} & \\
& \texttt{0} & \texttt{0} & \texttt{0} & $\cdots$ & \texttt{0} & & & \texttt{1} & $\cdots$ & \texttt{?} & \\
& \texttt{1} & \texttt{0} & \texttt{0} & $\cdots$ & \texttt{0} & & & \texttt{?} & $\cdots$ & \texttt{?} & \\
\vspace{-5pt}
& \texttt{0} & \texttt{0} & \texttt{0} & $\cdots$ & \texttt{1} & & & \texttt{?} & $\cdots$ & \texttt{1} & \\
& $\vdots$ & $\vdots$ & $\vdots$ & $\vdots$ & $\vdots$ & & & $\vdots$ & $\vdots$ & $\vdots$ & \\
& \texttt{0} & \texttt{0} & \texttt{1} & $\cdots$ & \texttt{0} & & & \texttt{?} & $\cdots$ & \texttt{?} & \\ \hline
\end{tabular}
\caption{Illustration of augmenting playlists as multi-label classification.}
\label{fig:pla}
\end{figure}



\paragraph{New song recommendation}
We formulate the task of \emph{new song recommendation} as a multi-label classification problem, 
where we predict, for each song in test set,
whether it will be included in a given playlist.
This formulation is illustrated in Figure~\ref{fig:mlr},
where rows represent songs (from top to bottom, sorted by the release date in ascending order)
and columns represent playlists (no specific order).
Further, rows with white colour represent songs in training set, and rows with grey colour represent songs in test set.
If entry $(i, j)$ is \texttt{1} (or \texttt{0}), it means the $i$-th song is (or not) found in the $j$-th playlist,
otherwise, we do not know whether the $i$-th song is found in the $j$-th playlist (\ie entry $(i, j)$ is a question mark \texttt{?}).
As a remark, we do not care about the order of songs in a playlist.

\input{fig_mlr}



\section{Method I: Playlist augmentation as multi-label classification}
{\it Two methods are described here, but only one of them is necessary for this paper.}


% so we can just do multiple bipartite ranking for these two problems?
% we can actually do better than this, given \ref{ertekin2011equivalence} showed that P-Classification == P-Norm Push, 
% we have (equivalence in binary setting) and (extend the equivalence to multi-label setting)
% so 
% - an independent  bipartite ranking seems to be a better baseline?
% - bottom push on MLC dataset?
% - Theorem 1
\section{Classification vs. bipartite ranking}
\label{sec:binary}


\subsection{P-Classification loss vs. P-Norm Push loss}
\label{ssec:pc=pn}

Given a binary dataset $\DCal = \SCal_+ \cup \SCal_-$, where $\SCal_+$ is a set of positive examples, 
\ie $\SCal_+ = \{(x_+, +1)\}$, and $\SCal_-$ is a set of negative examples, \ie $\SCal_- = \{(x_-, -1)\}$.

The empirical risk of the P-Classification loss is defined as~\cite{ertekin2011equivalence}
\begin{equation*}
\RCal_\textsc{pc}(\w, b) 
= \sum_{x_+ \in \SCal_+} e^{- (\w^\top x_+ + b)} +
  \frac{1}{p} \sum_{x_- \in \SCal_-} e^{p (\w^\top x_- + b)},
\end{equation*}
where $p > 0$ is a parameter.

The empirical risk of the P-Norm Push loss is defined as~\cite{rudin2009p}
\begin{equation*}
\begin{aligned}
\RCal_\textsc{pn}(\w)
&= \sum_{x_+ \in \SCal_+} \sum_{x_- \in \SCal_-} e^{-(\w^\top x_+ - \w^\top x_-)} \\
&= \sum_{x_+ \in \SCal_+} e^{-\w^\top x_+} \sum_{x_- \in \SCal_-} e^{\w^\top x_-}.
\end{aligned}
\end{equation*}

\citep{ertekin2011equivalence} showed the following equivalence relationship between P-Classification loss and P-Norm Push loss:
\begin{theorem}
\label{th:pc=pn}
If $(\w_\textsc{pc}, b_\textsc{pc}) \in \argmin_{\w,b} \RCal_\textsc{pc}(\w, b)$, 
then $\w_\textsc{pc} \in \argmin_\w \RCal_\textsc{pn}(\w)$.
Further, If $\w_\textsc{pn} \in \argmin_\w \RCal_\textsc{pn}(\w)$, 
then $(\w_\textsc{pn}, b_\textsc{pn}) \in \argmin_{\w,b} \RCal_\textsc{pn}(\w, b)$ where
$$
b_\textsc{pn} 
= \frac{1}{p + 1} \left( 
  \ln \sum_{x_+ \in \SCal_+} e^{-\w_\textsc{pn}^\top x_+} - 
  \ln \sum_{x_- \in \SCal_-} e^{p\w_\textsc{pn}^\top x_-} \right).
$$
\end{theorem}

It turns out Theorem~\ref{th:pc=pn} is due to a general equivalence relationship 
between binary classification and bipartite ranking, which we detail in the next section.



\subsection{Binary classification loss vs. bipartite ranking loss}

Let function $f(\cdot, \cdot)$ be
$$
f(x; \w) := g(x; \w) + b,
$$
where $\x$ is an input, $\w$ is a weight vector, $b$ is a bias parameter, 
and function $g(x; \w)$ is differentiable (w.r.t. $\w$) and bounded.

Suppose $\alpha, \beta, c, P, Q \in \R_+$ are \emph{finite} positive numbers, we define 
$\RCal_\textsc{bc}$ be the following classification risk\footnote{
We note that there is a equivalent definition:
$\RCal_\textsc{bc}(\w) = \frac{1}{\alpha} \sum_{x_+ \in \SCal_+} \exp(-\alpha f(x_+; \w)) + 
\frac{C}{\beta} \sum_{x_- \in \SCal_-} \exp( \beta f(x_-; \w))$
where $C = Q/P$, since multiplying a positive constant to a loss function will not change its minimiser.}
:
\begin{equation*}
%\label{eq:bc}
\resizebox{\linewidth}{!}{$
%\begin{aligned}
\RCal_\textsc{bc}(\w, b)
= \displaystyle 
  \frac{P}{\alpha} \sum_{x_+ \in \SCal_+} e^{-\alpha f(x_+; \w)} +
  \frac{Q}{\beta}  \sum_{x_- \in \SCal_-} e^{ \beta  f(x_-; \w)},
%\end{aligned}
$}
\end{equation*}
and let
$\RCal_\textsc{br}$ be a bipartite ranking risk defined as:
\begin{equation*}
%\label{eq:br}
\resizebox{\linewidth}{!}{$
\begin{aligned}
\RCal_\textsc{br}(\w)
&= \left[ \displaystyle 
   \sum_{x_+ \in \SCal_+} \left( \sum_{x_- \in \SCal_-} e^{-\beta (f(x_+; \w) - f(x_-; \w))} \right)^\frac{\alpha}{\beta} 
   \right]^c \\
&= \left[ \sum_{x_+ \in \SCal_+} e^{-\alpha f(x_+; \w)} \right]^\frac{c}{\alpha}
   \left[ \sum_{x_- \in \SCal_-} e^{ \beta  f(x_-; \w)} \right]^\frac{c}{\beta}.
\end{aligned}
$}
\end{equation*}
Note that $\RCal_\textsc{br}$ is independent of $b$.


\begin{theorem}
\label{th:bc=br}
If $(\w_\textsc{bc}, b_\textsc{bc}) \in \argmin_{\w,b} \RCal_\textsc{bc}(\w, b)$,
then $\w_\textsc{bc} \in \argmin_\w \RCal_\textsc{br}(\w)$, and
$$
\resizebox{\linewidth}{!}{$
\displaystyle
b_\textsc{bc} 
= \frac{1}{\alpha + \beta} \left( 
  P \ln \sum_{x_+ \in \SCal_+} e^{-\alpha g(x_+; \w_\textsc{bc})} -
  Q \ln \sum_{x_- \in \SCal_-} e^{  \beta g(x_-; \w_\textsc{bc})} \right).
$}
$$
Further, if $\w_\textsc{br} \in \argmin_\w \RCal_\textsc{br}(\w)$ (assuming minimisers exist),
then $(\w_\textsc{br}, b_\textsc{br}) \in \argmin_{\w,b} \, \RCal_\textsc{bc}(\w, b)$, where
$$
\resizebox{\linewidth}{!}{$
\displaystyle
b_\textsc{br} 
= \frac{1}{\alpha + \beta} \left( 
  P \ln \sum_{x_+ \in \SCal_+} e^{-\alpha g(x_+; \w_\textsc{br})} -
  Q \ln \sum_{x_- \in \SCal_-} e^{  \beta g(x_-; \w_\textsc{br})} \right).
$}
$$
\end{theorem}

We can get Theorem~\ref{th:pc=pn} from Theorem~\ref{th:bc=br} by letting $\alpha=c=P=Q=1$ and $\beta=p$.
Moreover, Theorem~\ref{th:bc=br} summaries a few more well-known special cases.

\TODO

{\it Cost-sensitive AdaBoost == RankBoost, Theorem 3 of~\cite{ertekin2011equivalence}.
IR Push == P-Classification,
IR Push = Top Push + exponential surrogate + log-sum-exp, P-Norm Push vs. Top Push vs. IR Push (approximation inside/outside)
Bottom Push + exponential surrogate + log-sum-exp == a classification loss similar to P-Classification.
}

\begin{table}[!h]
\centering
\caption{Summary of parameters for the equivalence of (risk of rank loss, risk of classification loss) pairs}
\label{tab:config}
\resizebox{\linewidth}{!}{
\begin{tabular}{r*{6}{c}}
\toprule
{\bf Loss pairs}                                         & $\alpha$ & $\beta$ & $\gamma$ & $\delta$ & $P$ & $Q$ \\ \hline
$\RCal_\textsc{pn} \equiva \RCal_\textsc{pc}$            & $1$      & $p$     & $p$      & $1$      & $\frac{1}{p}$ & $\frac{1}{p}$ \\
$\widetilde\RCal_\textsc{tp} \equiva \RCal_\textsc{pc}$  & $1$      & $r=p$   & $1$      & $\frac{1}{r}=\frac{1}{p}$ & $1$ & $1$ \\
$\widetilde\RCal_\textsc{bp} \equiva \RCal_\textsc{rc}$  & $r=q$    & $1$     & $\frac{1}{r}=\frac{1}{q}$ & $1$ & $1$ & $1$ \\
$\widetilde\RCal_\textsc{csa} \equiva \RCal_\textsc{rb}$ & $1$      & $1$     & $1$      & $1$      & $1$ & $C$ \\
\bottomrule
\end{tabular}
}
\end{table}




% 3. Playlist augmentation and multi-label classification
% Method:
% - augment multiple playlists -> multiple bipartite ranking -> multi-label classification
% - Theorem 2
% - one user has multiple playlists -> multi-task regularisation
%
\documentclass[9pt]{extarticle}
\usepackage[a4paper,top=0.79in,left=0.79in,bottom=0.79in,right=0.79in]{geometry} % A4 paper margins in LibreOffice
\usepackage{hyperref}
\usepackage{amsmath}
\usepackage{amsthm}
\usepackage{amsfonts}
\usepackage{mathrsfs}
\usepackage{bm}
\usepackage{bbm}
%\usepackage{ulem}
\usepackage{stmaryrd}
\usepackage{algorithm}
\usepackage{algorithmic}
\usepackage[sc]{mathpazo}
\linespread{1.05}       % Palladio needs more leading (space between lines)
\usepackage[T1]{fontenc}
\usepackage{footmisc}   % \footref, refer the same footnote at different places
\usepackage{subcaption} % sub-figures
\usepackage{setspace}   % set space between lines
\usepackage[utf8]{inputenc}
\usepackage[english]{babel}
\usepackage{xcolor}
\usepackage{graphicx}
\graphicspath{{fig/}}   % Location of the graphics files

\newtheorem{theorem}{Theorem}
\newtheorem{corollary}{Corollary}
\newtheorem{lemma}{Lemma}

\DeclareMathOperator*{\argmin}{argmin}
\DeclareMathOperator*{\argmax}{argmax}
\newcommand{\eat}[1]{}
\newcommand{\given}{\mid}
\newcommand{\llb}{\llbracket}
\newcommand{\rrb}{\rrbracket}
\newcommand{\bu}{\mathbf{u}}
\newcommand{\bv}{\mathbf{v}}
\newcommand{\h}{\mathbf{h}}
\newcommand{\x}{\mathbf{x}}
\newcommand{\X}{\mathbf{X}}
\newcommand{\Y}{\mathbf{Y}}
\newcommand{\y}{\mathbf{y}}
\newcommand{\w}{\mathbf{w}}
\newcommand{\p}{\mathbb{P}}
\newcommand{\E}{\mathbb{E}}
\newcommand{\R}{\mathbb{R}}
\newcommand{\q}{\mathbf{q}}
\newcommand{\LCal}{\mathcal{L}}
\newcommand{\XCal}{\mathcal{X}}
\newcommand{\YCal}{\mathcal{Y}}
\newcommand{\alphat}{\tilde{\alpha}}
\newcommand{\betat}{\tilde{\beta}}
\newcommand{\gammat}{\tilde{\gamma}}
\newcommand{\phit}{\tilde{\phi}}
% madeness: suPer-script in Brackets
\newcommand{\pb}[1]{^{({#1})}}

\newcommand{\eg}{e.g.\ }
\newcommand{\ie}{i.e.\ }
\newcommand{\downto}{\,\textbf{downto}\,}
\newcommand{\blue}[1]{{\color{blue}{#1}}}

\setlength{\columnsep}{1.5em} % spacing between columns

\title{Multi-label Classification, Bipartite Ranking and Playlist Generation}

\author{Dawei Chen}

\date{\today}

\begin{document}

\maketitle

\section{Multi-label classification~\cite{cheng:2010}}
\label{sec:mlc}

\noindent
\paragraph{Definition}
Let $\LCal = \{\lambda_1,\dots,\lambda_l\}$ be a finite set of class labels,
and example $(\x,\y) \in \XCal \times \YCal$, 
where $\YCal \in \{0,1\}^m$ is the set of all possible labels,
and $\y=y_{1:m}$ is a binary vector where $y_i = 1$ \emph{iff} $\lambda_i$ is a label of $\x$.
A multi-label classifier is a mapping $\h: \XCal \to \YCal$.

\noindent
\paragraph{Label dependence}
Suppose examples are independent and identically distributed (iid) according to a joint probability distribution $\p(\X,\Y)$ on $\XCal \times \YCal$,
where $\X$ is a random variable and $\Y=Y_{1:l}$ is a random vector,
Let $\p\pb{i}(Y_i |\x)$ be the marginal distribution of $Y_i$, then
\begin{equation*}
\p\pb{i}(Y_i=b |\x) = \sum_{\y \in \YCal:y_i = b} \p(\Y = \y |\x),
\end{equation*}
where $\p(\Y = \y |\x)$ is the posterior distribution given observation $\x$.
We note that the labels are not independent if 
\begin{equation*}
\p(\Y |\x) \ne \prod_{i=1}^l \p\pb{i}(Y_i |\x),
\end{equation*}
and the degree of dependence could be quantified in terms of measures such as cross entropy and KL divergence.

\noindent
\paragraph{Learning}
Given a loss function $\ell(\cdot)$, 
we can learn a multi-label classifier by find a model $\h^*$ that minimise the expected loss over the joint distribution $\p(\X,\Y)$:
\begin{equation*}
\h^* 
= \argmin_{\h} \, \E_{\X\Y} \, \ell(\Y,\h(\X))
= \argmin_{\h} \, \E_{\X} \, \E_{\Y|\X} \, \ell(\Y,\h(X))
= \argmin_{\h} \, \sum_{\x} \p(\x) \, \E_{\Y|\X} \, \ell(\Y,\h(\x)),
\end{equation*}
thanks to the summation, fix $\x$, we have
\begin{equation*}
\h^*(\x) = \argmin_{\y} \, \E_{\Y|\X} \, \ell(\Y,\y).
\end{equation*}
Frequently used loss functions in the context of multi-label classification including Hamming loss, rank loss and subset 0/1 loss~\cite{cheng:2010},
here we focus on a rank loss (taking care of ties):
\begin{equation}
\label{eq:loss_rank}
\ell(\y, \h(\x)) = \sum_{(i,j): y_i > y_j} \left( \llb h_i < h_j \rrb + \frac{1}{2} \llb h_i = h_j \rrb \right).
\end{equation}
\emph{Theorem 3.1 in~\cite{cheng:2010} here.}

\noindent
\paragraph{Probabilistic classifier chains}
Given a query $\x$, the posterior probability of a label $\y$ can be computed using the product rule of probability:
\begin{equation*}
\p(\y |\x) = \p(y_1) \cdot \prod_{i=2}^l \p(y_i |\x, y_{1:i-1}),
\end{equation*}
and we further define a function:
\begin{equation*}
f_i = 
\begin{cases}
\p(y_i = 1 |\x), & i = 1 \\
\p(y_i = 1 |\x, y_{1:i-1}), & 1 < i \le l
\end{cases}
\end{equation*}
then we have
\begin{equation*}
\p(\y |\x) = f_1 \cdot \prod_{i=2}^l f_i,
\end{equation*}
where $f_i$ uses $\x$ and $y_{1:i-1}$ as the input features. 
Theoretically, the results of the product rule does not depend on the order of variables, 
however, in practice, different order of variables will result in different model parameters (\ie the order of features depend on the order of variables). \\
\emph{Greedy approach -- classifier chain; assuming Markov property, we can use the Viterbi algorithm; with Neural net, we can build an order agnostic model.}


\section{Bipartite ranking}
\label{sec:birank}

\paragraph{Definition} 
Bipartite ranking is to learn a real-valued ranking function that places positive examples above negative examples~\cite{li:2014}.
Formally, given training examples $S = S_+ \cup S_-$ with $m$ positive examples $S_+ = \{\x_i^+\}_{i=1}^m$ and $n$ negative examples $S_- = \{\x_i^-\}_{i=1}^n$, 
bipartite ranking aims to learn a ranking function $f: \XCal \to \R$ that is likely ranks positive examples higher than negative examples.

\paragraph{Loss function}
AUC is a widely used as an evaluate metric for bipartite ranking, and it turns out that AUC can be optimised by minimising a loss defined as~\cite{cortes:2004}
\begin{equation}
\label{eq:loss_auc}
\ell_\text{rank}(f; S) = \frac{1}{mn} \sum_{i=1}^m \sum_{j=1}^n \llb f(\x_i^+) \le f(\x_j^-) \rrb,
\end{equation}
and this loss can be easily optimised (\eg by gradient descent) if we replace the indicator function with a convex surrogate such as the truncated quadratic loss 
$\ell(z) = (1+z)_+^2$, the exponential loss $\ell(z) = e^z$ and logistic loss $\ell(z) = \log(1+e^z)$.
One drawback of this loss function is enumerating all the positive-negative pairs, which is computationally expensive for large dataset. \\
\emph{Theorem 3.1 in~\cite{cheng:2010} for this loss function here.}

Alternatively, one may interested in optimising the ranking accuracy only at the top, 
or equivalently, we would like to minimize the number of positive examples that ranked below the highest-ranking negative instance~\cite{agarwal:2011,li:2014}:
\begin{equation}
\label{eq:loss_inf}
\begin{aligned}
\ell_{\infty}(f; S) 
&= \max_{1 \le j \le n} \frac{1}{m} \sum_{i=1}^m \, \llb f(\x_i^+) < f(\x_j^-) \rrb \\
&= \frac{1}{m} \sum_{i=1}^m \max_{1 \le j \le n} \llb f(\x_i^+) < f(\x_j^-) \rrb,
\end{aligned}
\end{equation}
by replace the indicator function in (\ref{eq:loss_inf}) with a convex surrogate $\ell(\cdot)$, we have
\begin{equation}
\label{eq:loss_inf1} 
\begin{aligned}
\tilde{\ell}_{\infty}(f; S) 
&= \frac{1}{m} \sum_{i=1}^m \max_{1 \le j \le n} \ell\left( f(\x_j^-) - f(\x_i^+) \right) \\
&= \frac{1}{m} \sum_{i=1}^m \ell\left( \max_{1 \le j \le n} f(\x_j^-) - f(\x_i^+) \right),
\end{aligned}
\end{equation}
which can be optimised more efficiently than (\ref{eq:loss_auc})~\cite{li:2014}.


\section{Playlist generation}
\label{sec:playlist}

The playlist generation problem can be formulated as a multi-label classification problem (we have a label for each song in library).
The idea is to replace the loss function (\ref{eq:loss_rank}) in multi-label classification with loss~(\ref{eq:loss_auc}) or (\ref{eq:loss_inf}). \\
We can build a probabilistic classifier chains with approximate inference (greedy, Viterbi, Neural net).


\bibliographystyle{ieeetr}
%\bibliographystyle{apalike}
\bibliography{ref_mlc}

\end{document}


% 4. New song recommendation: an extension of playlist augmentation
% Method:
% - the same as previous work (cold start)
% - the different (?)
%


\section{Method II: Playlist augmentation as bipartite ranking}
% For users with only one playlist, describe the Top push loss for bipartite ranking, and its dual
% For users with multiple playlists, use multitask regularisation + bipartite ranking w/ Top Push loss for each playlist, and the dual

Given a partial playlist with $K$ seed songs, a natural approach to recommend a subset of music collection $\SCal$ is 
ranking songs in $\SCal$ that are not seed songs, in particular, songs that are more relevant to the playlist should be
ranked higher that those that are not, which is a bipartite ranking problem.

%This can be formulated as a bipartite ranking problem where songs in the given playlist have positive labels,
%while songs that are not part of the playlist have negative labels.

% equation of bipartite ranking for playlist augmentation


One approach to \emph{focus on the most plausible songs}, for a given playlist,
is to learn a recommender system by minimising the rank of the top ranked song which is not in the playlist,
in other words, the higher it ranked over songs in playlist, the harder we penalise the learning system.
This is known as the Top Push loss in bipartite ranking~\cite{li2014top}, which is formally defined as,
\begin{equation}
\label{eq:toppush}
\LCal_\textsc{tp}(f; \DCal) 
= \frac{1}{M^+} \sum_{m: y^m = 1} \llb f(\x^m) \le \max_{n: y^n = 0} f(\x^n) \rrb,
\end{equation}
where $M = |\SCal|$ is the number of songs in $\SCal$, and $M^+$ is the number of songs in playlist.

To optimise the objective~\ref{eq:toppush}, 
we firstly have to replace the indicator function (0-1 loss) with one of its convex surrogate loss,
such as the exponential loss $\ell(f, y) = e^{-fy}$, logistic loss $\ell(f, y) = \log(1 + e^{-fy})$, 
or squared hinge loss $\ell(f, y) = \max\{0, (1 - fy)\}^2$.
Further, we have to deal with the challenge of \emph{max} operator in (\ref{eq:toppush}), which can sometimes be mitigated by solving the dual problem.
If the ranking function $f$ has a linear form, \ie $f(\x) = \w^\top \x$, it has been shown that the dual formulation of (\ref{eq:toppush}) 
with L2 regularisation is~\cite{li2014top}:
\begin{equation*}
\begin{aligned}
\min_{\alphabm, \betabm} \ & \frac{1}{2 C M^+} \| \alphabm^\top \X^+ - \betabm^\top \X^- \|^2 + \ell^*(\alphabm), \\
s.t. \ & \one^\top \alphabm = \one^\top \betabm, \\
       & \alphabm \in \R_+^{M^+}, \, \betabm \in \R_+^{M-M^+},
\end{aligned}
\end{equation*}
where $\ell^*$ is the convex conjugate of surrogate loss $\ell$.


\paragraph{Multitask regularisation}
% a user, in general, has multiple playlists, a reasonable assumption is that playlists of one user are more similar than playlists of different users,
% so we use a multitask regularisation to ensure the weights of playlists for the same users should be similar
Another observation is that a user generally has more than one playlists,
a reasonable assumption is playlists of the same user share similar characteristics. 
Suppose we learn a parameter vector for each playlist, then one approach to formalise this similarity is using an extra regulariser
for each user $u$ that has $N$ playlists,
\begin{equation}
\label{eq:mtreg}
\begin{aligned}
&\frac{1}{2} \cdot \frac{1}{N (N - 1) / 2} \sum_{i, j \in \{1,\dots,N\}, \, i < j} \| \w_i - \w_j \|^2 \\
&= \frac{1}{N (N - 1)} \left( (N - 1) \sum_{i=1}^{N} \w_i^\top \w_i - 2 \sum_{i, j \in \{1,\dots,N\}, \, i < j} \w_i^\top \w_j \right) \\
&= \frac{1}{N} \sum_{i=1}^{N} \w_i^\top \w_i - \frac{2}{N (N - 1)} \sum_{i, j \in \{1,\dots,N\}, \, i < j} \w_i^\top \w_j %\\
%&= \frac{1}{N} \sum_{i=1}^{N} \w_i^\top \w_i - \frac{1}{N (N - 1)} \sum_{i=1}^{N} \sum_{j \in \{1,\dots,N\}, j \ne i} \w_i^\top \w_j \\
%&= \left( \frac{1}{N} + \frac{1}{N (N - 1)} \right) \sum_{i=1}^{N} \w_i^\top \w_i 
%   - \frac{1}{N (N - 1)} \sum_{i=1}^{N} \sum_{j=1}^{N} \w_i^\top \w_j \\
%&= \frac{1}{N - 1} \sum_{i=1}^{N} \w_i^\top \w_i - \frac{1}{N (N - 1)} \sum_{i=1}^{N} \sum_{j=1}^{N} \w_i^\top \w_j.
\end{aligned}
\end{equation}
We call (\ref{eq:mtreg}) \emph{multitask regulariser} as we regularise the parameters of multiple bipartite ranking tasks. 

To optimise for the most plausible songs, we minimise the Top Push loss for each playlist of user $u$, 
we take into account the multitask regulariser in addition to the L2 regularisation, 
which results in an objective:
\begin{equation}
\label{eq:mtobj}
\begin{aligned}
&\frac{C_1}{2 N} \sum_{i=1}^{N} \| \w_i \|^2
+ \frac{C_2}{2 N (N - 1) / 2} \sum_{i, j \in \{1,\dots,N\}, \, i < j} \| \w_i - \w_j \|^2 
+ \frac{1}{N} \sum_{i = 1}^{N} \frac{1}{M_i^+} \sum_{m: y_i^m = 1} \llb f(\x^m) \le \max_{n: y_i^n = 0} f(\x^n) \rrb \\
&= \frac{C_1 + 2C_2}{2 N} \sum_{i=1}^{N} \w_i^\top \w_i 
- \frac{2C_2}{N (N - 1)} \sum_{i, j \in \{1,\dots,N\}, \, i < j} \w_i^\top \w_j
+ \frac{1}{N} \sum_{i = 1}^{N} \frac{1}{M_i^+} \sum_{m: y_i^m = 1} \llb f(\x^m) \le \max_{n: y_i^n = 0} f(\x^n) \rrb,
\end{aligned}
\end{equation}
where $C_1$ and $C_2$ are regularisation parameters.

If we assume a linear ranking function $f(\x) = \w^\top \x$, 
and use the exponential surrogate loss $\ell(f, y) = e^{-fy}$,
it can be shown that the dual of (\ref{eq:mtobj}) is
\begin{equation}
\label{eq:mtdual}
\begin{aligned}
\min_{\Thetabm} \ \ & \frac{1}{2} \sum_{d=1}^{D} \sum_{d'=1}^{D} \C \circ \left( \X^\top \Thetabm (\C^{-1})^2 \Thetabm^\top \X \right) 
    + \sum_{i = 1}^{N} \sum_{m: y_i^m = 1} \theta_i^m \left( 1 - \log(-\theta_i^m) - \log(N M_i^+) \right) \\
s.t. \ & \Thetabm^\top \one_M = \zero_{N} \\
       & \theta_k^n \ge 0, \, k \in \{1,\dots,N\}, \, n \in \{1,\dots,M\} \ \mathrm{and} \ y_k^n = 0.
\end{aligned}
\end{equation}
where $\circ$ denotes the element-wise multiplication,
$\Thetabm \in \R^{M \times N}$ are the dual variables, $D$ is the number of features of a song, 
$\X \in \R^{M \times D}$ is the design matrix where each row is the features of a song,
$M_i^+$ is the number of songs in the $i$-th playlist of user $u$,
and $\C \in \R^{N \times N}$ is a symmetric matrix such that
\begin{equation*}
C_{ij} = \begin{cases}
\frac{C_1 + 2C_2}{N}, & i = j \\
\frac{-2C_2}{N (N - 1)},  & \mathrm{otherwise}.
\end{cases}
\end{equation*}

Suppose $\Thetabm^*$ is the optimal solution of the dual problem (\ref{eq:mtdual}), 
then the optimal solution of the primal problem (\ref{eq:mtobj}) can be computed as
\begin{equation*}
\W^* = -\C^{-1} \Thetabm^{*\top} \X.
\end{equation*}


%\section{Experiment}
% experiments on two playlist dataset
\clearpage
\newpage

\section{Experiment}
\label{sec:experiment}

We first evaluate the proposed method on the task of tag recommendation from text data,
which was formulated as multi-label classification problem~\cite{katakis2008multilabel}.

\subsection{Tag recommendation as multi-label classification}

We experiment on two dataset, \texttt{bibtex} and \texttt{bookmarks}~\cite{katakis2008multilabel}.
The performance are evaluated on classification metrics, \ie F$_1$ scores averaged over either examples or labels 
(which are also known as instance-F$_1$ and macro-F$_1$, respectively),
as well as ranking metric, \ie R-Precision (averaged over either examples or labels).

\paragraph{Baselines}
We compare our method with four baselines.
\begin{itemize}
\item Logistic regression: independently learn a logistic regression classifier for each label, (a.k.a binary relevance).
\item PRLR~\cite{lin2014multi}: a multi-label classifier with a regulariser which encourages sparse and low-rank predictions.
\item SPEN~\cite{belanger2016structured}: a structured prediction framework which employs a deep network to represent the energy function,
      and predictions are produced by minimising the energy.
\item DVN~\cite{gygli2017deep}: a structured prediction method which uses a deep value network to distill the knowledge of a given loss function,
      which is the F$_1$ score (averaged over examples) in this task.
\end{itemize}

We implemented Logistic regression using scikit-learn~\cite{}.
The results of SPEN and DVN are reproduced using the coded released the authors,
and the results of PRLR are taken from \cite{lin2014multi}.

%$\RCal_\textsc{example}$ 
For the proposed method, we used a linear score function $f(\x) = \w_k^\top \x + b$ for the $k$-th label,
and the empirical risk was minimised with L2 regularisation using LBFGS provided by Scipy~\cite{},
and hyper-parameters were tuned using 5-fold cross validation.

\paragraph{Result analysis}
The results on test set are summarised in Table~\ref{tab:perf_mlc},
\begin{itemize}
\item $\RCal_\textsc{example}$ outperform the independent logistic regression baseline by a large margin.
\item $\RCal_\textsc{example}$ also achieves better performance than PRLR~\cite{lin2014multi} which regularisation specific to multi-label classification.
\item Finally, it is encouraging that our method performs better than (\cite{belanger2016structured}) and (\cite{gygli2017deep}),
both work learn complex non-linear functions using deep neural networks to achieve state-of-the-art performance, while our method uses a linear function.
\end{itemize}

\TODO
\begin{itemize}
\item evaluation metric: add R-Precision averaged over both examples and labels, remove AUC.
\item results of more variants: $\RCal_\textsc{label}$ and $\RCal_\textsc{both}$?
\end{itemize}


\begin{table}[!h]
\centering
\caption{Performance on multi-label dataset}
\label{tab:perf_mlc}
%\resizebox{\linewidth}{!}{
\setlength{\tabcolsep}{2pt} % tweak the space between columns
%\begin{tabular}{l*{6}{c}}
\begin{tabular}{l|ccc|ccc}
\toprule
{} & \multicolumn{3}{c|}{\textbf{bibtex}} & \multicolumn{3}{c}{\textbf{bookmarks}} \\
{} &   F$_1$ Example & F$_1$ Label &    AUC &      F$_1$ Example & F$_1$ Label &    AUC \\
\midrule
Binary Relevance~\cite{}           &          $37.9$ &      $30.1$ & $85.3$ &             $29.5$ &      $21.0$ & $87.2$ \\
PRLR~\cite{lin2014multi}           &          $44.2$ &      $37.2$ &    N/A &             $34.9$ &      $23.0$ &    N/A \\
SPEN~\cite{belanger2016structured} &          $41.3$ &      $33.7$ & $92.6$ &             $35.5$ &      $24.1$ & $90.8$ \\
DVN~\cite{gygli2017deep}           &          $44.7$ &      $32.4$ & $86.7$ &             $37.2$ &      $23.7$ & $76.9$ \\
MLR (Ours)                         &          ${\bf 47.0}$ & ${\bf 38.8}$ & ${\bf 93.3}$ & ${\bf 37.7}$ & ${\bf 28.4}$ & ${\bf 91.8}$ \\
\bottomrule
\end{tabular}
%}
\end{table}



\subsection{New song recommendation}
\label{ssec:newsongrec}

\paragraph{Task:} 
We are interested in the task of recommending newly released songs to users,
in particular, to augment users' existing playlists with these songs,
which is a cold-start problem.

\paragraph{Problem formulation:}
We formulate the task of recommending newly released songs to augment existing playlists
as a multi-label classification problem, where we predict, for each song, 
whether it will be included in a given playlist.
This formulation is illustrated in Figure~\ref{fig:mlr},
where rows represent songs (from top to bottom, sorted by the release date in ascending order)
and columns represent playlists (no specific order).
Further, rows with white colour represent songs in training set, and rows with grey colour represent songs in test set.
If entry $(i, j)$ is \texttt{1} (or \texttt{0}), it means the $i$-th song is (or not) found in the $j$-th playlist,
otherwise, we do not know whether the $i$-th song is found in the $j$-th playlist (\ie entry $(i, j)$ is a question mark \texttt{?}).


\TODO
{\it Formulas for each variant.
The objective is the same as $\RCal_\textsc{row}$ and $\RCal_\textsc{col}$ except that,
for the playlists that we choose to hold the later half, all we observed is the first half, 
all other songs for these playlists are unobserved (they can be positive/negative examples),
this is different from the case that we explicitly observed that songs are not in playlists (they are negative examples).
}


\input{fig_mlr}

\paragraph{Dataset:}
We make use of publicly available playlist dataset: the AotM-2011~\cite{mcfee2012hypergraph} and 30Music~\cite{30music2015} playlist dataset. \\
%
{\bf AotM-2011 Dataset} is a collection of playlists shared by users\footnote{\url{http://www.artofthemix.org}} ranging from 1998 to 2011, 
songs in the dataset had been matched to those in the Million Song Dataset (MSD)~\cite{msd2011}.
We filtered out playlists with less than 5 songs, which results in roughly 84K playlists over 114K songs from 14K users. \\
%
{\bf 30Music Dataset} is a collection of listening events and playlists retrieved from Last.fm\footnote{\url{https://www.last.fm}}.
We utilise the playlists data by first intersecting with the MSD, leveraging the Last.fm dataset~\cite{lastfmdataset} 
which matched songs from Last.fm with those in MSD, then filtering out playlists with less than 5 songs, 
which results in roughly 17K playlists over 45K songs from 8K users.

We make use of the audio features of songs provided by MSD, 
and genre data from the Top-MAGD genre dataset~\cite{schindler2012facilitating} and tagtraum genre annotations for MSD~\cite{schreiber2015improving},
which results in 202 audio features and 15 one-hot encoding for genres.

%% details for compute song features.
%% - temporal audio features: use 5-number (percentiles) summary: min, max, median, Q1 and Q3.
%% - missing genre: imputed using the mean values of the genre

Table~\ref{tab:stats_newsongrec} summarises the statistics of the two dataset used for this task.

\begin{table}[!h]
\centering
\caption{Statistics of dataset for new song recommendation}
\label{tab:stats_newsongrec}
%\resizebox{\linewidth}{!}{
\begin{tabular}{ccccccc}
\toprule
Dataset & \#Users & \#Songs (train/dev/test) & \#Playlists & \#Song Features \\
\midrule
AotM-2011 & 14,182  & 68,657 / 22,885 / 22,886 & 84,710 & 217 \\
30Music   & 8,070   & 27,281 / 9,093 / 9,094   & 17,457 & 217 \\
\bottomrule
\end{tabular}
%}
\end{table}


\paragraph{Experimental design:}
In each dataset, we hold (a random) half of the 40\% latest released songs for test,
and other half as validation set, the remaining 60\% of songs are used for training.
All playlists in the dataset are used for this task.


\subsubsection{A few conclusions}

\paragraph{Which type of loss is most helpful?}
\begin{table}[!h]
\centering
\caption{Empirical results (AUC)}
%\resizebox{\linewidth}{!}{
\begin{tabular}{l|cccc}
\toprule
{}            & $\RCal_\textsc{example}$ & $\RCal_\textsc{label}$ & $\RCal_\textsc{both}$ & Independent L.R. \\
\midrule
AotM-2011     & 0.64792  & 0.67782 & 0.59602  & 0.62226 \\
30Music       & 0.6768   & 0.70917 & 0.70914  & 0.6654 \\
%30Music       & 0.54413  & 0.55894 & 0.55864  & 0.53698 \\
\bottomrule
\end{tabular}
%}
\end{table}

\paragraph{Experimental design:}
C: 1, 1, 1, p: 1, no multi-task regularisation.

\paragraph{Is multi-task regularisation helpful?}

\begin{table}[!h]
\centering
\caption{Empirical results}
%\resizebox{\linewidth}{!}{
\begin{tabular}{l|ccc}
\toprule
{}            & Multi-task Reg. + $\RCal_\textsc{example}$ & Multi-task reg. + $\RCal_\textsc{label}$ \\
\midrule
AotM-2011     & 0.65778     & 0.6884618 \\
30Music       & 0.68179     & 0.7149 \\
%30Music       & 0.549567    & 0.557156 \\
\bottomrule
\end{tabular}
%}
\end{table}



\subsection{Playlist augmentation}
\label{ssec:pla}

\paragraph{Task:}
We are interested in augmenting user created playlists with songs from a music library,
in particular, for a partial playlist, we would like to add more songs from an existing collection of songs.
The difference between this task from the task in Section~\ref{ssec:newsongrec} is that,
it is possible to choose any songs from the entire collection of songs, which is not a cold-start problem,
while the task described in Section~\ref{ssec:newsongrec} restricts that 
we choose songs from a subset of the entire collection (the newly released songs), 
which is a cold-start problem.

\paragraph{Problem formulation:}
We formulate the task of augmenting existing playlist as a multi-label classification problem,
that is, for each song that is not in the given playlist, 
we predict whether it will be added to augment the given playlist.
This formulation is illustrated in Figure~\ref{fig:pla},
where rows represent songs (no specific order) and columns represent playlists (no specific order).
Further, columns with white colour represent playlists in training set, 
and columns with grey colour represent playlists that should be augmented (\ie test set).
Similar to the formulation in Section~\ref{ssec:newsongrec}, if entry $(i, j)$ is \texttt{1} (or \texttt{0}), 
it means the $i$-th song is (or not) found in the $j$-th playlist, 
and a question mark \texttt{?} means that we do not know whether the $i$-th song is found in the $j$-th playlist.
As a remark, columns represent playlists in test set contain only \texttt{1} and \texttt{?} entries.

\begin{figure}[!h]
\centering
\setlength{\tabcolsep}{1pt} % tweak the space between columns
\begin{tabular}{|*{7}{c}|ccccc|} \hline
%\rule{.3em}{0pt} 
\rule{0em}{10pt}
& \texttt{0} & \texttt{1} & \texttt{0} & $\cdots$ & \texttt{0} & & & \texttt{?} & $\cdots$ & \texttt{?} & \\
& \texttt{0} & \texttt{0} & \texttt{0} & $\cdots$ & \texttt{0} & & & \texttt{1} & $\cdots$ & \texttt{?} & \\
& \texttt{1} & \texttt{0} & \texttt{0} & $\cdots$ & \texttt{0} & & & \texttt{?} & $\cdots$ & \texttt{?} & \\
\vspace{-5pt}
& \texttt{0} & \texttt{0} & \texttt{0} & $\cdots$ & \texttt{1} & & & \texttt{?} & $\cdots$ & \texttt{1} & \\
& $\vdots$ & $\vdots$ & $\vdots$ & $\vdots$ & $\vdots$ & & & $\vdots$ & $\vdots$ & $\vdots$ & \\
& \texttt{0} & \texttt{0} & \texttt{1} & $\cdots$ & \texttt{0} & & & \texttt{?} & $\cdots$ & \texttt{?} & \\ \hline
\end{tabular}
\caption{Illustration of augmenting playlists as multi-label classification.}
\label{fig:pla}
\end{figure}


\paragraph{Dataset:}
We again use the AotM-2011~\cite{mcfee2012hypergraph} and 30Music~\cite{30music2015} playlist dataset,
the pre-process of the two dataset is the same as that in Section~\ref{ssec:newsongrec}.
Besides making use of the audio features and genres of songs,
we also use the popularity of a given song as a feature, which is defined as the number of occurrence in all playlists,
including the partial playlists in test set.

Table~\ref{tab:stats_pla} summarises the statistics of the two dataset used for this task.

\begin{table}[!h]
\centering
\caption{Statistics of dataset for playlist augmentation}
\label{tab:stats_pla}
%\resizebox{\linewidth}{!}{
\begin{tabular}{ccccccc}
\toprule
Dataset & \#Users & \#Songs & \#Playlists (train/dev/test)  & \#Song Features \\
\midrule
AotM-2011 & 14,182 & 114,428 & 60,260 / 12,225 / 12,225 & 218 \\
30Music   & 8,070  & 45,468  & 15,591 / 933 / 933       & 218 \\
\bottomrule
\end{tabular}
%}
\end{table}

\paragraph{Experimental design:}
In each dataset, we create the test set such that it contains 20\% of each user's playlists (chosen uniformly at random),
if the user has 5 or more playlists, the validation set is constructed the same as the test test.
All remaining playlists are used for training.
As a remark, we observed all songs during training.


\subsubsection{A few conclusions}

\paragraph{Which type of loss is most helpful?}

{\bf Row-wise loss}: weighting by the number of positive/negative labels for each example.
\ie we perform a classification/bipartite ranking task on each multilabel example 
which forms a dataset of examples with binary labels: $\{(\x_n, y_k\}_{k=1}^K$ for the $n$-th multilabel example.

\begin{equation*}
%\resizebox{\linewidth}{!}{$
\RCal_\textsc{row} 
= \displaystyle \sum_s 
  \frac{1}{K_+^s} \sum_{s \in pl} e^{-(\w_{pl}^\top \phi(s) + b)} +
  \frac{1}{K_-^s} \sum_{s \notin pl} \frac{1}{p} e^{p \w_{pl}^\top \phi(s)}.
%$}
\end{equation*}
where normalising factor $K_+^s$ is the number of playlists that include song $s$,
and $K_-^s$ is the number of playlists that do not include song $s$.


{\bf Column-wise loss}: weighting by the number of positive/negative examples for each label.
\ie we perform a classification/bipartite ranking task on each label which forms a binary dataset:
$\{\x_n, y_k\}_{n=1}^N$ for the $k$-th label.

\begin{equation*}
%\resizebox{\linewidth}{!}{$
\RCal_\textsc{col} 
= \displaystyle \sum_{pl}
  \frac{1}{N_+^{pl}} \sum_{s \in pl} e^{-(\w_{pl}^\top \phi(s) + b)} +
  \frac{1}{N_-^{pl}} \sum_{s \notin pl} \frac{1}{p} e^{p \w_{pl}^\top \phi(s)}.
%$}
\end{equation*}
where normalising factor $N_+^{pl}$ is the number of songs in playlist $pl$,
and $N_-^{pl}$ is the number of songs in a music library that playlist $pl$ does not include.


{\bf Row-wise + column-wise loss}: the summation of both: $\RCal_\textsc{row} + C \RCal_\textsc{col}$ 
where $C$ is a trade-off parameter.

The binary relevance baseline is learning a logistic regression for each playlist independently.


\begin{table}[!h]
\centering
\caption{Empirical results}
%\resizebox{\linewidth}{!}{
\begin{tabular}{l|ccccc}
\toprule
{}            & $\RCal_\textsc{example}$ & $\RCal_\textsc{label}$ & $\RCal_\textsc{both}$ & Independent L.R. & Pop-rank \\
\midrule
%AotM-2011     & 0.6827396 & 0.743770 & 0.7385298 & 0.6924 & 0.80199 \\
AotM-2011     & 0.68459 & 0.747755 & 0.7429  & 0.6924 & 0.80199 \\
30Music       & 0.7168  & 0.76867  & 0.76917 & 0.7225 & 0.7165 \\
%30Music       & 0.56766179 & 0.6350941 & 0.63555 & 0.575567 & 0.80558 \\
\bottomrule
\end{tabular}
%}
\end{table}

\paragraph{Experimental design:}
C: 1, 1, 1, p: 1, no multi-task regularisation

\paragraph{Is multi-task regularisation helpful?}

multi-task regularisation: we regularise the difference of playlist parameters 
such that $\|\w_j - \w_k\|_2$ is small if playlist $j$ and $k$ belong to the same user.

\begin{equation*}
\RCal_\textsc{reg} = \frac{1}{\sum_u N_u (N_u - 1)} \sum_u \sum_{j, k \in u} (\w_j - \w_k)^\top (\w_j - \w_k)
\end{equation*}
where $N_u$ is the number of playlist user $u$ has.

\begin{table}[!h]
\centering
\caption{Empirical results}
%\resizebox{\linewidth}{!}{
\begin{tabular}{l|ccc}
\toprule
{}            & Multi-task Reg. + $\RCal_\textsc{example}$ & Multi-task reg. + $\RCal_\textsc{label}$ \\
\midrule
%AotM-2011     & 0.6882099 & 0.7615590 \\
AotM-2011     & 0.69167 & 0.7819 \\
30Music       & 0.7177  & 0.7840 \\
%30Music       & 0.581426 & 0.6597  \\
%30Music       & 0.574175 & 0.667804  \\
%AUC           & 0.66583  & 0.68517   \\
\bottomrule
\end{tabular}
%}
\end{table}



\TODO
measure performance by AUC and HitRate@K,
compare with baselines such as independent logistic regression (\ie binary relevance), popularity based recommendation,
and matrix factorisation.



\subsection{Discussion}

{\it the choice of playlist dataset?}

