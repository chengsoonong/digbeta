\section{Multitask learning for music recommendation}

In this work, we treat a playlist as a set of songs by discarding the order of songs.
Suppose we have a dataset $\DCal$ with $N$ playlists from $U$ users, songs in every playlist are from a music collection 
with $M$ songs, and each song in the collection is at least in one playlist.

The task of recommending a set of songs (to form a playlist) for a given user is illustrated in Figure~\ref{fig:pla},
where rows represent songs (no specific order) and columns represent playlists (no specific order).
Further, columns with white colour represent playlists in training set, 
and columns with grey colour represent playlists that should be augmented (\ie test set).
If entry $(m, i)$ is \texttt{1} (or \texttt{0}), 
it means song $m$ can (or not) be found in playlist $i$,
and a question mark \texttt{?} means that we do not know whether song $m$ is in playlist $i$.
As a remark, columns represent playlists in test set contain only \texttt{?} entries.

\begin{figure}[!h]
\centering
\newcolumntype{a}{>{\columncolor[gray]{0.9}}c}  % define a new column type
\setlength{\tabcolsep}{1pt} % tweak the space between columns
%\begin{tabular}{|*{7}{c}|ccccc|} \hline
\begin{tabular}{|*{7}{c}|aaaaa|} \hline
%\rule{.3em}{0pt} 
\rule{0em}{10pt}
& \texttt{0} & \texttt{1} & \texttt{0} & $\cdots$ & \texttt{0} & & & \texttt{?} & $\cdots$ & \texttt{?} & \\
& \texttt{0} & \texttt{0} & \texttt{0} & $\cdots$ & \texttt{0} & & & \texttt{?} & $\cdots$ & \texttt{?} & \\
& \texttt{1} & \texttt{0} & \texttt{0} & $\cdots$ & \texttt{0} & & & \texttt{?} & $\cdots$ & \texttt{?} & \\
%\vspace{-5pt}
\vspace{-3pt}
& \texttt{0} & \texttt{0} & \texttt{0} & $\cdots$ & \texttt{1} & & & \texttt{?} & $\cdots$ & \texttt{?} & \\
& $\vdots$ & $\vdots$ & $\vdots$ & $\vdots$ & $\vdots$ & & & $\vdots$ & $\vdots$ & $\vdots$ & \\
& \texttt{0} & \texttt{0} & \texttt{1} & $\cdots$ & \texttt{0} & & & \texttt{?} & $\cdots$ & \texttt{?} & \\ \hline
\end{tabular}
\caption{Illustration of music recommendation for playlist.}
\label{fig:pla}
\end{figure}



\paragraph{Song scoring}
%
We aim to learn a function $f(i, m)$ that scores the affinity between playlist $i$ and song $m$.
Suppose function $f(i, m)$ has a linear form, \ie
$$
f(i, m) = \w_i^\top \x_m,
$$
where vector $\w_i$ represents playlist $i$ and $\x_m$ is a feature vector for song $m$.

We assume that playlist of a user can help the prediction of songs in another playlist of the same user,
further, playlists of one user can also help predict songs in playlist of another user.
In other words, we jointly learn from multiple tasks where each task is to predict a set of songs in a playlist.
We therefore decompose the representation of a playlist into three components, formally,
$$
\w_i = \bv_i + \bu_j + \widebar\bu,
$$
where $\bv_i$ is a specific for playlist $i$, $\bu_j$ is the component that for user $j=u(i)$ that owns playlist $i$,
and $\widebar\bu$ is a shared representation for all users.


\paragraph{Multitask learning objective}
%
The learning task is to minimise the empirical risk of scoring function $f(\dot)$ on $\DCal$ over model parameters $\Theta$,
which represents all parameters in $\bv_i, \bu_j, \widebar\bu, \, i=1,\dots,N, \, j=1,\dots,U$.
Formally,
$$
\min_{\Theta} \, \RCal(f, \DCal).
$$

Another assumption we make in this work is playlists of the \emph{same} user have \emph{similar} representations,
\ie we want the difference between the representations of playlist $i$ and $i'$ to be small,
we can formalise the assumption by requiring $\|\w_i - \w_{i'}\|_2^2$ be a small number. 
Note that
$$
\|\w_i - \w_{i'}\|_2^2 = \|\bv_i + \bu_j + \widebar\bu - (\bv_{i'} + \bu_j + \widebar\bu) \|_2^2 = \|\bv_i - \bv_{i'}\|_2^2,
$$
one way to make this quantity small is requiring $\bv$ be a sparse vector by adding a L1 regularisation term 
to our learning objective, \ie
$$
\min_{\Theta} \, \RCal(f, \DCal) + \lambda \sum_{i=1}^N \|\w_i\|_1.
$$

The last assumption we make is that representations of different users are different, 
but they nonetheless share a small number of features in their representations.
This assumption can be formalised similarly by imposing an additional L1 regularisation term $\|\widebar\bu|_1$.

In summary, our multitask learning objective is
$$
\min_{\Theta} \, \RCal(f, \DCal) 
   + \lambda_1 \sum_{j=1}^U \|\bu_j\|_2^2 + \lambda_2 \sum_{i=1}^N \|\w_i\|_1 + \lambda_3 \|\widebar\bu\|_1,
$$
where we add an L2 regularisation term to penalise large values in user representation,
and $\lambda_1, \lambda_2, \lambda_3 \in \R_+$ are regularisation constants.
