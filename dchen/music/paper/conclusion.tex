\section{Conclusion}
We investigated the problem of recommending songs to form playlist for a given user.
A multitask learning objective was proposed that aims to provide a decent default recommendation 
if the given user is new to system, and make better recommendations if the given user already has 
a few playlists accessible to the system.
We learned the multitask objective from playlists of existing users by using a ranking approach 
which ranks songs in playlists higher than those are not.
Parameters of the multitask objective are optimised by solving either a constrained optimisation problem 
or an unconstrained optimisation problem which approximates the objective but results in more efficient training.
Empirical experiments on two real playlist datasets show the proposed approaches work effectively.

We are aware of a few limitations of the proposed approaches, which we leave as future work.
Specifically, additional data sources (\eg music information shared on social media) or song and 
user features (\eg lyrics, user profile), as well as the sequential order of songs could provide 
additional information that could help make better recommendations.
Further, non-linear models such as deep neural networks have shown strong performance in a wide arrange of tasks,
and the linear model with sparse parameters in this work could potentially be more compact with few parameters 
if non-linear objective is employed.
Finally, as a remark, we want to mention the challenge the community faced when evaluating recommended results.
While metrics in information retrieval are commonly used, recommender system is more like a generative process
than a information retrieval task. Fortunately, this challenge has been noticed and been attacked in many 
ways~\cite{schedl2017}, we believe that promising automatic evaluation methods that accepted by the (majority of) 
community is one premise of significant progress in music recommendation.
