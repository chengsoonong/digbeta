% !TEX root = ./main.tex

\appendix
%\section{Supplementary material to Structured Recommendation}
%\label{sec:supplement}
%{\Large\bf Supplementary material to Structured Recommendation}
\begin{center}
  {\Large\bf Supplementary material for ``Trajectory Recommendation via Structured Prediction''}
\end{center}

%\setlength{\belowdisplayskip}{2pt} \setlength{\belowdisplayshortskip}{1pt}
%\setlength{\abovedisplayskip}{2pt} \setlength{\abovedisplayshortskip}{1pt}

\section{Model learning and prediction}
\label{sec:supplement}

In this section, we describe the 1-slack formulation for the proposed SR model 
and the details of the list Viterbi algorithm.

\subsection{1-slack formulation for the SR model}
\label{ssec:1slack_sr}

We can use \emph{one} slack variable to represent the sum of the $N$ hinge losses:
\begin{equation*}
%%\label{eq:hingeloss}
%%\resizebox{1.1\linewidth}{!}{
%%\begin{minipage}{\linewidth}
%\begin{align*}
\xi_i = \max \left( 0, \, 
        \max_{\bar{\y} \in \mathcal{Y}}
        \left\{ \Delta(\y^{(i)}, \bar{\y}) + \w^\top \Psi(\x^{(i)}, \bar{\y}) \right\} - \w^\top \Psi(\x^{(i)}, \y^{(i)}) \right).
%\end{align*}
%%\end{minipage}
%%}
\end{equation*}
Which results in the 1-slack formulation for the SR model:
\begin{equation*}
%%\label{eq:1slack_ml}
%\resizebox{0.9\linewidth}{!}{$
\begin{aligned}
\min_{\w, \, \xi \ge 0} ~\frac{1}{2} \w^\top \w + C \xi, ~~s.t.~ \frac{1}{N} \left( \sum_{i,j} \w^\top \Psi(\x^{(i)}, \y^{(ij)}) - \w^\top \Psi(\x^{(i)}, \bar{\y}^{(i)}) \right) 
  \ge \frac{1}{N} \sum_{i,j} \Delta(\y^{(ij)}, \bar{\y}^{(i)}) - \xi.
\end{aligned}
%$}
\end{equation*}


\subsection{The list Viterbi algorithm}
\label{sec:listviterbi-supp}

We make use of the (serial) list Viterbi in four situations:
\begin{enumerate}
  \item To avoid sequence with loops during the prediction phase of the SP and SR models
  \item To make top-$k$ prediction using the SP and SR models
  \item To eliminate known ground truths during the training phase (\ie loss augmented inference) of the SR and \textsc{SRpath} models
  \item To avoid sequence with loops during the training phase (\ie loss augmented inference) of the \textsc{SPpath} and \textsc{SRpath} models
\end{enumerate}

Here, we describe two well-known proposals of the serial list Viterbi algorithm in the context of hidden Markov models (HMMs).
Suppose we have an HMM with states $\SSf_{t}$, observations $\OSf_{t}$, transitions $a(i, j) = \Pr( \SSf_{t+1} = j \mid \SSf_t = i )$, and emissions $b(i, k) = \Pr( \OSf_{t} = k \mid \SSf_t = i )$.
Suppose $s^*_{1:T}$ is the most likely length $T$ sequence given observations $\OSf_{1:T}$, and
$\delta(j, t)$ is the value of the best sequence up till position $t$ ending at state $j$ as computed by the Viterbi algorithm.

To find the second-best sequence with value $ M \defEq \max_{\SSf_{1:T} \neq s^*_{1 : T}}{\Pr( \SSf_{1:T}, \OSf_{1:T} )}. $
\citet{seshadri1994list} observed that $M = \bar{\delta}_{T+1}$, where $\bar{\delta}_t$ has a Viterbi-like recurrence
\begin{equation}
    \label{eqn:att-recurrence}
    %\resizebox{0.9\linewidth}{!}{$
    \begin{aligned}
        \bar{\delta}_t &\defEq 
        \indicator{t > 0} \cdot
        \max
        \begin{cases}
        \max_{i \neq s^*_{t-1}} \delta(i, t-1) \cdot a( i, s^*_{t} ) \cdot b( s^*_{t}, \OSf_{t} ) \\
        \bar{\delta}_{t - 1} \cdot a( s^*_{t - 1}, s^*_{t} ) \cdot b( s^*_{t}, \OSf_{t} ).
        \end{cases}
    \end{aligned}
    %$}
\end{equation}
Intuitively, $\bar{\delta}_t$ finds the value of the second-best sequence that merges with the best sequence by at least time $t$,
to find the third, \dots, k-th best sequences, this algorithm keeps track of the ``next-best'' sequence terminating at each state in the current list of best sequences.

On the other hand, \citet{nilsson2001sequentially} observed that $M = \max_t \widehat{\rho}_t$, where
\begin{align*}
	\widehat{\rho}_{t} &\defEq \max_{i \neq s^*_{t}} \max_{S_{t+1:T}} {\Pr( \SSf_{1:t-1} = s^*_{1:t-1}, \SSf_t = i, \SSf_{t+1:T}, \OSf_{1:T} )}.
\end{align*}
Intuitively, $\widehat{\rho}_t$ finds the value of the second-best sequence that first deviates from the best sequence exactly at time $t$,
We note that one can compute $\widehat{\rho}_{t}$ using $\eta_{i, j, t} \defEq \max_{\SSf \colon S_{t - 1} = i, \SSf_{t} = j} \Pr( \SSf_{1:T}, \OSf_{1:T} )$ \citep{nilsson2001sequentially},
which in turn can be computed from the ``backward'' analogue of $\delta$.
In the original work, \citeauthor{nilsson2001sequentially} illustrated this approache as cleverly partitions the search space into subsets of sequences that share a prefix with the current list of best sequences.

If we unroll the recurrence in Equation \ref{eqn:att-recurrence}, and note the definition of $\delta$, we have
%\resizebox{\linewidth}{!}{
%    \begin{minipage}{\linewidth}
        \begin{align*}
        	M &= \max_t \widehat{\mu}_t \\
        	\widehat{\mu}_{t} &\defEq \left[ \prod_{k = t+2}^T a( s^*_{k-1}, s^*_{k} ) \cdot b( s^*_{k}, \OSf_{k} ) \right] \cdot \max_{i \neq s^*_{t}} \delta(i, t) \cdot a( i, s^*_{t+1} ) \cdot b( s^*_{t+1}, \OSf_{t+1} ) \\
        	&= \max_{i \neq s^*_{t}} \max_{S_{1:t-1}} {\Pr( \SSf_{1:t-1}, \SSf_t = i, \SSf_{t+1:T} = s^*_{t+1:T}, \OSf_{1:T} )};
        \end{align*}
%    \end{minipage}
%}%
\ie, the two approaches computed the same quantities, 
while the former fixes the suffix of the candidate sequence, and the latter fixes the prefix. 
The case of finding the $K$-th best sequence can be shown in a similar manner.

For sequence recommendation, as the underlying structure of SSVM is a Markov chain, which means we can use the same algorithms described above, 
except that the emissions are uniform distributions since all states are observed.

%%There are two general approaches for generalising Viterbi decoding to when we require $k$
%%sequences to be decoded: by maintaining $k$ paths through the trellis while decoding; or by
%%careful book keeping of the best $k-1$ paths through the trellis found so far and avoiding them.
%%We choose the latter approach as it is more memory efficient.

%The serial list Viterbi algorithm~\cite{nilsson2001sequentially,seshadri1994list} maintains
%a heap (\ie priority queue) of potential solutions, which are then checked for the desired property (for example
%whether there are loops). Once the requisite number of trajectories with the desired
%property are found, the algorithm terminates (for example once $k$ trajectories without loops are found when performing top-k prediction)
%The heap is initialised by running forward-backward (Algorithm~\ref{alg:forward-backward}) followed by the vanilla Viterbi (Algorithm~\ref{alg:viterbi}).
%
%\begin{algorithm}[htbp]
%\caption{Forward-backward procedure~\cite{rabiner1989tutorial}}
%\label{alg:forward-backward}
%\begin{algorithmic}[1]
%  \STATE $\forall p_j \in \mathcal{P},~ \alpha_t(p_j) =
%          \begin{cases}
%          0,~ t = 1 \\
%          \max_{p_i \in \mathcal{P}} \left\{ \alpha_{t-1}(p_i) + \mathbf{w}_{ij}^\top \psi_{ij}(\mathbf{x}, p_i, p_j) +
%          \mathbf{w}_j^\top \psi_j(\mathbf{x}, p_j) \right\},~ t=2,\dots,l
%          \end{cases}$
%
%  \STATE $\forall p_i \in \mathcal{P},~ \beta_t(p_i) =
%          \begin{cases}
%          0,~ t = l \\
%          \max_{p_j \in \mathcal{P}} \left\{ \mathbf{w}_{ij}^\top \psi_{ij}(\mathbf{x}, p_i, p_j) +
%          \mathbf{w}_j^\top \psi_j(\mathbf{x}, p_j) + \beta_{t+1}(p_j) \right\},~ t = l-1,\dots,1
%          \end{cases}$
%
%  %\STATE $\forall p_i \in \mathcal{P},~ f_t(p_i) = \alpha_t(p_i) + \beta_t(p_i),~ t = 1,\dots,K$
%  \STATE $\forall p_i, p_j \in \mathcal{P},~ f_{t,t+1}(p_i, p_j) = \alpha_t(p_i) + \mathbf{w}_{ij}^\top \psi_{ij}(\mathbf{x}, p_i, p_j) +
%                                \mathbf{w}_j^\top \psi_j(\mathbf{x}, p_j) + \beta_{t+1}(p_j),~ t = 1,\dots,l-1$
%\end{algorithmic}
%\end{algorithm}
%
%\begin{algorithm}[htbp]
%\caption{Viterbi}
%\label{alg:viterbi}
%\begin{algorithmic}[1]
%  \STATE $y_t^1 = \begin{cases}
%                  s,~ t = 1 \\
%  %                \argmax_{p \in \mathcal{P}} \left\{ f_{1,2}(s, p) \right\},~ t = 2, \\
%                  \argmax_{p \in \mathcal{P}} \left\{ f_{t-1,t}(y_{t-1}^1, p) \right\},~ t = 2,\dots,l
%                  \end{cases}$
%
%  %\STATE $r^1 = \max_{p \in \mathcal{P}} \left\{ f_K(p) \right\}~~~ \triangleright$ $r^1$ is the score/priority of $\mathbf{y}^1$
%  \STATE $r^1 = \max_{p \in \mathcal{P}} \left\{ \alpha_{l}(p) \right\}~~~ \triangleright$ $r^1$ is the score/priority of $\mathbf{y}^1$
%\end{algorithmic}
%\end{algorithm}
%
%Given an existing heap containing potential trajectories,
%list Viterbi maintains a set of POIs $S$ to exclude, which is updated
%sequentially by considering the front sequence of the heap.
%
%Recall that for trajectory recommendation we are given the query $\mathbf{x}=(s, l)$, where
%$s$ is the starting POI from the set of POIs $\mathcal{P}$,
%and $l$ is the desired length of the trajectory.
%We assume the score function is of the form $\mathbf{w}^\top \Psi$ where $\Psi$ is the joint
%feature vector. $\mathbf{w}$ could be the value of the weight in the current iteration in training,
%or the learned weight vector during prediction.
%
%The list Viterbi algorithm for performing top-$k$ prediction is described in Algorithm~\ref{alg:listviterbi}.
%To eliminating known ground truths in loss augmented inference,
%we modify the forward-backward procedure (Algorithm~\ref{alg:forward-backward}) to account for the loss term $\Delta(\cdot,\cdot)$,
%and Algorithm~\ref{alg:viterbi} and Algorithm~\ref{alg:listviterbi} can be used without modification.

%\begin{algorithm}[htbp]
%\caption{The list Viterbi algorithm for top-$K$ prediction~\cite{nilsson2001sequentially,seshadri1994list}}
%\label{alg:listviterbi}
%\begin{algorithmic}[1]
%\STATE \textbf{Input}: $\mathbf{x}=(s, l),~ \mathcal{P},~ \mathbf{w},~ \Psi, ~K$
%%\STATE Initialise score matrices $\alpha,~ \beta,~ f_t,~ f_{t, t+1}$, a max-heap $H,~ k=0$.
%\STATE Initialise score matrices $\alpha,~ \beta,~ f_{t, t+1}$, a max-heap $H$, result set $R$, $k=0$.
%\STATE $\triangleright$ Do the forward-backward procedure (Algorithm~\ref{alg:forward-backward})
%\STATE $\triangleright$ Identify the best (scored) trajectory $\mathbf{y}^1=(y_1^1,\dots,y_l^1)$
%       with score $r^1$ using Viterbi (Algorithm~\ref{alg:viterbi}). $\y^1$ may be a trajectory that violates the desired condition.
%\STATE $H.\textit{push}\left(r^1,~ (\mathbf{y}^1, \textsc{nil}, \emptyset) \right)$
%\STATE Set $R=\emptyset$, the list of trajectories to be returned.
%\WHILE{$H \ne \emptyset$ \textbf{and} $k < \,|\mathcal{P}|^{l-1} - \prod_{t=2}^l (|\mathcal{P}|-t+1) + K$}
%    \STATE $r^k,~ (\mathbf{y}^k, I, S) = H.\textit{pop}()~~~ \triangleright$
%           $r^k$ is the score of $\mathbf{y}^k=(y_1^k,\dots,y_l^k)$, $I$ is the partition index, and $S$ is the exclude set
%    \STATE $k = k + 1$
%    \STATE Add $\mathbf{y}^k$ to $R$ if it satisfies the desired property
%    \RETURN $R$ if it contains the required number of trajectories
%    \STATE $I' = \begin{cases}
%                  2, & I = \textsc{nil} \\
%                  I, & \text{otherwise}
%                 \end{cases}$
%
%    \FOR{$t = I',\dots,l$}
%        \STATE $S' = \begin{cases}
%                      S \cup \{ y_t^k \}, & t = I' \\
%                      \{ y_t^k \},        & \text{otherwise}
%                     \end{cases}$
%
%        \STATE $y'_j = \begin{cases}
%                            y_j^k,                                                                             & j=1,\dots,t-1 \\
%                            \argmax_{p \in \mathcal{P} \setminus S'} \left\{ f_{t-1,t}(y_{t-1}^k, p) \right\}, & j=t \\
%                            \argmax_{p \in \mathcal{P}} \left\{ f_{j-1, j}(y'_{j-1}, p) \right\},              & j=t+1,\dots,l
%                \end{cases}$
%        %\STATE $\bar{r} = \begin{cases}
%        %                  f_{t-1,t}(y_{t-1}^k, \bar{y}_t),&I = \textsc{nil} \\
%        %                  r^k + f_{t-1,t}(y_{t-1}^k, \bar{y}_t) - f_{t-1,t}(y_{t-1}^k, y_t^k), &\text{otherwise}
%        %                  \end{cases}$
%        \STATE \vspace{3pt}$r' = r^k + f_{t-1,t}(y_{t-1}^k, y'_t) - f_{t-1,t}(y_{t-1}^k, y_t^k)$
%
%        \STATE \vspace{3pt}$H.\textit{push}(r', (\y', t, S') )$ \vspace{2pt}
%    \ENDFOR
%\ENDWHILE
%\end{algorithmic}
%\end{algorithm}

%\clearpage

\section{Experiment}

In this section, we describe trajectory dataset used in experiments as well as features for each methods.
In addition, the details of evaluation and empirical results are also described.

\subsection{Photo trajectory dataset and features}
\label{sec:feature}

%\textbf{Dataset}.
\subsubsection{Dataset}
In the interests of reproducibility we present further details of our empirical experiment.
The histogram of the number of trajectories per query is shown in Figure~\ref{fig:hist_query},
where we can see each query has 4-9 ground truths (\ie trajectories) on average, and 30-60 trajectories at most.
The histogram of trajectory length (\ie the number of POIs in a trajectory) is shown in Figure~\ref{fig:hist_length},
where we can see the majority are short trajectories (\ie length $\le$ 5).

% % histogram of #ground truth
% \begin{figure}[t]
% 	\centering
% 	\includegraphics[width=.7\linewidth]{hist_query.pdf}
% 	\caption{Histograms of the number of trajectories per query.}
% 	\label{fig:hist_query}
% \end{figure}
% %
% %
% %% histogram of trajectory length
% \begin{figure}[t]
% 	\centering
% 	\includegraphics[width=.7\linewidth]{hist_length.pdf}
% 	\caption{Histograms of trajectory length.}
% 	\label{fig:hist_length}
% \end{figure}


%\textbf{Features}.
\subsubsection{Features}
The POI-query features used by \textsc{PoiRank}, SP and SR methods and their extensions 
(\ie the \textsc{SPpath} and \textsc{SRpath} models) are shown in Table~\ref{tab:poifeature},
pairwise features used in SP and SR methods and their extensions are shown in Table~\ref{tab:tranfeature}.

\begin{table}[!h]
\small
\setlength{\tabcolsep}{10pt} % tweak the space between columns
\begin{tabular}{l|l} \hline
\textbf{Feature}       & \textbf{Description} \\ \hline
\texttt{category}      & one-hot encoding of the category of $p$ \\
\texttt{neighbourhood} & one-hot encoding of the POI cluster that $p$ resides in \\
\texttt{popularity}    & logarithm of POI popularity of $p$ \\
\texttt{nVisit}        & logarithm of the total number of visit by all users at $p$ \\
\texttt{avgDuration}  & logarithm of the average visit duration at $p$ \\
\hline
%\texttt{nOccurrence}            & the number of times $p$ occurred in a trajectory that satisfies the query \\ DON'T know given new query

\texttt{trajLen}           & trajectory length $l$, i.e., the number of POIs required \\
\texttt{sameCatStart}      & $1$ if the category of $p$ is the same as that of $s$, $-1$ otherwise \\
\texttt{sameNeighbourhoodStart} & $1$ if $p$ resides in the same POI cluster as $s$, $-1$ otherwise \\
\texttt{diffPopStart}    & real-valued difference in POI popularity of $p$ from that of $s$ \\
\texttt{diffNVisitStart}        & real-valued difference in the total number of visit at $p$ from that at $s$ \\
\texttt{diffDurationStart}  & real-valued difference in average duration at $p$ from that at $s$ \\
\texttt{distStart}          & distance between $p$ and $s$, calculated using the Haversine formula \\
\hline
\end{tabular}
\caption{POI-query features: features of POI $p$ with respect to query $(s,l)$}
\label{tab:poifeature}
\centering
\end{table}



\begin{table}[!h]
\centering
\small
\setlength{\tabcolsep}{2pt} % tweak the space between columns
\begin{tabular}{l|l} \hline
\textbf{Feature}       & \textbf{Description} \\ \hline
\texttt{category}      & category of POI \\
\texttt{neighbourhood} & the cluster that a POI resides in \\
\texttt{popularity}    & (discretised) popularity of POI \\
\texttt{nVisit}        & (discretised) total number of visit at POI \\
\texttt{avgDuration}  & (discretised) average duration at POI \\ \hline
\end{tabular}
\caption{Pairwise POI features}
\label{tab:tranfeature}
\end{table}


%\clearpage
\subsection{Evaluation settings}
\label{sec:metric}

%\textbf{Top-k prediction for baselines}.
\subsubsection{Top-k prediction for baselines}
\begin{itemize}
\item To perform top-$k$ prediction with \textsc{Random} baseline, we simply repeat the \textsc{Random} method $k$ times.
\item To perform top-$k$ prediction with \textsc{Popularity} and \textsc{PoiRank}, we make use of the list Viterbi algorithm 
      %(Algorithm~\ref{alg:listviterbi} 
      to get $k$ best scored paths, in particular, 
      for \textsc{Popularity}, the score of a path is the accumulated popularity of all POIs in the path; 
      for \textsc{PoiRank}, the score of a path is the likelihood 
      (the ranking scores for POIs are first transformed into a probability distribution using the softmax function, as described in~\cite{cikm16paper}).
\end{itemize}

To evaluate the performance of a certain recommendation algorithm,
we need to measure the similarity (or loss) given prediction $\hat{\mathbf{y}}$
and ground truth $\mathbf{y}$.


%\textbf{F$_1$ score on points}.
\subsubsection{F$_1$ score on points}
F$_1$ score on points~\cite{ijcai15} cares about only the set of correctly recommended POIs.
%%Let $\texttt{set}(\mathbf{y})$ denote the set of POIs in trajectory $\mathbf{y}$, F$_1$ score on points is defined as
\begin{equation*}
F_1(\mathbf{y}, \hat{\mathbf{y}}) = \frac{2  P_{\textsc{point}}  R_{\textsc{point}}}{P_{\textsc{point}} + R_{\textsc{point}}}
%~~\text{where}~
%P_{\textsc{point}} = \frac{| \texttt{set}(\hat{\mathbf{y}}) \cap \texttt{set}(\mathbf{y}) |}{| \texttt{set}(\hat{\mathbf{y}}) |}~\text{and}~
%R_{\textsc{point}} = \frac{| \texttt{set}(\hat{\mathbf{y}}) \cap \texttt{set}(\mathbf{y}) |}{| \texttt{set}(\mathbf{y}) |}.
\end{equation*}
where $P_\textsc{point}$, $R_\textsc{point}$ are respectively the precision and recall for points in $\hat\y$ and $\y$.
If $| \hat{\mathbf{y}} | = | \mathbf{y} |$, this metric is just the unordered Hamming loss,
i.e., Hamming loss between two binary indicator vectors of size $| \mathcal{P} |$.

%\textbf{F$_1$ score on pairs}.
\subsubsection{F$_1$ score on pairs}
To take into account the orders in recommended sequence, 
we also use the F$_1$ score on pairs~\cite{cikm16paper} measure, which considers the set of correctly predicted POI pairs,
\begin{equation*}
\text{pairs-F}_1(\mathbf{y}, \hat{\mathbf{y}}) = \frac{2 P_{\textsc{pair}} R_{\textsc{pair}}}{P_{\textsc{pair}} + R_{\textsc{pair}}}
%~~\text{where}~
%P_{\textsc{pair}} = \frac{N_c} {| \texttt{set}(\hat{\mathbf{y}}) | (| \texttt{set}(\hat{\mathbf{y}}) | - 1) / 2}~\text{and}~
%R_{\textsc{pair}} = \frac{N_c} {| \texttt{set}(\mathbf{y}) | (| \texttt{set}(\mathbf{y}) | - 1) / 2},
\end{equation*}
where $P_\textsc{point}$, $R_\textsc{point}$ are respectively the precision and recall for all possible pairs of $\hat\y$ and $\y$.
%%\eat{
%%and $N_c = \sum_{j=1}^{| \mathbf{y} | - 1} \sum_{k=j+1}^{| \mathbf{y} |} \llb y_j \prec_{\bar{\mathbf{y}}} y_k \rrb$,
%%here $y_j \prec_{\bar{\mathbf{y}}} y_k$ denotes that POI $y_j$ appears before POI $y_k$ in trajectory $\bar{\mathbf{y}}$.
%%We define pairs-F$_1 = 0$ when $N_c = 0$.
%%}


%\textbf{Kendall's $\tau$ with ties}
\subsubsection{Kendall's $\tau$ with ties}
Alternatively, we can cast a trajectory $\y = y_{1:l}$ as a ranking of POIs in $\mathcal{P}$,
where $y_j$ has a rank $| \mathcal{P} | - j + 1$ and any other POI $p \notin \mathbf{y}$ has a rank $0$ ($0$ is an arbitrary choice),
then we can make use of ranking evaluation metrics such as Kendall's $\tau$ by taking care of ties in ranks.
In particular, given a prediction $\hat\y = \hat{y_{1:l}}$ and ground truth $\y = y_{1:l}$,
we produce two ranks for $\mathbf{y}$ and $\hat{\mathbf{y}}$ with respect to 
a specific ordering of POIs $(p_1, p_2, \dots, p_{|\mathcal{P}|})$:
\begin{align*}
r_i       &= \sum_{j=1}^l (| \mathcal{P} | - j + 1)  \llb p_i = y_j \rrb,~
i = 1, \dots, | \mathcal{P} | \\
\hat{r}_i &= \sum_{j=1}^l (| \mathcal{P} | - j + 1)  \llb p_i = \hat{y}_j \rrb,~
i = 1, \dots, | \mathcal{P} |
\end{align*}
where POIs not in $\mathbf{y}$ will have a rank of $0$.
Then we compute the following metrics:
\begin{itemize}
\item the number of concordant pairs \(
      C = \frac{1}{2} \sum_{i,j} \left(\llb r_i < r_j \rrb  \llb \hat{r}_i < \hat{r}_j \rrb +
                      \llb r_i > r_j \rrb  \llb \hat{r}_i > \hat{r}_j \rrb \right) \)
\item the number of discordant pairs \(
      D = \frac{1}{2} \sum_{i,j} \left(\llb r_i < r_j \rrb  \llb \hat{r}_i > \hat{r}_j \rrb +
                      \llb r_i > r_j \rrb  \llb \hat{r}_i < \hat{r}_j \rrb \right) \)
\item the number of ties in ground truth $\y$: \(
      T_{\mathbf{y}} = \frac{1}{2} \sum_{i \ne j} \llb r_i = r_j \rrb 
%                     = \frac{1}{2} \sum_{i \ne j} \llb r_i = 0 \rrb  \llb r_j = 0 \rrb 
                     = \frac{1}{2} \left( |\mathcal{P}| - l \right) \left( |\mathcal{P}| - l - 1 \right) \)
\item the number of ties in prediction $\hat\y$: \(
      T_{\hat{\mathbf{y}}} = \frac{1}{2} \sum_{i \ne j} \llb \hat{r}_i = \hat{r}_j \rrb 
%                           = \frac{1}{2} \sum_{i \ne j} \llb \hat{r}_i = 0 \rrb  \llb \hat{r}_j = 0 \rrb 
                           = \frac{1}{2} \left( |\mathcal{P}| - l \right) \left( |\mathcal{P}| - l - 1 \right) \)
\item the number of ties in both $\y$ and $\hat\y$: \(
      T_{\mathbf{y},\hat{\mathbf{y}}} = \frac{1}{2} \sum_{i \ne j} \llb r_i = r_j \rrb  \llb \hat{r}_i = \hat{r}_j \rrb \)
%                                      = \frac{1}{2} \sum_{i \ne j} \llb r_i = 0 \rrb  \llb r_j = 0 \rrb
%                                        \llb \hat{r}_i = 0 \rrb  \llb \hat{r}_j = 0 \rrb \)
\end{itemize}
Kendall's $\tau$ (version $b$)~\cite{agresti2010analysis} is %,kendall1945} is
\begin{equation*}
\tau_b(\mathbf{y}, \hat{\mathbf{y}}) = \frac{C - D}{\sqrt{(C + D + T) (C + D + U)}},
\end{equation*}
where $T = T_{\mathbf{y}} - T_{\mathbf{y},\hat{\mathbf{y}}}$ and $U = T_{\hat{\mathbf{y}}} - T_{\mathbf{y},\hat{\mathbf{y}}}$.

%Furthermore, F$_1$ score on points can be written as
%\begin{equation*}
%F_1(\mathbf{y}, \hat{\mathbf{y}}) = \frac{1}{l} \sum_i \llb r_i > 0 \rrb  \llb \hat{r}_i > 0 \rrb,
%\end{equation*}
%and F$_1$ score on pairs can be written as
%\begin{align*}
%& \text{pairs-F}_1(\mathbf{y}, \hat{\mathbf{y}}) \\
%&= \left( \frac{1}{2} \sum_{i,j} \llb r_i < r_j \rrb  \llb r_i > 0 \rrb \llb \hat{r}_i < \hat{r}_j \rrb  \llb \hat{r}_i > 0 \rrb \right. \\
%&  \left. ~+ \frac{1}{2} \sum_{i,j} \llb r_i > r_j \rrb  \llb r_j > 0 \rrb \llb \hat{r}_i > \hat{r}_j \rrb  \llb \hat{r}_j > 0 \rrb \right)
%   \cdot \frac{1}{l(l-1)/2} \\
%&= \frac{\sum_{i,j} \llb r_i < r_j \rrb  \llb r_i > 0 \rrb \llb \hat{r}_i < \hat{r}_j \rrb  \llb \hat{r}_i > 0 \rrb +
%         \sum_{i,j} \llb r_i > r_j \rrb  \llb r_j > 0 \rrb \llb \hat{r}_i > \hat{r}_j \rrb  \llb \hat{r}_j > 0 \rrb}
%        {l(l-1)}
%\end{align*}

%\clearpage
\subsection{Empirical results}

%\textbf{The effects of $k$ for top-$k$ prediction}.
\subsubsection{The effects of $k$ for top-$k$ prediction}
The performance of baselines and structured recommendation algorithms for top-$k$ ($k=1,3,5,10$) 
prediction are shown in the following tables.
Dark gray entries: best performing method for each metric; light gray entries: the \textit{next best}. 

% topk evaluation table
\begin{table*}[!h]
\centering
\small
\setlength{\tabcolsep}{3pt} % tweak the space between columns
\begin{tabular}{l|cc|ccc|ccc} \hline
& \multicolumn{8}{c}{\bf Kendall's $\tau$} \\ \hline
 & \textsc{Random} & \textsc{Popularity} & \textsc{PoiRank} & \textsc{Markov} & \textsc{SP} & \textsc{SPpath} & \textsc{SR} & \textsc{SRpath} \\ \hline
Osaka & $0.420\pm0.030$ & $0.566\pm0.034$ & $\firstBest{0.644\pm0.040}$ & $0.600\pm0.036$ & $0.525\pm0.037$ & $0.525\pm0.039$ & $0.608\pm0.042$ & $\secondBest{0.613\pm0.044}$ \\
Glasgow & $0.430\pm0.031$ & $0.644\pm0.036$ & $\firstBest{0.733\pm0.030}$ & $0.623\pm0.030$ & $0.564\pm0.029$ & $0.615\pm0.034$ & $0.708\pm0.031$ & $\secondBest{0.712\pm0.031}$ \\
Toronto & $0.394\pm0.025$ & $0.626\pm0.023$ & $\secondBest{0.714\pm0.024}$ & $0.629\pm0.023$ & $0.543\pm0.026$ & $0.572\pm0.026$ & $0.714\pm0.026$ & $\firstBest{0.717\pm0.026}$ \\
\hline
& \multicolumn{8}{c}{\bf F$_1$ score on points} \\ \hline
Osaka & $0.459\pm0.027$ & $0.601\pm0.031$ & $\firstBest{0.678\pm0.037}$ & $0.630\pm0.034$ & $0.555\pm0.034$ & $0.558\pm0.036$ & $0.638\pm0.039$ & $\secondBest{0.645\pm0.040}$ \\
Glasgow & $0.478\pm0.027$ & $0.681\pm0.032$ & $\firstBest{0.764\pm0.027}$ & $0.654\pm0.027$ & $0.604\pm0.026$ & $0.653\pm0.031$ & $0.741\pm0.028$ & $\secondBest{0.743\pm0.028}$ \\
Toronto & $0.461\pm0.020$ & $0.671\pm0.021$ & $\secondBest{0.756\pm0.021}$ & $0.676\pm0.021$ & $0.594\pm0.023$ & $0.623\pm0.023$ & $0.753\pm0.023$ & $\firstBest{0.757\pm0.022}$ \\
\hline
& \multicolumn{8}{c}{\bf F$_1$ score on pairs} \\ \hline
Osaka & $0.104\pm0.037$ & $0.281\pm0.051$ & $\firstBest{0.428\pm0.059}$ & $0.331\pm0.053$ & $0.243\pm0.052$ & $0.254\pm0.055$ & $0.375\pm0.059$ & $\secondBest{0.401\pm0.060}$ \\
Glasgow & $0.154\pm0.035$ & $0.426\pm0.051$ & $\firstBest{0.545\pm0.046}$ & $0.368\pm0.045$ & $0.289\pm0.042$ & $0.389\pm0.048$ & $0.506\pm0.048$ & $\secondBest{0.516\pm0.048}$ \\
Toronto & $0.143\pm0.025$ & $0.381\pm0.034$ & $0.503\pm0.036$ & $0.391\pm0.034$ & $0.299\pm0.033$ & $0.340\pm0.035$ & $\secondBest{0.530\pm0.037}$ & $\firstBest{0.533\pm0.037}$ \\
\hline
\end{tabular}
\caption{Results on trajectory recommendation datasets on best of top-1.}
\end{table*}


\begin{table*}[!h]
\centering
\small
\setlength{\tabcolsep}{3pt} % tweak the space between columns
\begin{tabular}{l|cc|ccc|ccc} \hline
& \multicolumn{8}{c}{\bf Kendall's $\tau$} \\ \hline
 & \textsc{Random} & \textsc{Popularity} & \textsc{PoiRank} & \textsc{Markov} & \textsc{SP} & \textsc{SPpath} & \textsc{SR} & \textsc{SRpath} \\ \hline
Osaka & $0.556\pm0.037$ & $0.666\pm0.039$ & $\firstBest{0.726\pm0.042}$ & $\secondBest{0.718\pm0.039}$ & $0.630\pm0.044$ & $0.698\pm0.040$ & $0.711\pm0.042$ & $0.697\pm0.042$ \\
Glasgow & $0.563\pm0.031$ & $0.693\pm0.036$ & $0.781\pm0.030$ & $0.684\pm0.032$ & $0.666\pm0.033$ & $0.688\pm0.032$ & $\secondBest{0.803\pm0.029}$ & $\firstBest{0.808\pm0.030}$ \\
Toronto & $0.521\pm0.026$ & $0.670\pm0.025$ & $0.746\pm0.023$ & $0.712\pm0.023$ & $0.629\pm0.027$ & $0.650\pm0.027$ & $\firstBest{0.753\pm0.025}$ & $\secondBest{0.749\pm0.024}$ \\
\hline
& \multicolumn{8}{c}{\bf F$_1$ score on points} \\ \hline
Osaka & $0.587\pm0.034$ & $0.691\pm0.035$ & $\firstBest{0.750\pm0.039}$ & $\secondBest{0.740\pm0.037}$ & $0.656\pm0.040$ & $0.724\pm0.037$ & $0.735\pm0.038$ & $0.723\pm0.039$ \\
Glasgow & $0.598\pm0.028$ & $0.722\pm0.033$ & $0.803\pm0.027$ & $0.711\pm0.029$ & $0.698\pm0.030$ & $0.716\pm0.029$ & $\secondBest{0.825\pm0.026}$ & $\firstBest{0.829\pm0.026}$ \\
Toronto & $0.577\pm0.022$ & $0.704\pm0.023$ & $0.776\pm0.021$ & $0.748\pm0.021$ & $0.674\pm0.023$ & $0.693\pm0.023$ & $\firstBest{0.784\pm0.022}$ & $\secondBest{0.780\pm0.021}$ \\
\hline
& \multicolumn{8}{c}{\bf F$_1$ score on pairs} \\ \hline
Osaka & $0.288\pm0.055$ & $0.448\pm0.058$ & $\firstBest{0.578\pm0.060}$ & $0.538\pm0.060$ & $0.425\pm0.062$ & $0.511\pm0.059$ & $\secondBest{0.549\pm0.060}$ & $0.520\pm0.059$ \\
Glasgow & $0.300\pm0.043$ & $0.524\pm0.053$ & $0.625\pm0.046$ & $0.465\pm0.048$ & $0.464\pm0.049$ & $0.481\pm0.048$ & $\secondBest{0.666\pm0.045}$ & $\firstBest{0.678\pm0.045}$ \\
Toronto & $0.281\pm0.032$ & $0.473\pm0.036$ & $0.572\pm0.035$ & $0.517\pm0.035$ & $0.429\pm0.037$ & $0.461\pm0.037$ & $\firstBest{0.592\pm0.036}$ & $\secondBest{0.583\pm0.036}$ \\
\hline
\end{tabular}
\caption{Results on trajectory recommendation datasets on best of top-3.}
\end{table*}


\begin{table*}[!h]
\centering
\small
\setlength{\tabcolsep}{3pt} % tweak the space between columns
\begin{tabular}{l|cc|ccc|ccc} \hline
& \multicolumn{8}{c}{\bf Kendall's $\tau$} \\ \hline
 & \textsc{Random} & \textsc{Popularity} & \textsc{PoiRank} & \textsc{Markov} & \textsc{SP} & \textsc{SPpath} & \textsc{SR} & \textsc{SRpath} \\ \hline
 & \textsc{Random} & \textsc{Popularity} & \textsc{PoiRank} & \textsc{SP} & \textsc{SPpath} & \textsc{SR} & \textsc{SRpath} \\ \hline
Osaka & $0.618\pm0.038$ & $0.674\pm0.038$ & $\secondBest{0.750\pm0.040}$ & $\firstBest{0.769\pm0.036}$ & $0.678\pm0.045$ & $0.735\pm0.039$ & $0.741\pm0.039$ & $0.729\pm0.041$ \\
Glasgow & $0.623\pm0.029$ & $0.727\pm0.037$ & $0.801\pm0.030$ & $0.712\pm0.032$ & $0.727\pm0.033$ & $0.743\pm0.031$ & $\secondBest{0.826\pm0.028}$ & $\firstBest{0.832\pm0.028}$ \\
Toronto & $0.574\pm0.025$ & $0.687\pm0.025$ & $0.754\pm0.023$ & $0.749\pm0.024$ & $0.662\pm0.027$ & $0.683\pm0.026$ & $\firstBest{0.778\pm0.023}$ & $\secondBest{0.769\pm0.024}$ \\
\hline
& \multicolumn{8}{c}{\bf F$_1$ score on points} \\ \hline
Osaka & $0.646\pm0.035$ & $0.699\pm0.034$ & $\secondBest{0.772\pm0.037}$ & $\firstBest{0.789\pm0.033}$ & $0.700\pm0.041$ & $0.757\pm0.036$ & $0.761\pm0.036$ & $0.751\pm0.037$ \\
Glasgow & $0.655\pm0.026$ & $0.754\pm0.033$ & $0.821\pm0.026$ & $0.736\pm0.029$ & $0.755\pm0.030$ & $0.770\pm0.027$ & $\secondBest{0.847\pm0.024}$ & $\firstBest{0.850\pm0.025}$ \\
Toronto & $0.624\pm0.022$ & $0.719\pm0.023$ & $0.781\pm0.021$ & $0.783\pm0.021$ & $0.705\pm0.023$ & $0.724\pm0.022$ & $\firstBest{0.808\pm0.021}$ & $\secondBest{0.798\pm0.021}$ \\
\hline
& \multicolumn{8}{c}{\bf F$_1$ score on pairs} \\ \hline
Osaka & $0.375\pm0.058$ & $0.459\pm0.057$ & $\secondBest{0.607\pm0.058}$ & $\firstBest{0.621\pm0.055}$ & $0.507\pm0.064$ & $0.568\pm0.058$ & $0.584\pm0.058$ & $0.575\pm0.058$ \\
Glasgow & $0.377\pm0.044$ & $0.590\pm0.052$ & $0.670\pm0.045$ & $0.507\pm0.048$ & $0.563\pm0.048$ & $0.573\pm0.047$ & $\secondBest{0.701\pm0.043}$ & $\firstBest{0.715\pm0.044}$ \\
Toronto & $0.343\pm0.034$ & $0.500\pm0.036$ & $0.590\pm0.034$ & $0.581\pm0.034$ & $0.483\pm0.037$ & $0.509\pm0.037$ & $\firstBest{0.624\pm0.035}$ & $\secondBest{0.609\pm0.035}$ \\
\hline
\end{tabular}
\caption{Results on trajectory recommendation datasets on best of top-5.}
\end{table*}


\begin{table*}[!h]
\centering
\small
\setlength{\tabcolsep}{3pt} % tweak the space between columns
\begin{tabular}{l|cc|ccc|ccc} \hline
& \multicolumn{8}{c}{\bf Kendall's $\tau$} \\ \hline
 & \textsc{Random} & \textsc{Popularity} & \textsc{PoiRank} & \textsc{Markov} & \textsc{SP} & \textsc{SPpath} & \textsc{SR} & \textsc{SRpath} \\ \hline
Osaka & $0.685\pm0.035$ & $0.768\pm0.038$ & $0.787\pm0.037$ & $\firstBest{0.824\pm0.031}$ & $0.749\pm0.043$ & $0.791\pm0.036$ & $0.777\pm0.036$ & $\secondBest{0.803\pm0.034}$ \\
Glasgow & $0.703\pm0.029$ & $0.748\pm0.036$ & $0.830\pm0.029$ & $0.781\pm0.031$ & $0.790\pm0.030$ & $0.787\pm0.029$ & $\firstBest{0.868\pm0.026}$ & $\secondBest{0.853\pm0.026}$ \\
Toronto & $0.652\pm0.024$ & $0.719\pm0.024$ & $0.784\pm0.023$ & $0.789\pm0.022$ & $0.697\pm0.027$ & $0.719\pm0.026$ & $\firstBest{0.802\pm0.022}$ & $\secondBest{0.797\pm0.022}$ \\
\hline
& \multicolumn{8}{c}{\bf F$_1$ score on points} \\ \hline
Osaka & $0.703\pm0.032$ & $0.786\pm0.034$ & $0.804\pm0.034$ & $\firstBest{0.840\pm0.029}$ & $0.770\pm0.039$ & $0.809\pm0.033$ & $0.793\pm0.033$ & $\secondBest{0.820\pm0.031}$ \\
Glasgow & $0.731\pm0.026$ & $0.771\pm0.033$ & $0.847\pm0.025$ & $0.800\pm0.028$ & $0.810\pm0.027$ & $0.807\pm0.026$ & $\firstBest{0.883\pm0.023}$ & $\secondBest{0.868\pm0.023}$ \\
Toronto & $0.696\pm0.021$ & $0.746\pm0.022$ & $0.807\pm0.020$ & $0.819\pm0.019$ & $0.733\pm0.023$ & $0.755\pm0.022$ & $\firstBest{0.828\pm0.019}$ & $\secondBest{0.823\pm0.020}$ \\
\hline
& \multicolumn{8}{c}{\bf F$_1$ score on pairs} \\ \hline
Osaka & $0.451\pm0.057$ & $0.626\pm0.055$ & $0.661\pm0.056$ & $\firstBest{0.693\pm0.051}$ & $0.620\pm0.061$ & $0.664\pm0.055$ & $0.637\pm0.055$ & $\secondBest{0.671\pm0.053}$ \\
Glasgow & $0.495\pm0.046$ & $0.623\pm0.051$ & $0.726\pm0.043$ & $0.635\pm0.048$ & $0.658\pm0.046$ & $0.648\pm0.045$ & $\firstBest{0.770\pm0.039}$ & $\secondBest{0.746\pm0.041}$ \\
Toronto & $0.438\pm0.034$ & $0.546\pm0.036$ & $0.646\pm0.034$ & $0.644\pm0.033$ & $0.530\pm0.037$ & $0.552\pm0.036$ & $\firstBest{0.660\pm0.033}$ & $\secondBest{0.656\pm0.034}$ \\
\hline
\end{tabular}
\caption{Results on trajectory recommendation datasets on best of top-10.}
\end{table*}


%\clearpage
%\textbf{Performance on short and long trajectories}.
\subsubsection{Performance on short and long trajectories}
The performance of baselines and structured recommendation algorithms 
for short (length $<$ 5) and long (length $\ge$ 5) trajectories 
with top-$k$ ($k=1:10$) prediction are shown in the following figures.

\clearpage
% topk evaluation plots
%\includepdf[pages={1-},scale=0.75]{plots.pdf}
\includepdf[pages={1-}]{plot_topk.pdf}
