\documentclass[letterpaper]{article} % Required US Letter
\usepackage{aaai18}  % Required
\usepackage{times}   % Required
\usepackage{helvet}  % Required
\usepackage{courier} % Required

\setlength{\pdfpagewidth}{8.5in}  % Required
\setlength{\pdfpageheight}{11in}  % Required

%PDF Info Is Required:
\pdfinfo{
/Title (Trajectory Recommendation via Structured Prediction)
%/Author (Authors list)
}
\setcounter{secnumdepth}{0}  
 
%\usepackage[numbers,compress]{natbib}
% if you need to pass options to natbib, use, e.g.:
%\PassOptionsToPackage{numbers, compress}{natbib}

% to compile a camera-ready version, add the [final] option, e.g.:
% \usepackage[final]{nips_2017}

\usepackage[utf8]{inputenc} % allow utf-8 input
\usepackage[T1]{fontenc}    % use 8-bit T1 fonts
%\usepackage[colorlinks=true,citecolor=blue]{hyperref}       % hyperlinks
\usepackage{url}            % simple URL typesetting
\usepackage{booktabs}       % professional-quality tables
\usepackage{amsfonts}       % blackboard math symbols
\usepackage{nicefrac}       % compact symbols for 1/2, etc.
\usepackage{microtype}      % microtypography


%%%%%%%%%%%%%%%%%%%%%%%%%%%%%%%%%
% Custom packages
%%%%%%%%%%%%%%%%%%%%%%%%%%%%%%%%%
% AKM: already present above!
%\usepackage{hyperref}
\usepackage{amsmath}
\usepackage{amsfonts}
\usepackage{mathrsfs}
\usepackage{bm}
\usepackage{bbm}
\usepackage{stmaryrd}
\usepackage{algorithm}
\usepackage{algorithmic}
%\usepackage[sc]{mathpazo}
%\linespread{1.05}         % Palladio needs more leading (space between lines)
\usepackage[T1]{fontenc}    % use 8-bit T1 fonts
\usepackage{nicefrac}       % compact symbols for 1/2, etc.
\usepackage{microtype}      % microtypography
\usepackage{subcaption}     % sub-fig
\usepackage{booktabs} % professional-quality tables
\usepackage{graphicx}
\graphicspath{{fig/}} % Location of the graphics files
\usepackage{pdfpages} % include multi-page PDF file
\usepackage{wrapfig}  % wrap text around figure, table and minipage
\usepackage{footmisc} % provide labels for footnote
\usepackage{colortbl} % cellcolor

\usepackage{enumitem}
\usepackage{capt-of}
\usepackage{varwidth}

%%%%%%%%%%%%%%%%%%%%%%%%%%%%%%%%%
% Custom notation
%%%%%%%%%%%%%%%%%%%%%%%%%%%%%%%%%
\DeclareMathOperator*{\argmin}{argmin}
\DeclareMathOperator*{\argmax}{argmax}
\newcommand{\given}{\mid}
\newcommand{\llb}{\llbracket}
\newcommand{\rrb}{\rrbracket}
\newcommand{\indicator}[1]{\llbracket #1 \rrbracket}
\newcommand{\eat}[1]{}
\newcommand{\Q}{\mathsf{Q}}
\newcommand{\X}{\mathsf{X}}
\newcommand{\Y}{\mathsf{Y}}
\newcommand{\CCal}{\mathscr{C}}
\newcommand{\QCal}{\mathscr{Q}}
\newcommand{\XCal}{\mathscr{X}}
\newcommand{\YCal}{\mathscr{Y}}
\newcommand{\OSf}{O}
\newcommand{\SSf}{\mathsf{S}}
\newcommand{\Real}{\mathbb{R}}
\newcommand{\E}[2]{\underset{#1}{\mathbb{E}}\left[ #2 \right]}
\newcommand{\ES}[2]{\underset{#1}{\mathbb{E}}\, #2}
\newcommand{\defEq}{\stackrel{.}{=}}


% madeness: suPer-script in Brackets
\newcommand{\pb}[1]{^{({#1})}}
%\newcommand{\citet}[1]{\cite{#1}}
%\newcommand{\citep}[1]{\cite{#1}}
\newcommand{\citet}[1]{\citeauthor{#1}~\shortcite{#1}}
\newcommand{\citep}{\cite}
\newcommand{\citealp}[1]{\citeauthor{#1}~\citeyear{#1}}

\newcommand{\bu}{\mathbf{u}}
\newcommand{\bv}{\mathbf{v}}
\newcommand{\bz}{\mathbf{z}}
\newcommand{\x}{\mathbf{x}}
\newcommand{\y}{\mathbf{y}}
\newcommand{\w}{\mathbf{w}}
\newcommand{\xii}{\mathbf{x}^{(i)}}
\newcommand{\yii}{\mathbf{y}^{(i)}}
\newcommand{\yij}{\mathbf{y}^{(ij)}}
\newcommand{\yik}{\mathbf{y}^{(ik)}}

\newcommand{\eg}{e.g.\ }
\newcommand{\ie}{i.e.\ }

\newcommand{\TODO}[1]{ {\color{blue}{\bf TODO:~{#1}}} }
%\newcommand{\rev}[1]{{\color{red}{#1}}}
\newcommand{\rev}[1]{{\color{black}{#1}}}

\newcommand{\trajrec}{trajectory recommendation}
\newcommand{\trajectory}{trajectory}
\newcommand{\seqrec}{sequence recommendation}

% \newcommand{\firstBest}[1]{\cellcolor{green!25}{#1}}
% \newcommand{\secondBest}[1]{\cellcolor{yellow!25}{#1}}

\newcommand{\firstBest}[1]{\cellcolor{gray!62}{#1}}
\newcommand{\secondBest}[1]{\cellcolor{gray!20}{#1}}

% spacing black magic 
%\newcommand{\secmoveup}{\vspace{-3.5mm}}                %{\vspace{-0.12in}}
%\newcommand{\ssecmoveup}{\vspace{-0.mm}}                %{\vspace{-0.12in}}
%\newcommand{\bigsecmoveup}{\secmoveup\vspace{-0mm}}   %{\vspace{-0.08in}}
%\newcommand{\textmoveup}{\vspace{-2.2mm}}               %{\vspace{-0.08in}}
%\newcommand{\bigtextmoveup}{\textmoveup\vspace{-0.0in}} %{\vspace{-0.06in}}
%\newcommand{\itemmoveup}{\vspace{-0.5mm}}                %{\vspace{-0.04in}}
%\newcommand{\eqmoveup}{\vspace{-1.0mm}}                 %{\vspace{-0.16in}}
%\newcommand{\captionmoveup}{\eqmoveup\vspace{-2.80mm}}   %{\vspace{-0.16in}}
%\newcommand{\refitemmoveup}{\vspace{-0.mm}}   			  %{\vspace{-0.16in}}


%\title{Structured Recommendation}
%\title{Travel Sequence Recommendation via Structured Prediction}
\title{Trajectory Recommendation via Structured Prediction}

% The \author macro works with any number of authors. There are two
% commands used to separate the names and addresses of multiple
% authors: \And and \AND.
%
% Using \And between authors leaves it to LaTeX to determine where to
% break the lines. Using \AND forces a line break at that point. So,
% if LaTeX puts 3 of 4 authors names on the first line, and the last
% on the second line, try using \AND instead of \And before the third
% author name.

\allowdisplaybreaks

\begin{document}

\maketitle

\begin{abstract}
% !TEX root=./main.tex

%Playlists are a core feature of music streaming services.
Playlist recommendation concerns producing a sequence of songs that a user might enjoy.
We investigate this problem in three different cold-start scenarios:
%Specifically, we investigate three settings with different cold items:
(i) \emph{cold songs}, where we recommend newly released songs to extend existing playlists;
(ii) \emph{cold playlists}, where we recommend a set of songs to form a new playlist for an existing user; %without additional context except the user;
(iii) \emph{cold users}, where we recommend a set of songs to form a new playlist for a new user. %, without any other context.

We propose a flexible multitask learning method to deal with all three settings.
The method learns from user-curated playlists,
%the %multitask learning
%method
and encourages songs in the playlist 
to be ranked higher than those are not
by minimising a %the Bottom-Push
bipartite ranking loss.
We formulate the objective as a constrained convex optimisation problem,
and show how this may be approximated by an unconstrained objective
%then address the difficulty of a large number of constraints by approximating the %Bottom-Push loss
%bipartite ranking loss
%with a classification loss
inspired by an equivalence relationship between bipartite ranking and binary classification.
Empirical results on two real music playlist datasets show the proposed approach has good performance for playlist recommendation
in cold-start settings.
%in three cold-start settings.

\end{abstract}
%\captionmoveup

%!TEX root = main.tex

\secmoveup
\section{Introduction}
\label{sec:intro}
\textmoveup

Content recommendation has been the subject of a rich body of literature~\citep{Goldberg:1992,Sarwar:2001,Koren:2010},
with established techniques seeing widespread adoption in industry~\citep{Linden:2003,Agarwal:2013,Amatriain:2015,Gomez-Uribe:2015}.
The success of these methods is explained by both the explosion in availability of user's explicit and implicit preferences for content,
as well as the design of methods that can suitably exploit these to make useful recommendations~\citep{Koren:2009}.

For the most part, models for recommendation have been limited to the case of static, \emph{unstructured} content.
While this setting has considerable value by itself,
in many important scenarios one needs to recommend content that possesses some \emph{structure}.
In its simplest form, this structure may be manifest by the items we wish to recommend being \emph{sequential} in nature.
For example, consider the problem of
recommending a trajectory of points of interest in a city to a visitor~\citep{lu2010photo2trip,lu2012personalized,ijcai15,cikm16paper},
or alternately,
recommending a playlist of songs to users based on their listening history~\citep{McFee:2011,chen2012playlist,hidasi2015session,choi2016towards}.

A na\"{i}ve approach to such sequential recommendation problems is to simply ignore the structure,
and learn a standard recommendation model.
% In the playlist example, we could learn a user's preference for individual songs,
% and then create a playlist based on the top ranked songs.
In the trajectory example, we could learn a user's preference for individual locations,
and then create a trajectory based on the top ranked songs.
However, it is easy to construct cases where such an approach is sub-optimal:
%This however completely ignores the fact that while a user's
for example,
in the trajectory recommendation problem, it is unlikely a user will want to visit three restaurants in a row.
Similarly, 
while a user's two favourite songs might belong
the metal and country genres respectively,
it is questionable that a playlist featuring these songs in succession will be as enjoyable to the user.

The above raises the question of how one can effectively learn from such sequential content.
In this paper, we show how to cast sequence recommendation as a \emph{structured prediction} problem,
which allows us to leverage the toolkit of structured SVMs~\citep{tsochantaridis2005large}.
However, a vanilla application of such methods does not suffice,
as they do not account for the fact that each input can have multiple ground truths,
and that \emph{loops} in predicted sequences are undesirable.
We show how to overcome this by
suitably normalising the loss function for the model,
and by modifying the inference and prediction steps using a variant of the Viterbi algorithm.
Specifically, our contributions are as follows:
\begin{itemize}[noitemsep,leftmargin=12pt]\itemmoveup
	\item We formalise the problem of sequence recommendation, cast it as a structured prediction task (\S\ref{sec:recseq}), and show how trajectory recommendation can be seen as a special case (\S\ref{sec:trajrec}).

	\item We show how sequential recommendation may be attacked using structured SVMs (\S\ref{sec:recseq}).
	We propose a correction to the na\"{i}ve implementation of structured SVMs to the recommendation problem, so as to account for the existence of multiple ground truths for each input (\S\ref{ssec:sr}). Following \citep{joachims2009cutting}, we propose both $n$-slack and 1-slack versions of the structured recommender.%\footnote{This new structured recommender can in principle be applied to any problem where loss augmented inference can be efficiently computed. We focus on sequence recommendation in this paper.}
	
	\item We show how one can avoid recommending sequences with loops or repetitions via an extension of the classic Viterbi algorithm that returns a list of the highest scored sequences under some model; we show that this extension may be incorporated during both the training (via loss augmented inference) and prediction steps of our structured recommendation (\S\ref{ssec:subtour}).
	
	\item We present experiments on two real-world trajectory recommendation problems, and demonstrate our structured prediction approaches improve over existing non-structured baselines (\S\ref{sec:experiment}).\itemmoveup
\end{itemize}

%We begin with an overview of related work, before presenting our model.
%We begin with an overview of the sequence recommendation problem, before presenting our model.

%% !TEX root=main.tex

We now formalise the problem of interest and outline its challenges.

%
\subsection{Trajectory recommendation}

Fix some set $\PCal$ of points-of-interest (POIs) in a city.
A \emph{trajectory}\footnote{In graph theory, this is also referred to as a walk.} is any sequence of POIs, possibly containing loops (repeated POIs).
In the \emph{trajectory recommendation} problem, we are given as input a training set of historical tourists' trajectories.
From this, we wish to design a \emph{trajectory recommender}, which accepts a
\emph{trajectory query} $\x = (s, l)$, comprising a start POI $s \in \PCal$, and trip length $l \!>\! 1$, %(\ie the desired number of POIs, including $s$),
and produces one or more sequences of $l$ POIs starting from $s$. %that conform to the query.

Formally, let $\XCal \defEq \PCal \times \{ 2, 3, \ldots \}$ be the set of possible queries,
$\YCal \defEq \bigcup_{l = 2}^\infty \PCal^l$ be the set of all possible trajectories,
and for fixed $\x \in \XCal$, $\YCal_{\x} \subset \YCal$ be the set of trajectories that conform to the constraints imposed by $\x$,
\ie if $\x = (s, l)$ then
%$\YCal_{\x} = \left\{ \y \in \PCal^l \mid y_1 = s \right\}$.
$\YCal_{\x} = \{ \y \in \PCal^l \mid y_1 = s \}$.
Then, the {trajectory recommendation} problem has:

\vspace{0.5\baselineskip}

\begin{mdframed}[innertopmargin=3pt,innerbottommargin=3pt,skipbelow=5pt,roundcorner=8pt,backgroundcolor=red!3,topline=false,rightline=false,leftline=false,bottomline=false]
	\begin{tabular}{ll}
		{\sc Input}:  & training set $\left\{ \left( \x^{(i)}, \y^{(i)} \right) \right\}_{i = 1}^n \in ( \XCal \times \YCal )^n$ \\
		{\sc Output}: & a trajectory recommender $r \colon \XCal \to \YCal$ \\
	\end{tabular}
\end{mdframed}

\vspace{0.5\baselineskip}

One way to design a trajectory recommender is to find a (query, trajectory) affinity function $f \colon \XCal \times \YCal \to \mathbb{R}$, and let
\begin{equation}
	\label{eqn:argmax}
	r( x ) \defEq \argmax_{\y \in \YCal_x}~f(\x, \y).
\end{equation}
%In particular, $\y = (s,~ y_2, \dots, y_l)$ is a trajectory with $l$ POIs. %, which has no sub-tours. %i.e. $y_j \ne y_k$ if $j \ne k$.
%This was the view proposed in~\cite{cikm16paper} where they authors considered an
%objective function that added two components together: a POI score and a transition score.
Several choices of $f$ are possible.
\citet{cikm16paper} proposed to use $f$ given by a RankSVM model. %, combined with a transition score between POIs.
While offering strong performance, this has a conceptual disadvantage highlighted in the previous section:
it does not model global cohesion, and could result in solutions such as recommending three restaurants in a row.

To overcome this, \citet{Chen:2017} proposed to use $f$ given by a structured SVM (SSVM),
wherein $f( \x, \y ) = \mathbf{w}^T \Phi( \x, \y )$ for a suitable feature mapping $\Phi$.
%In the case of an SSVM with pairwise potentials,
When this feature mapping decomposes into terms that depend only on adjacent elements in the sequence $\y$ (akin to a linear-chain conditional random field),
the optimisation in Equation \ref{eqn:argmax} can be solved with the classic Viterbi algorithm.

% !TEX root=main.tex

\tikzstyle{state}=[shape=circle,draw=blue!50,fill=blue!20]
\tikzstyle{state2}=[shape=circle,draw=purple!50,fill=purple!20]
\tikzstyle{hiddenState}=[shape=circle,draw=gray!50,fill=gray!20,dashed]
\tikzstyle{specialState}=[shape=circle,double=red,draw=blue!50,fill=blue!20,dashed]
\tikzstyle{observation}=[shape=rectangle,draw=orange!50,fill=orange!20]
\tikzstyle{hiddenObservation}=[shape=rectangle,draw=gray!50,fill=gray!20,dashed]
\tikzstyle{lightedge}=[<-,thin]
\tikzstyle{mainstate}=[state,ultra thick]
\tikzstyle{mainedge}=[<-,ultra thick]

\begin{figure*}[!htb]
    \centering
    %\resizebox{0.2\textwidth}{!}{
    \subfloat[{\sc LoopElim} (\S\ref{sec:loop-elim}).]{
    \begin{tikzpicture}[baseline=(s0.base)]
        % states
        \node[specialState] (s0) at (0,0) {$1$};
        \node[specialState] (s1) at (1,0) {$2$}
            edge [<-,ultra thick] (s0);
        \node[specialState] (s2) at (2,0) {$3$}
            edge [<-,ultra thick] (s1);
        %\node[state] (s3) at (3,0) {$4$}
        %    edge [<-,ultra thick] (s2);
        
        %\draw [<-,ultra thin,bend right] (s1) to [looseness=1.25] (s3) node[sloped,draw=none] at (2,-0.45) {$/$};
        \draw [<-,ultra thin,bend right] (s0) to [looseness=1.25] (s2) node[sloped,draw=none] at (1,-0.45) {$/$};

        \node[draw=none] at (0,-1.15)  {};
    \end{tikzpicture}
    %}
    }%
    \qquad    
    %\subfloat[Original prediction with loop (dashed).]{
    \subfloat[{\sc Greedy} (\S\ref{sec:greedy}).]{
    \begin{tikzpicture}[baseline=(s12.base)]
        % first-best
        \node[specialState] (s11) at (0,1)  {$1$};
        \node[hiddenState]  (s12) at (0,0)  {$2$};
        \node[hiddenState]  (s13) at (0,-1) {$3$};

        %\node[hiddenState] (s2) at (1,1.5)  {$2$};
        \node[hiddenState]  (s21) at (1,1)  {$1$};
        \node[specialState] (s22) at (1,0)  {$2$};
        \node[hiddenState]  (s23) at (1,-1) {$3$};
        %\node[hiddenState] (s6) at (1,-1.5)  {$6$};

        \node[draw=none] (s32) at (2,1)  {};
        \node[draw=none] (s31) at (2,0)  {};
        \node[draw=none] (s33) at (2,-1) {};

        \node[draw=none] (s411) at (2.25,0)     {$\ldots$};

        %\draw [->,ultra thin] (s1) to (s2);
        \draw [->,ultra thin]      (s11) to (s21) node [draw=none] at (0.5,1) {$/$};
        \draw [->,ultra thick] (s11) to (s22);
        \draw [->,ultra thin]  (s11) to (s23);
        %\draw [->,ultra thin] (s1) to (s6);

        \draw [->,ultra thin] (s22) to (s32) node [draw=none] at (1.5,0.5) {$/$};
        \draw [->,ultra thin] (s22) to (s31) node [draw=none] at (1.5,0)     {$/$};
        \draw [->,ultra thin] (s22) to (s33);
    \end{tikzpicture}
    }%   
    \qquad
    % \subfloat[Modified prediction with loop removed.]{
    % \begin{tikzpicture}[baseline=(s0.base)]
    %     % states
    %     \node[state] (s0) at (-2,2) {$1$};
    %     \node[state] (s1) at (0,2) {$2$}
    %         edge [<-] (s0);
    %     \node[state] (s2) at (2,2) {$3$}
    %         edge [<-] (s1);
    %     \node[state] (s3) at (4,2) {$4$}
    %         edge [<-] (s2);
    %     \draw [color=white,dashed,bend right] (s1) to [looseness=1.25] (s3);            
    % \end{tikzpicture}
    % }
    %\resizebox{0.2\textwidth}{!}{
    \subfloat[{\sc ILP} (\S\ref{sec:ilp}).]{
    \begin{tikzpicture}[baseline=(s1.base)]
        % first-best
        \node[specialState] (s1) at (0,0)  {$1$};
        \node[specialState] (s2) at (1,1)  {$2$};
        \node[hiddenState]  (s3) at (2,1)  {$3$};
        \node[hiddenState]  (s4) at (1,-1) {$4$};
        \node[specialState] (s5) at (2,-1) {$5$};
        \node[specialState] (s6) at (3,0)  {$6$};

        %\node[draw=none] (juka) at (0,-1.5)  {};

        \draw [->,ultra thick] (s1) to (s2);
        \draw [->,ultra thin] (s1) to (s3);
        \draw [->,ultra thin] (s1) to (s4);
        \draw [->,ultra thin] (s1) to (s5);
        \draw [->,ultra thin] (s1) to (s6);    

        \draw [->,ultra thin] (s2) to (s1);
        \draw [->,ultra thin] (s2) to (s3);
        \draw [->,ultra thin] (s2) to (s4);
        \draw [->,ultra thick] (s2) to (s5);
        \draw [->,ultra thin] (s2) to (s6);    

        \draw [->,ultra thin] (s3) to (s1);
        \draw [->,ultra thin] (s3) to (s2);
        \draw [->,ultra thin] (s3) to (s4);
        \draw [->,ultra thin] (s3) to (s5);
        \draw [->,ultra thin] (s3) to (s6);    

        \draw [->,ultra thin] (s4) to (s2);
        \draw [->,ultra thin] (s4) to (s3);
        \draw [->,ultra thin] (s4) to (s1);
        \draw [->,ultra thin] (s4) to (s5);
        \draw [->,ultra thin] (s4) to (s6);    

        \draw [->,ultra thin] (s5) to (s2);
        \draw [->,ultra thin] (s5) to (s3);
        \draw [->,ultra thin] (s5) to (s4);
        \draw [->,ultra thin] (s5) to (s1);
        \draw [->,ultra thick] (s5) to (s6);    

        \draw [->,ultra thin] (s6) to (s2);
        \draw [->,ultra thin] (s6) to (s3);
        \draw [->,ultra thin] (s6) to (s4);
        \draw [->,ultra thin] (s6) to (s5);
        \draw [->,ultra thin] (s6) to (s1);                    
    \end{tikzpicture}
    %}
    }%
    \qquad
    \subfloat[{\sc List Viterbi} (\S\ref{sec:viterbi}).]{
    \begin{tikzpicture}[baseline=(s1.base)]
        % first-best
        \node[specialState] (s1) at (0,0)  {$1$};
        \node[specialState] (s2) at (1,0)  {$2$}            
            edge [<-,ultra thick] (s1);

        \node[specialState] (ss1) at (2,-1) {{${5}$}}
        	edge [<-,ultra thick,decorate,decoration={snake}] (s2);
        \node[specialState] (ss2) at (3,-1) {{${6}$}}
            edge [<-,ultra thick,decorate,decoration={snake}] (ss1);

        \node[hiddenState] (s3) at (2,0)  {$3$}            
            edge [<-,ultra thin] (s2);

        % \node[hiddenState] (s4) at (3,0) {$4$}
        %     edge [<-,ultra thin] (s3);

        %\draw [<-,ultra thin,bend left] (s2) to [looseness=1.25] (s4); 
        \draw [<-,ultra thin,bend left] (s1) to [looseness=1.25] (s3); 
    \end{tikzpicture}
    }
    
    %\caption{Example of heuristically removing loops. The nodes are numbered by the POI, with edges denoting order in the sequence. While the modified prediction removes the loop in the original sequence, it is necessarily at the expense of returning a path with fewer number of POIs.}
    \caption{Schematics of different algorithms to return a loop-free prediction.
    Nodes such as
	{\protect\tikz[baseline=(X.base)]{\protect\node[specialState,inner sep=2pt] (X) {\small$1$}}}
    are selected by the algorithm, with thick edges such as
    {\protect\tikz{\protect\coordinate (X) at (0,0) {}; \protect\coordinate (Y) at (0.5,0) {}; \protect\draw[->,ultra thick] (X) to (Y); \protect\node[draw=none] at (0.25,-0.00625) {}}}
    denoting the sequence ordering.
    {\sc LoopElim} removes the loop from the Viterbi solution (here the POI sequence $(1,2,3,1)$), possibly returning a path of shorter length than requested;    
    {\sc Greedy} incrementally selects POIs which have not been selected before, and locally maximises the sub-path score;
    {\sc ILP} solves an integer linear program to find the optimal length $l$ path in a complete graph over POIs;
    {\sc ListViterbi} finds where the second-best sequence diverges from the standard Viterbi sequence ($(1,2,3,1)$ as before); if not loop-free, it finds where the third-best diverges from the second-best, \emph{etc}.
    }
    \label{fig:schematics}
    \vspace{-0.5\baselineskip}
\end{figure*}



%
\subsection{Path recommendation}

We argue that the definition of trajectory recommendation is incomplete for a simple reason:
in most cases, a tourist will not want to revisit the same POI.
Instead, what is needed is to recommend a \emph{path}, \ie a trajectory that does not have any repeated POIs.
Let $\thickbar{\YCal} \subset \YCal$ be the set of all possible paths,
and for fixed $\x \in \XCal$, let $\thickbar{\YCal}_{\x} \subset \thickbar{\YCal}$ be the set of paths that conform to the constraints imposed by $\x$.
We now wish to construct a \emph{path recommender} $r \colon \XCal \to \thickbar{\YCal}$ via
\begin{equation}
	\label{eqn:argmax-path}
	r( x ) \defEq \argmax_{\y \in \thickbar{\YCal}_x}~f(\x, \y).
\end{equation}
For $f$ given by an SSVM, Equation \ref{eqn:argmax-path} requires we depart from the standard Viterbi algorithm, as the sequence in Equation \ref{eqn:argmax} may well have a loop.\footnote{In SSVMs, this issue also arises during training \citep{Chen:2017}, but we focus here only on the prediction problem. See \S\ref{sec:discussion} for additional comments.}
There are two distinct modes of attack available:
\begin{enumerate}
	\item seek an approximate solution to the problem,
	via heuristics that exploit a graph view of Equation \ref{eqn:argmax-path}.
	%that remove loops present in the standard Viterbi solution,
	%or by greedily constructing a loop-free solution.

	\item seek an exact solution to the problem,
	via integer linear programming,
	or top-$K$ extensions of the Viterbi algorithm. %(known as list Viterbi algorithms).
\end{enumerate}
While \citet{Chen:2017} suggested the latter exact approaches, they did not formally compare their performance either qualitatively or quantitatively;
they did not detail the different top-$K$ extensions of the Viterbi algorithm and the connections thereof;
and they did not consider approximate approaches.

In the sequel, we thus detail the above approaches in more detail.
Figure \ref{fig:schematics} gives a schematic overview of these algorithms.


% {\color{red!75}
% \begin{itemize}
%   %\item connect to workshop
%   %\item distinguish between next location vs whole trajectory
%   %\item define word usage: trajectory, path, walk, sequence, tour, etc.
%   \item describe relation to travelling salesman, and say why different
%   %\item contributions of this paper
% \end{itemize}
% }

%!TEX root = main.tex

%\section{Recommending sequences}
\secmoveup
\section{The sequence recommendation problem}
\label{sec:recseq}
\textmoveup

We introduce the sequence recommendation problem that is the focus of this paper.
%To do so, we first
We then provide some motivating examples, in particular the problem of trajectory recommendation.
%and show how it may be viewed as a kind of structured prediction problem.


%
%\subsection{Trajectory recommendation as a structured prediction problem}
%\section{Trajectory Recommendation as Structured Prediction}
\label{sec:trajrec}

The trajectory recommendation problem is: given a set of points-of-interest (POI) $\mathcal{P}$ and a trajectory query $\mathbf{x} = (s, K)$,
where $s \in \mathcal{P}$ is the desired start POI and $K > 1$ is the number of POIs in the desired trajectory (including the start location $s$).
We want to recommend a sequence of POIs $\mathbf{y}^*$ that maximises utility, i.e., for a suitable function $f(\cdot,\cdot)$,
\begin{equation*}
\mathbf{y}^* = \argmax_{\mathbf{y} \in \mathcal{Y}}~f(\mathbf{x}, \mathbf{y}),
\end{equation*}
where $\mathcal{Y}$ is the set of all possible trajectories with POIs in $\mathcal{P}$ and conform to query $\mathbf{x}$.
$\mathbf{y} = (y_1 = s,~ y_2, \dots, y_K)$ is a trajectory with $K$ POIs, and $y_j \ne y_k$ if $j \ne k$, 
i.e.,
there is no sub-tours in trajectory $\mathbf{y}$.


Instead of specifying the number of desired POIs, we can constrain the trajectory with a total time budget $T$.
In this case, the number of POIs $K$ can be treated as a \emph{hidden} variable, with additional constraint $\sum_{k=1}^K t_k \le T$
where $t_k$ is the time spent at POI $y_k$.


This problem is related to automatic playlist generation,
where we recommend a sequence of songs given a specified song (a.k.a. the seed) and the number of new songs.
Formally, given a library of songs and a query $\mathbf{x} = (s, K)$, where $s$ is the seed and $K$ is the number of songs in playlist,
we produce a list with $K$ songs (without duplication) by maximising the likelihood~\cite{chen2012playlist},
\begin{equation*}
%\max_{(y_1,\dots,y_K)} \prod_{k=2}^K \mathbb{P}(y_{k-1} \given y_k),~ y_1 = s ~\text{and}~ y_j \ne y_k,~ j \ne k.
\mathbf{y}^* = \argmax_{\mathbf{y} \in \mathcal{P}_\mathbf{x}}~ \mathbb{P}(\mathbf{y} \given \mathbf{x}),~ \mathbf{y} = (y_1=s,\dots,y_K)
~\text{and}~ y_j \ne y_k ~\text{if}~ j \ne k.
\end{equation*}

Another similar problem is choosing a small set of photos from a large photo library and compiling them into a slideshow or movie.


% standard SSVM
We can learn a recommender by training a SSVM on the set of observed trajectories $\{\mathbf{x}^{(i')}, \mathbf{y}^{(i')}\}_{i'=1}^{N'}$,
However, we ignore the fact that for the same query, we normally observed more than one trajectory,
we would like to exploit this fact to better modelling the observed trajectories.


\subsection{Query Aggregation}
\label{sec:query}

To modelling the fact that a given query has multiple observed trajectories, 
we firstly group trajectories according to queries, in other words,
we now have a dataset $\{\mathbf{x}^{(i)}, \{\mathbf{y}^{(ij)}\}_{j=1}^{N_i}\}_{i=1}^N$
with $N$ queries and queries $\mathbf{x}^{(i)}$ has $N_i$ trajectories observed.


\subsection{Recommendation with Multiset SSVM}
\label{sec:trajrec-ssvm}

We can learn to recommend trajectories by training a multiset SSVM described in Section~\ref{sec:ssvm-ms}

% multiset SSVM

% multiset SSVM: training, prediction




%
%\subsection{From trajectory to sequence recommendation}
\subsection{Structured and sequence recommendation}
\label{sec:seqrec-defn}

%We now generalise the previous discussion to cover a broad class of problems.
Consider the following abstract
\emph{structured recommendation} problem:
given an input query $\x \in \mathcal{X}$ (representing \eg a user, a location, or some ``seed'' song)
we wish to recommend one or more \emph{structured outputs} $\y \in \mathcal{Y}$ (representing \eg a sequence of locations, or songs)
according to a learned \emph{score function} $f(\x,\y)$.
To learn $f$,
we are provided as input a training set
%$(\x\pb{i}, \{ \y\pb{ij} \}_{j=1:n^i})$, $i=1:n$,
$\{ ( \x\pb{i}, \{ \y\pb{ij} \}_{j=1}^{n_i} ) \}_{i=1}^{n}$,
comprising a collection of inputs $\x\pb{i}$ with an associated \emph{set} of $n_i$ output structures $\{ \y\pb{ij} \}$.

For this work, we assume the output $\y$ is a \emph{sequence} of $l$ points, denoted $y_{1:l}$
where each $y_i$ belongs to some fixed set (e.g.\ places of interest in a city, or songs).
%For example, each $\y$ may be a sequences of places in a city, or a playlist of songs.
%Thus, for example, the training set might represent a collection of users in a city, along with a set of trajectories () they have visited.
We call the resulting specialisation the \emph{sequence recommendation} problem,
and this shall be our primary interest in this paper.
In many settings, one further requires the sequences to be \emph{paths} \ie not contain any repetitions.

As a remark, we note that the assumption that $\y$ is a sequence does not limit the generality of our approach,
as inferring $\y$ of other structure can be achieved using corresponding inference and loss-augmented inference algorithms~\cite{joachims2009predicting}.  %LX - this sentence can be cut or merged above


%
\subsection{Sequence recommendation versus existing problems}

There are key differences between sequence recommendation and %what is being solved in
standard problems in structured prediction and recommender systems;
%This setting generalises from structured prediction and recommendation problems in the following ways.
this brings unique challenges for both inference and learning.

In a structured prediction problem, the goal is to learn from a set of
input vector and output sequence tuples %$(\x\pb{i}, \y\pb{i})$, $i=1:n$.
$\{ (\x\pb{i}, \y\pb{i}) \}_{i = 1}^n$, where
for each input $\x\pb{i}$ there is usually one \emph{unique} output sequence $\y\pb{i}$.
In a sequence recommendation problem, however, we expect that %learn from
%tuples $(\x\pb{i}, \{ y\pb{ij} \}_{j=1:n^i})$, $i=1:n$. That is to say,
for each input $\x\pb{i}$ (\eg users),
there %is %have not one, but a set of
are \emph{multiple} associated outputs %$\{ y\pb{ij} \}_{j=1:n^i}$ (\eg movies).
$\{ \y\pb{ij} \}_{j=1}^{n_i}$ (\eg trajectories they have visited).
%Indeed, the existence of multiple outputs is the basis on which even non-structured recommendation systems are built, as one looks to exploit signal embedded in the aggregate information.
Structured prediction approaches do not have a standard way to handle such multiple output sequences.
%$\{ \y\pb{ij} \}_{j=1:n^i}$
%for each input %$\x\pb{i}$
%yet.

In a typical recommender systems problem, the outputs are non-structured; canonically, one works with {static} content such as books or movies~\citep{Goldberg:1992,Sarwar:2001,Netflix}.
Thus, making a prediction involves enumerating all {\em non-structured} items $y$ in order to compute $\argmax_y f(\x,y)$ for suitable score function $f$ \eg some form of matrix factorisation~\citep{Koren:2009}.
For sequence recommendation, computing $\argmax_\y f(\x,\y)$ is harder since it is often impossible to efficiently enumerate $\y$ (\eg all possible trajectories in a city).
This inability to enumerate $\y$ also poses a challenge in designing a suitable $f(\x,\y)$;
\eg
%the standard matrix factorisation approach to recommender systems~\citep{Koren:2009}
matrix factorisation
would require associating a latent feature with each $\y$, which will be infeasible.


%
\subsection{Examples of sequence recommendation}
\label{sec:trajrec}

To make the sequence recommendation problem more concrete,
we provide three specific examples,
starting with the problem of trajectory recommendation
that shall serve as a recurring motivation.
%we explicate how a recently studied problem may be viewed as a special case.
Note that in all these problems, one is specifically interested in sequences that are paths.

\section{Trajectory Recommendation as Structured Prediction}
\label{sec:trajrec}

The trajectory recommendation problem is: given a set of points-of-interest (POI) $\mathcal{P}$ and a trajectory query $\mathbf{x} = (s, K)$,
where $s \in \mathcal{P}$ is the desired start POI and $K > 1$ is the number of POIs in the desired trajectory (including the start location $s$).
We want to recommend a sequence of POIs $\mathbf{y}^*$ that maximises utility, i.e., for a suitable function $f(\cdot,\cdot)$,
\begin{equation*}
\mathbf{y}^* = \argmax_{\mathbf{y} \in \mathcal{Y}}~f(\mathbf{x}, \mathbf{y}),
\end{equation*}
where $\mathcal{Y}$ is the set of all possible trajectories with POIs in $\mathcal{P}$ and conform to query $\mathbf{x}$.
$\mathbf{y} = (y_1 = s,~ y_2, \dots, y_K)$ is a trajectory with $K$ POIs, and $y_j \ne y_k$ if $j \ne k$, 
i.e.,
there is no sub-tours in trajectory $\mathbf{y}$.


Instead of specifying the number of desired POIs, we can constrain the trajectory with a total time budget $T$.
In this case, the number of POIs $K$ can be treated as a \emph{hidden} variable, with additional constraint $\sum_{k=1}^K t_k \le T$
where $t_k$ is the time spent at POI $y_k$.


This problem is related to automatic playlist generation,
where we recommend a sequence of songs given a specified song (a.k.a. the seed) and the number of new songs.
Formally, given a library of songs and a query $\mathbf{x} = (s, K)$, where $s$ is the seed and $K$ is the number of songs in playlist,
we produce a list with $K$ songs (without duplication) by maximising the likelihood~\cite{chen2012playlist},
\begin{equation*}
%\max_{(y_1,\dots,y_K)} \prod_{k=2}^K \mathbb{P}(y_{k-1} \given y_k),~ y_1 = s ~\text{and}~ y_j \ne y_k,~ j \ne k.
\mathbf{y}^* = \argmax_{\mathbf{y} \in \mathcal{P}_\mathbf{x}}~ \mathbb{P}(\mathbf{y} \given \mathbf{x}),~ \mathbf{y} = (y_1=s,\dots,y_K)
~\text{and}~ y_j \ne y_k ~\text{if}~ j \ne k.
\end{equation*}

Another similar problem is choosing a small set of photos from a large photo library and compiling them into a slideshow or movie.


% standard SSVM
We can learn a recommender by training a SSVM on the set of observed trajectories $\{\mathbf{x}^{(i')}, \mathbf{y}^{(i')}\}_{i'=1}^{N'}$,
However, we ignore the fact that for the same query, we normally observed more than one trajectory,
we would like to exploit this fact to better modelling the observed trajectories.


\subsection{Query Aggregation}
\label{sec:query}

To modelling the fact that a given query has multiple observed trajectories, 
we firstly group trajectories according to queries, in other words,
we now have a dataset $\{\mathbf{x}^{(i)}, \{\mathbf{y}^{(ij)}\}_{j=1}^{N_i}\}_{i=1}^N$
with $N$ queries and queries $\mathbf{x}^{(i)}$ has $N_i$ trajectories observed.


\subsection{Recommendation with Multiset SSVM}
\label{sec:trajrec-ssvm}

We can learn to recommend trajectories by training a multiset SSVM described in Section~\ref{sec:ssvm-ms}

% multiset SSVM

% multiset SSVM: training, prediction




%!TEX root = main.tex
%\section{Recommending sequences}
\secmoveup
\section{The sequence recommendation problem}
\label{sec:recseq}
\textmoveup

We begin with an overview of the sequence recommendation problem, before presenting our model.
Consider first the following abstract
\emph{structured recommendation} problem:
given an input query $\x \in \mathcal{X}$ -- representing for example a user, a location, or some ``seed'' item --
we wish to recommend one or more \emph{structured outputs} $\y \in \mathcal{Y}$ according to a learned \emph{score function} $f(\x,\y)$.
To learn a suitable $f$,
we are provided as input a training set
%$(\x\pb{i}, \{ \y\pb{ij} \}_{j=1:n^i})$, $i=1:n$,
$\{ ( \x\pb{i}, \{ \y\pb{ij} \}_{j=1}^{n_i} ) \}_{i=1}^{n}$,
comprising a collection of inputs $\x\pb{i}$ with an associated \emph{set} of output structures $\{ \y\pb{ij} \}$.
As an example, this might represent a collection of users in a city, along with a set of trajectories (sequences of places) that the user has visited.

For this work, we assume the output $\y$ is a \emph{sequence} of $l$ points, denoted $y_{1:l}$
where each $y_i$ belongs to some fixed set (e.g.\ places of interest in a city).
We call the resulting specialisation the \emph{sequence recommendation} problem,
and this shall be our primary interest in this paper.
The assumption that $\y$ is a sequence does not limit the generality of our approach,
as inferring $\y$ of other structure can be achieved using corresponding inference and loss-augmented inference algorithms~\cite{joachims2009predicting}.  %LX - this sentence can be cut or merged above

There are notable differences between the sequence recommendation problem and %what is being solved in
the standard problems considered in structured prediction and recommender systems.
%This setting generalises from structured prediction and recommendation problems in the following ways.
These differences bring unique challenges for both inference and learning.
In a typical structured prediction setting, the goal is to learn from a collection of $n$
input vector and output sequence tuples %$(\x\pb{i}, \y\pb{i})$, $i=1:n$.
$\{ (\x\pb{i}, \y\pb{i}) \}_{i = 1}^n$. Here,
for each distinct input $\x\pb{i}$ there is usually one \emph{unique} output sequence $\y\pb{i}$.
In a typical sequence recommendation problem, however, we expect that %learn from
%tuples $(\x\pb{i}, \{ y\pb{ij} \}_{j=1:n^i})$, $i=1:n$. That is to say,
for each input $\x\pb{i}$ (\eg users),
there %is %have not one, but a set of
are multiple associated outputs %$\{ y\pb{ij} \}_{j=1:n^i}$ (\eg movies).
$\{ \y\pb{ij} \}_{j=1}^{n_i}$ (\eg trajectories they have visited).
Indeed, the existence of multiple outputs is the basis on which even non-structured recommendation systems are built, as one looks to exploit signal embedded in the aggregate information.
For model learning, structured prediction approaches do not have a standard way to take into account multiple output sequences %$\{ \y\pb{ij} \}_{j=1:n^i}$
for each input %$\x\pb{i}$
yet.

On the other hand, for typical recommender systems problems, one assumes that the outputs are non-structured (\eg real-valued ratings for movies).
Thus, making a prediction involves enumerating all {\em non-structured} items $y$ in order to compute $\argmax_y f(\x,y)$.
For structured recommendation problems, computing $\argmax_\y f(\x,\y)$ is harder since it is often impossible to efficiently enumerate $\y$ (\eg all possible trajectories in a city).

In the rest of this section, we will first review the background of structured prediction problems (Sec~\ref{ssec:ssvm}), then present a model for structured recommendation (Sec~\ref{ssec:sr}), followed by algorithms for its learning (Sec~\ref{ssec:subtour}) and inference (Sec~\ref{ssec:SRinf}).

% In many practical
% problems we may observe more than one label for the same set of features, which violates
% the implicit assumptions of many learning algorithms. In this work we explicitly consider
% all observed labels of a particular example to be useful for training, that is we use
% the multiset of ground truths in training.
% In particular we focus on the structured prediction case,
% where the output of the classifier is from a large set $\mathcal{Y}$ with internal structure.
% An example of this is when $y\in\mathcal{Y}$ is a sequence of binary values.
% Given an example $x_i$ there may be multiple label sequences $y_{ij}$, where $j=1,...,J$.

\eat{
Suggested order:
\begin{enumerate}
  \item structured SVM
  \item multiset SSVM
  \item list Viterbi for multiple ground truths
\end{enumerate}

Then focus on trajectory
\begin{enumerate}
  \item Trajectory recommendation
  \item ILP for subtour elimination
  \item 2 uses of list Viterbi
  \begin{itemize}
    \item multiple ground truths
    \item subtour elimination
  \end{itemize}
\end{enumerate}
}

\secmoveup
\subsection{Preliminaries: structured SVMs}
\label{ssec:ssvm}
\textmoveup

%In structured prediction, the output of classifier given feature vector $\x$ is
%\begin{equation*}
%\y^* = \argmax_{\y \in \mathcal{Y}}~ f(\x, \y),
%\end{equation*}
%where $\mathcal{Y}_\mathbf{x}$ is the set of all possible trajectories with POIs in $\mathcal{P}$ and satisfying query $\mathbf{x}$,
One well known model for structured prediction is the Structured Support Vector Machines (SSVM)~\cite{joachims2009predicting,tsochantaridis2005large}.
This comprises three essential ingredients.

\emph{Score function}. In SSVMs, we specify that the score function $f(\x, \y)$ takes a linear form:
%is a function that scores the compatibility between features $\mathbf{x}$ and a specific label $\mathbf{y}$,
%in the case of structured SVM (SSVM), the compatibility function $f(\mathbf{x}, \mathbf{y})$ for structured SVM is this linear form,
\begin{equation*}
f(\x, \y) = \w^\top \Psi(\x, \y),
\end{equation*}
where $\w$ is a weight vector, and $\Psi(\x, \y)$ is a \emph{joint feature map}
between the input $\x$ and label sequence $\y$.
The design of the feature map $\Psi(\cdot,\cdot)$ is problem specific.
%For many problems, we can assume that it is a vector whose elements represent unary
%terms for each element in the label $y_{1:l}$, and pairwise interactions between the labels.
%For sequence data, in particular, we also assume that the pair-wise interactions are between
%adjacent elements, i.e. $y_j$ and $y_{j+1}$ for $j=1:l-1$.
%Subsequently, the score function $f(\x,\y)$ decomposes into a sum of
%each of these unary and pairwise features with the corresponding feature weight:
%\begin{equation}
%\label{eq:jointfeature}
%\resizebox{0.9\linewidth}{!}{$\displaystyle
%f(\x, \y) =  %\w^\top \Psi(\x,\y)
%\sum_{j=1}^l \w_j^\top \Psi_j(\x, y_j)
%  + \sum_{j=1}^{l-1} \w_{j,j+1}^\top \Psi_{j,j+1}(\x, y_{j}, y_{j+1}). %\nonumber
%  $}
%\end{equation}
%Here, $\Psi_j$ is a feature map between the input and one output label element $y_j$, with a corresponding weight vector $\w_j$
%and $\Psi_{j,j+1}$ is a pairwise feature vector that captures the interactions between consecutive labels $y_j$ and $y_{j+1}$,
%with a corresponding weight vector $\w_{j,j+1}$.

%To learn the parameters, we train the structured SVM by optimising a quadratic program (QP),
\emph{Objective function}.
To learn a suitable set of weights $\w$, SSVMs solve the following optimisation problem:
\begin{equation}
\label{eq:nslack}
%%\resizebox{0.9\linewidth}{!}{$\displaystyle
\begin{aligned}
&\min_{\w, \, \bm{\xi} \ge 0} ~\frac{1}{2} \w^\top \w + \frac{C}{n} \sum_{i=1}^n \xi_i \\
&s.t.~~  \forall i, 
  \w^\top \Psi(\x^{(i)}, \y^{(i)}) - \w^\top \Psi(\x^{(i)}, \bar{\y}) \ge
  \Delta(\y^{(i)}, \bar{\y}) - \xi_i, \, \forall \bar\y \in \mathcal{Y}.
\end{aligned}
%%$}
\end{equation}

Here, 
%$\bar\y$ is an arbitrary candidate sequence,  % \in \mathcal{Y} -- LX: what is cal Y anyway?
$\mathcal{Y}$ is the set of all possible sequences
and $\Delta(\y, \bar\y)$ is the loss function between $\bar\y$ and the ground truth $\y$,
%%LX: we don't need the def of \xi_i below?
and slack variable $\xi_i$ is the \emph{hinge loss} for the prediction of the $i$-th example~\cite{tsochantaridis2005large},
%%\resizebox{1.1\linewidth}{!}{
%%\begin{minipage}{\linewidth}
\begin{align*}
\xi_i = \max \left( 0, \, 
        \max_{\bar{\y} \in \mathcal{Y}}
        \left\{ \Delta(\y^{(i)}, \bar{\y}) + \w^\top \Psi(\x^{(i)}, \bar{\y}) \right\} - \w^\top \Psi(\x^{(i)}, \y^{(i)}) \right).
\end{align*}
%%\end{minipage}
%%}
%This formulation is called "$n$-slack" as we have one slack variable for each example in training set.
Here the inner maximisation is known as the \emph{loss-augmented inference}.
Loss-augmented inference can be efficiently done if loss function $\Delta(\cdot,\cdot)$ is also decomposable
with respect to individual and pairs of label elements.
% in a way similar to Equation~\eqref{eq:jointfeature}.

To solve problem (\ref{eq:nslack}), one option is to simply enumerate all constraints, and feed the problem into a standard solver.
However, this approach is impractical as there is a constraint for every possible label $\bar{\y}$.
Instead, we can resort to a cutting-plane algorithm which repeatedly solves the quadratic program (\ref{eq:nslack})
with a growing set of constraints~\cite{joachims2009predicting}.
In each iteration, a new constraint is formed by solving the loss-augmented inference,
which helps shrink the feasible region of the problem.

\secmoveup
\subsection{SSVM for recommendation: the SR model}
\label{ssec:sr}
\textmoveup

Recall that the structured recommendation problem
involves observing \emph{multiple} ground truth output sequences for each input.
%If we observed more than one labels for a particular set of features,
The classic SSVM described in Section~\ref{ssec:ssvm} can be generalised to capture this setting:
given feature vector $\x^{(i)}$ and the corresponding set of ground truths $\{\y^{(ij)}\}_{j=1}^{n_i}$
where $n_i$ is the number of labels for $\x^{(i)}$,
we can train a \emph{structured recommendation} (\emph{SR}) model by optimising a quadratic program similar to (\ref{eq:nslack}),
\begin{equation}
\label{eq:nslack_ml}
%%\resizebox{0.9\linewidth}{!}{$
\begin{aligned}
&\min_{\w, \, \bm{\xi} \ge 0} ~ \frac{1}{2} \w^\top \w + \frac{C}{N} \sum_{i=1}^n \sum_{j=1}^{n_i} \xi_{ij} \\
&s.t.~~ \forall i, \, \forall j, 
  \w^\top \Psi(\x^{(i)}, \y^{(ij)}) - \w^\top \Psi(\x^{(i)}, \bar{\y}^{(i)}) \ge
  \Delta(\y^{(ij)}, \bar{\y}^{(i)}) - \xi_{ij}.
\end{aligned}
%%$}
\end{equation}
where $N = \sum_i n_i$ and $\bar{\y}^{(i)} \in \mathcal{Y} \setminus \{\y^{(ij)}\}_{j=1}^{n_i}$.
Compared to (\ref{eq:nslack}), the key distinction is that the above
explicitly aggregates all the ground truth labels for each input when generating the constraints,
i.e., the loss-augmented inference becomes
\begin{equation}
\label{eq:loss_aug_inf}
%%\resizebox{0.9\linewidth}{!}{$
\max_{\bar{\y}^{(i)} \in \ \mathcal{Y} \setminus \{\y^{(ij)}\}_{j=1}^{n_i}}
     \left( \Delta(\y^{(ij)}, \bar{\y}^{(i)}) + \w^\top \Psi(\x^{(i)}, \bar\y^{(i)}) \right).
%%$}
\end{equation}
In this way, we do not have apparently contradictory constraints where
two ground truth sequences are each required to have larger score than the other.
Instead of using $N$ slack variables as that in (\ref{eq:nslack_ml}),
we can use one slack variable to represent the sum of the $N$ hinge losses in (\ref{eq:nslack_ml}),
which leads to this $1$-slack formulation,
\begin{equation}
\label{eq:1slack_ml}
%%\resizebox{0.9\linewidth}{!}{$
\begin{aligned}
& \min_{\w, \, \xi \ge 0} ~\frac{1}{2} \w^\top \w + C \xi \\
& s.t.~~ \frac{1}{N} \left( \sum_{i,j} \w^\top \Psi(\x^{(i)}, \y^{(ij)}) - \w^\top \Psi(\x^{(i)}, \bar{\y}^{(i)}) \right) 
  \ge \frac{1}{N} \sum_{i,j} \Delta(\y^{(ij)}, \bar{\y}^{(i)}) - \xi.
\end{aligned}
%%$}
\end{equation}
This $1$-slack formulation can be trained more efficiently than the $N$-slack formulation (\ref{eq:nslack_ml}) and we use it here.

The objective in (\ref{eq:nslack_ml}) is similar to a ranking objective, as the constraints enforce
that the positively labeled sequences (the known items that the user likes) are scored
higher than all other unseen sequences.
Such objectives have proven useful in classic unstructured recommendation~\cite{bpr09}.
%Recent work on positive and unlabelled data have
%theoretically shown the close relationship between positive and unlabelled learning and two class classification.



\subsection{Sequence decoding for the SR model}
\label{ssec:subtour}

The SR model require two algorithmic components for inference and learning
(missing from the SSVM algorithms).
For inference, SR needs to predict a {\em path}, \ie a sequence whose elements are distinct from each other.
As described in Section~\ref{sec:intro}, this is desirable since users traversing a sequence (of locations or music)
would not want to see repeated entries.
Given desired sequence length $l$ among $m$ possible points, the Viterbi algorithm~\cite{tsochantaridis2005large}
will generate the length-$l$ sequence with the best score, i.e. $\y^* = \argmax_\y f(\x,\y)$.
One may then use the well-known
Christofides algorithm~\cite{christofides1976} on $\y^*$ to eliminate loops in the sequence.
%is known to generate a solutions within a factor of 3/2 of the optimal solution for traveling salesman problems.
However, the resulting sequence will have less than the desired number of points, and the resulting score will not be optimal in general.

For learning an SR model, loss-augmented inference (\ref{eq:loss_aug_inf}) needs to be done by excluding multiple known sequences.
As described in Section~\ref{ssec:sr}, %this involves %we would want to maximize the loss-augmented objective function
%$\max_{\bar\y} \left\{ \w^\top \Psi(\x^{(i)}, \bar\y) + \Delta(\y^{(ij)}, \bar{\y}) \right\}$
%where the domain of candidate sequences excludes the known sequences for query $\x^{(i)}$, \ie $\bar{\y} \in \mathcal{Y} \setminus \{\y^{(ij)}\}_{j=1}^{n_i}$.
however, the Viterbi algorithm uses back-tracking to find the best sequence,
and cannot easily exclude known sequences.
The rest of this section describes two algorithms, each intuitively aimed to address one of the two requirements above.
Both can be applied in novel contexts of the SR model.
We will also describe practical choices about which algorithm to use when.


\subsubsection{Finding paths with integer programming}
% ILP for subtour elimination
Inference in the SR model requires finding the best path that traverses exactly $l$ of $m$ candidate points.
This can be seen as a variant of the travelling salesman problem (TSP), or the best of ${m \choose l}$ TSPs.
Such a point traversal problem can be solved by incorporating
sub-tour elimination constraints of the TSP.
In particular, the following integer linear program (ILP) formulation~\cite{ijcai15,cikm16paper}
will solve SR inference for trajectory $\y$ of length $l$.

Consider $u_{jk}$ binary decision variables that are true if and only if
the transition from $y_j$ to $y_k$ is in the resulting trajectory,
and
binary decision variables
$z_j$ that are true iff $y_j$ is the last POI in trajectory.
%Suppose that $l$ is the number of candidate POIs.
For brevity, we index the POIs such that $y_1 = 1$.
Firstly, the desired trajectory should start from $y_1$ (Constraint~\ref{eq:cons2}).
In addition, any POI could be visited at most once (Constraint~\ref{eq:cons3}).
Moreover, only $l-1$ transitions between POIs are permitted
since the trajectory length is $l$ (Constraint~\ref{eq:cons4}).
The last constraint, where $v_j$ is an auxiliary variable,
enforces that only a single sequence of POIs without sub-tours is permitted in the trajectory.
%\TODO{LX: i do not understand the last contraint, $v_j$ did not seem to have been defined? are they binary or something else?}
% DW: v_j defined in the first constraint, which is a natural number

%%\resizebox{.99\columnwidth}{!}{
%%  \begin{minipage}{\linewidth}
\begin{alignat}{5}
\max_{\bu,\bv} ~& \sum_{k=1}^m \w_k^\top \Psi_k(\x, y_k) \sum_{j=1}^m u_{jk}
+ \sum_{j,k=1}^m u_{jk} \w_{jk}^\top \Psi_{j, k}(\x, y_j, y_k) \nonumber\\
  %              & + \sum_{j=1}^l \sum_{k=1}^l u_{jk} \w_{jk}^\top \Psi_{j, k}(\x, y_j, y_k) \\
s.t. ~& u_{jk}, ~z_j \in \{0, 1\}, ~u_{jj}=0, ~z_1=0, ~v_j \in \mathbf{Z},~                          \label{eq:cons1} \\
  & y_j \in \{1,\dots,m\}, ~\forall j, k = 1,\cdots,m                                                \nonumber \\
  & \sum_{k=2}^m u_{1k} = 1, ~\sum_{j=2}^m u_{j1} = 0  \label{eq:cons2} \\
  & \sum_{j=1}^m u_{ji} = z_i + \sum_{k=2}^m u_{ik} \le 1,  ~\forall i=2,\cdots,m                    \label{eq:cons3} \\
  & \sum_{j=1}^m \sum_{k=1}^m u_{jk} = l-1,                                                          \label{eq:cons4} \\
  & v_j - v_k + 1 \le (m-1) (1-u_{jk}),                     ~\forall j,k=2,\cdots,m                  \nonumber
\end{alignat}
%%\end{minipage}
%%}

TSPs are NP-hard,
and so the ILP solver may not find a solution in reasonable time;
however, when the ILP solver finds a solution,
it will be optimal.
%and so the ILP formulation will give the optimal solution when the solver finds the optimal.
We note, however, that an ILP cannot be readily used for loss-augmented inference
for the Hamming loss, the most common loss function for sequence prediction tasks.
This is because ILP requires the loss to be a linear function of $u_{jk}$,
\eg, the number of mis-predicted POIs disregarding the order $\Delta(\y, \bar\y) = 1 - \sum_{j=1}^m \sum_{k=1}^m u_{j, y_k}$,
while Hamming loss cannot be expressed as a linear function of $u_{jk}$.
%if we define the loss as the number of mispredicted POIs,
%where $\mathbf{y}$ is the ground truth and $\bar{\mathbf{y}}$ is the trajectory corresponding to the optimal solution of this ILP.
%=======
%If we employ the above ILP to do loss-augmented inference, one restriction is that the loss should be a linear function of $u_{jk}$,
%e.g., $\Delta(\y, \bar{\y}) = 1 - \sum_{j=1}^M \sum_{k=1}^M u_{j, y_k}$ if we define the loss as the number of mispredicted POIs,
%where $\y$ is the ground truth and $\bar{\y}$ is the trajectory corresponding to the optimal solution of this ILP.

\subsubsection{Finding $k$-best sequences}
% 2 uses of list Viterbi: 1) multiple ground truths; 2) subtour elimination

The algorithm that seems closest to loss-augmented inference with exclusions are several extensions
of the Viterbi algorithm that aim to find more than one high-scoring sequences.
The \emph{parallel list Viterbi} algorithm~\cite{seshadri1994list} finds the top $k$ sequences
by keeping $k$ backtrack pointers for each possible state in each position of the sequence.
This algorithm is computationally efficient
-- a factor of $\log L$ more than the standard Viterbi for score sorting in each step, rather than simple maximization
-- but it keeps many unnecessary pointers. It is also difficult to apply to the SR inference scenario,
since we do not know $k$ beforehand. This is because it is generally impossible to foresee
the rank (according to the score) of the first path among all valid sequences (that may have repeats).

We resort to the \emph{serial list Viterbi} algorithm~\cite{nilsson2001sequentially}.
These algorithms sequentially find the $k$-th best sequence given the best, $2$nd best, \dots, $(k-1)$-th best sequences.
The key insight here is that the 2nd best sequence deviates from the best sequence
for one continuous segment, and then finally merges back to the best sequence without deviating again
-- otherwise replacing one of the continuous segments with those from the best sequence will increase the score.
Subsequently, the $k$th best sequence can be the 2nd best sequence relative to the $k-1$th sequence
at the point of final merge, or the 2nd or 3rd best sequence to the $k-1$th sequence at the point of final merge, \ldots, and so on.
%The version of serial list Viterbi presented by~\cite{nilsson2001sequentially} accommodates
The serial list Viterbi allows exclusion of sequences with repeats, by checking whether or not the current $k$th best sequence has a repeat, if it does, discard and proceed to the $k+1$th. It also allows excluding an arbitrary number of known sequences, by additionally checking, when we get each of the $k$th best sequence, whether or not it is in the exclusion set.
%by partitioning the remaining space
%from each of the known sequence according to their points of deviation,
%finding the best trajectory within each partition and then identifying the best among different partitions.
Due to space limitations, the list Viterbi algorithm is fully detailed in the supplement.

For the structured recommendation problem, the serial list Viterbi algorithm can be used for inference
by sequentially getting the next best sequence, until one or more paths are found.
This algorithm can also be used for loss-augmented inference with Hamming loss,
since it only requires the loss function be decomposable with respect to the label elements.
%This list Viterbi algorithm performs a backward search for trajectories that merges into the existing candidates at any given point.
%as described in Algorithm~\ref{alg:listviterbi}.

%\TODO{Briefly discuss which one to use, when}



\secmoveup
\subsection{Inference and learning for SR}
\label{ssec:SRinf}
\textmoveup

For inference in SR model, both ILP and list Viterbi algorithms can be used, and they will generate the
same result if terminated and converged within a given time budget.
List Viterbi algorithms are polynomial time given the list depth $k$~\cite{nilsson2001sequentially},
but in reality $k$ is unknown a priori and can be very large for long sequences.
We found that state-of-the-art ILP solvers converge to a solution faster if the sequence length $l$ is large.
In the experiments, we use list Viterbi for short sequences, and ILP for long ($l>10$) sequences.

%\TODO{revise this para to describe listViterbi for lost-augmented inf, without loops, and with multiple examples.}
%\TODO{DW: describe top-k prediction using ILP and list Viterbi}

\eat{
Similar to the loss-augmented inference described in Section~\ref{ssec:ssvm},
we can rewrite the constraints in problem (\ref{eq:nslack_ml}) into
\begin{equation*}
%%\resizebox{0.99\linewidth}{!}
%%{$
\w^\top \Psi(\x^{(i)}, \y^{(ij)}) + \xi_{ij} \ge
\max_{\bar{\y}} \left( \Delta(\y^{(ij)}, \bar{\y}) + \w^\top \Psi(\x^{(i)}, \bar{\y}) \right),
%%$}
\eqmoveup
\end{equation*}
Normally, the maximisation at the right side of the above inequality cannot be solved efficiently due to the constraint that
$\bar{\y}$ is in $\mathcal{Y}$,
but should not be in the set of observed labels $\{\y^{(ij)}\}_{j=1}^{n_i}$.
However, we can pretend that $\bar{\y}$ can be any label in $\mathcal{Y}$ and do the unconstrained optimisation,
which can sometimes be solved efficiently, then we filter out the optimal solution if it has been observed,
i.e., in $\{\y^{(ij)}\}_{j=1}^{n_i}$.
In addition to train multiset SSVM, this technique can be further used to deal with other constraints such as sub-tour elimination
%as described in Section~\ref{ssec:subtour}.
as described in Section~\ref{ssec:subtour}.
}
%\TODO{talk about multiple truths and what it means in prediction}

\clearpage
\newpage

\section{Experiment}
\label{sec:experiment}

We first evaluate the proposed method on the task of tag recommendation from text data,
which was formulated as multi-label classification problem~\cite{katakis2008multilabel}.

\subsection{Tag recommendation as multi-label classification}

We experiment on two dataset, \texttt{bibtex} and \texttt{bookmarks}~\cite{katakis2008multilabel}.
The performance are evaluated on classification metrics, \ie F$_1$ scores averaged over either examples or labels 
(which are also known as instance-F$_1$ and macro-F$_1$, respectively),
as well as ranking metric, \ie R-Precision (averaged over either examples or labels).

\paragraph{Baselines}
We compare our method with four baselines.
\begin{itemize}
\item Logistic regression: independently learn a logistic regression classifier for each label, (a.k.a binary relevance).
\item PRLR~\cite{lin2014multi}: a multi-label classifier with a regulariser which encourages sparse and low-rank predictions.
\item SPEN~\cite{belanger2016structured}: a structured prediction framework which employs a deep network to represent the energy function,
      and predictions are produced by minimising the energy.
\item DVN~\cite{gygli2017deep}: a structured prediction method which uses a deep value network to distill the knowledge of a given loss function,
      which is the F$_1$ score (averaged over examples) in this task.
\end{itemize}

We implemented Logistic regression using scikit-learn~\cite{}.
The results of SPEN and DVN are reproduced using the coded released the authors,
and the results of PRLR are taken from \cite{lin2014multi}.

%$\RCal_\textsc{example}$ 
For the proposed method, we used a linear score function $f(\x) = \w_k^\top \x + b$ for the $k$-th label,
and the empirical risk was minimised with L2 regularisation using LBFGS provided by Scipy~\cite{},
and hyper-parameters were tuned using 5-fold cross validation.

\paragraph{Result analysis}
The results on test set are summarised in Table~\ref{tab:perf_mlc},
\begin{itemize}
\item $\RCal_\textsc{example}$ outperform the independent logistic regression baseline by a large margin.
\item $\RCal_\textsc{example}$ also achieves better performance than PRLR~\cite{lin2014multi} which regularisation specific to multi-label classification.
\item Finally, it is encouraging that our method performs better than (\cite{belanger2016structured}) and (\cite{gygli2017deep}),
both work learn complex non-linear functions using deep neural networks to achieve state-of-the-art performance, while our method uses a linear function.
\end{itemize}

\TODO
\begin{itemize}
\item evaluation metric: add R-Precision averaged over both examples and labels, remove AUC.
\item results of more variants: $\RCal_\textsc{label}$ and $\RCal_\textsc{both}$?
\end{itemize}


\begin{table}[!h]
\centering
\caption{Performance on multi-label dataset}
\label{tab:perf_mlc}
%\resizebox{\linewidth}{!}{
\setlength{\tabcolsep}{2pt} % tweak the space between columns
%\begin{tabular}{l*{6}{c}}
\begin{tabular}{l|ccc|ccc}
\toprule
{} & \multicolumn{3}{c|}{\textbf{bibtex}} & \multicolumn{3}{c}{\textbf{bookmarks}} \\
{} &   F$_1$ Example & F$_1$ Label &    AUC &      F$_1$ Example & F$_1$ Label &    AUC \\
\midrule
Binary Relevance~\cite{}           &          $37.9$ &      $30.1$ & $85.3$ &             $29.5$ &      $21.0$ & $87.2$ \\
PRLR~\cite{lin2014multi}           &          $44.2$ &      $37.2$ &    N/A &             $34.9$ &      $23.0$ &    N/A \\
SPEN~\cite{belanger2016structured} &          $41.3$ &      $33.7$ & $92.6$ &             $35.5$ &      $24.1$ & $90.8$ \\
DVN~\cite{gygli2017deep}           &          $44.7$ &      $32.4$ & $86.7$ &             $37.2$ &      $23.7$ & $76.9$ \\
MLR (Ours)                         &          ${\bf 47.0}$ & ${\bf 38.8}$ & ${\bf 93.3}$ & ${\bf 37.7}$ & ${\bf 28.4}$ & ${\bf 91.8}$ \\
\bottomrule
\end{tabular}
%}
\end{table}



\subsection{New song recommendation}
\label{ssec:newsongrec}

\paragraph{Task:} 
We are interested in the task of recommending newly released songs to users,
in particular, to augment users' existing playlists with these songs,
which is a cold-start problem.

\paragraph{Problem formulation:}
We formulate the task of recommending newly released songs to augment existing playlists
as a multi-label classification problem, where we predict, for each song, 
whether it will be included in a given playlist.
This formulation is illustrated in Figure~\ref{fig:mlr},
where rows represent songs (from top to bottom, sorted by the release date in ascending order)
and columns represent playlists (no specific order).
Further, rows with white colour represent songs in training set, and rows with grey colour represent songs in test set.
If entry $(i, j)$ is \texttt{1} (or \texttt{0}), it means the $i$-th song is (or not) found in the $j$-th playlist,
otherwise, we do not know whether the $i$-th song is found in the $j$-th playlist (\ie entry $(i, j)$ is a question mark \texttt{?}).


\TODO
{\it Formulas for each variant.
The objective is the same as $\RCal_\textsc{row}$ and $\RCal_\textsc{col}$ except that,
for the playlists that we choose to hold the later half, all we observed is the first half, 
all other songs for these playlists are unobserved (they can be positive/negative examples),
this is different from the case that we explicitly observed that songs are not in playlists (they are negative examples).
}


\input{fig_mlr}

\paragraph{Dataset:}
We make use of publicly available playlist dataset: the AotM-2011~\cite{mcfee2012hypergraph} and 30Music~\cite{30music2015} playlist dataset. \\
%
{\bf AotM-2011 Dataset} is a collection of playlists shared by users\footnote{\url{http://www.artofthemix.org}} ranging from 1998 to 2011, 
songs in the dataset had been matched to those in the Million Song Dataset (MSD)~\cite{msd2011}.
We filtered out playlists with less than 5 songs, which results in roughly 84K playlists over 114K songs from 14K users. \\
%
{\bf 30Music Dataset} is a collection of listening events and playlists retrieved from Last.fm\footnote{\url{https://www.last.fm}}.
We utilise the playlists data by first intersecting with the MSD, leveraging the Last.fm dataset~\cite{lastfmdataset} 
which matched songs from Last.fm with those in MSD, then filtering out playlists with less than 5 songs, 
which results in roughly 17K playlists over 45K songs from 8K users.

We make use of the audio features of songs provided by MSD, 
and genre data from the Top-MAGD genre dataset~\cite{schindler2012facilitating} and tagtraum genre annotations for MSD~\cite{schreiber2015improving},
which results in 202 audio features and 15 one-hot encoding for genres.

%% details for compute song features.
%% - temporal audio features: use 5-number (percentiles) summary: min, max, median, Q1 and Q3.
%% - missing genre: imputed using the mean values of the genre

Table~\ref{tab:stats_newsongrec} summarises the statistics of the two dataset used for this task.

\begin{table}[!h]
\centering
\caption{Statistics of dataset for new song recommendation}
\label{tab:stats_newsongrec}
%\resizebox{\linewidth}{!}{
\begin{tabular}{ccccccc}
\toprule
Dataset & \#Users & \#Songs (train/dev/test) & \#Playlists & \#Song Features \\
\midrule
AotM-2011 & 14,182  & 68,657 / 22,885 / 22,886 & 84,710 & 217 \\
30Music   & 8,070   & 27,281 / 9,093 / 9,094   & 17,457 & 217 \\
\bottomrule
\end{tabular}
%}
\end{table}


\paragraph{Experimental design:}
In each dataset, we hold (a random) half of the 40\% latest released songs for test,
and other half as validation set, the remaining 60\% of songs are used for training.
All playlists in the dataset are used for this task.


\subsubsection{A few conclusions}

\paragraph{Which type of loss is most helpful?}
\begin{table}[!h]
\centering
\caption{Empirical results (AUC)}
%\resizebox{\linewidth}{!}{
\begin{tabular}{l|cccc}
\toprule
{}            & $\RCal_\textsc{example}$ & $\RCal_\textsc{label}$ & $\RCal_\textsc{both}$ & Independent L.R. \\
\midrule
AotM-2011     & 0.64792  & 0.67782 & 0.59602  & 0.62226 \\
30Music       & 0.6768   & 0.70917 & 0.70914  & 0.6654 \\
%30Music       & 0.54413  & 0.55894 & 0.55864  & 0.53698 \\
\bottomrule
\end{tabular}
%}
\end{table}

\paragraph{Experimental design:}
C: 1, 1, 1, p: 1, no multi-task regularisation.

\paragraph{Is multi-task regularisation helpful?}

\begin{table}[!h]
\centering
\caption{Empirical results}
%\resizebox{\linewidth}{!}{
\begin{tabular}{l|ccc}
\toprule
{}            & Multi-task Reg. + $\RCal_\textsc{example}$ & Multi-task reg. + $\RCal_\textsc{label}$ \\
\midrule
AotM-2011     & 0.65778     & 0.6884618 \\
30Music       & 0.68179     & 0.7149 \\
%30Music       & 0.549567    & 0.557156 \\
\bottomrule
\end{tabular}
%}
\end{table}



\subsection{Playlist augmentation}
\label{ssec:pla}

\paragraph{Task:}
We are interested in augmenting user created playlists with songs from a music library,
in particular, for a partial playlist, we would like to add more songs from an existing collection of songs.
The difference between this task from the task in Section~\ref{ssec:newsongrec} is that,
it is possible to choose any songs from the entire collection of songs, which is not a cold-start problem,
while the task described in Section~\ref{ssec:newsongrec} restricts that 
we choose songs from a subset of the entire collection (the newly released songs), 
which is a cold-start problem.

\paragraph{Problem formulation:}
We formulate the task of augmenting existing playlist as a multi-label classification problem,
that is, for each song that is not in the given playlist, 
we predict whether it will be added to augment the given playlist.
This formulation is illustrated in Figure~\ref{fig:pla},
where rows represent songs (no specific order) and columns represent playlists (no specific order).
Further, columns with white colour represent playlists in training set, 
and columns with grey colour represent playlists that should be augmented (\ie test set).
Similar to the formulation in Section~\ref{ssec:newsongrec}, if entry $(i, j)$ is \texttt{1} (or \texttt{0}), 
it means the $i$-th song is (or not) found in the $j$-th playlist, 
and a question mark \texttt{?} means that we do not know whether the $i$-th song is found in the $j$-th playlist.
As a remark, columns represent playlists in test set contain only \texttt{1} and \texttt{?} entries.

\begin{figure}[!h]
\centering
\setlength{\tabcolsep}{1pt} % tweak the space between columns
\begin{tabular}{|*{7}{c}|ccccc|} \hline
%\rule{.3em}{0pt} 
\rule{0em}{10pt}
& \texttt{0} & \texttt{1} & \texttt{0} & $\cdots$ & \texttt{0} & & & \texttt{?} & $\cdots$ & \texttt{?} & \\
& \texttt{0} & \texttt{0} & \texttt{0} & $\cdots$ & \texttt{0} & & & \texttt{1} & $\cdots$ & \texttt{?} & \\
& \texttt{1} & \texttt{0} & \texttt{0} & $\cdots$ & \texttt{0} & & & \texttt{?} & $\cdots$ & \texttt{?} & \\
\vspace{-5pt}
& \texttt{0} & \texttt{0} & \texttt{0} & $\cdots$ & \texttt{1} & & & \texttt{?} & $\cdots$ & \texttt{1} & \\
& $\vdots$ & $\vdots$ & $\vdots$ & $\vdots$ & $\vdots$ & & & $\vdots$ & $\vdots$ & $\vdots$ & \\
& \texttt{0} & \texttt{0} & \texttt{1} & $\cdots$ & \texttt{0} & & & \texttt{?} & $\cdots$ & \texttt{?} & \\ \hline
\end{tabular}
\caption{Illustration of augmenting playlists as multi-label classification.}
\label{fig:pla}
\end{figure}


\paragraph{Dataset:}
We again use the AotM-2011~\cite{mcfee2012hypergraph} and 30Music~\cite{30music2015} playlist dataset,
the pre-process of the two dataset is the same as that in Section~\ref{ssec:newsongrec}.
Besides making use of the audio features and genres of songs,
we also use the popularity of a given song as a feature, which is defined as the number of occurrence in all playlists,
including the partial playlists in test set.

Table~\ref{tab:stats_pla} summarises the statistics of the two dataset used for this task.

\begin{table}[!h]
\centering
\caption{Statistics of dataset for playlist augmentation}
\label{tab:stats_pla}
%\resizebox{\linewidth}{!}{
\begin{tabular}{ccccccc}
\toprule
Dataset & \#Users & \#Songs & \#Playlists (train/dev/test)  & \#Song Features \\
\midrule
AotM-2011 & 14,182 & 114,428 & 60,260 / 12,225 / 12,225 & 218 \\
30Music   & 8,070  & 45,468  & 15,591 / 933 / 933       & 218 \\
\bottomrule
\end{tabular}
%}
\end{table}

\paragraph{Experimental design:}
In each dataset, we create the test set such that it contains 20\% of each user's playlists (chosen uniformly at random),
if the user has 5 or more playlists, the validation set is constructed the same as the test test.
All remaining playlists are used for training.
As a remark, we observed all songs during training.


\subsubsection{A few conclusions}

\paragraph{Which type of loss is most helpful?}

{\bf Row-wise loss}: weighting by the number of positive/negative labels for each example.
\ie we perform a classification/bipartite ranking task on each multilabel example 
which forms a dataset of examples with binary labels: $\{(\x_n, y_k\}_{k=1}^K$ for the $n$-th multilabel example.

\begin{equation*}
%\resizebox{\linewidth}{!}{$
\RCal_\textsc{row} 
= \displaystyle \sum_s 
  \frac{1}{K_+^s} \sum_{s \in pl} e^{-(\w_{pl}^\top \phi(s) + b)} +
  \frac{1}{K_-^s} \sum_{s \notin pl} \frac{1}{p} e^{p \w_{pl}^\top \phi(s)}.
%$}
\end{equation*}
where normalising factor $K_+^s$ is the number of playlists that include song $s$,
and $K_-^s$ is the number of playlists that do not include song $s$.


{\bf Column-wise loss}: weighting by the number of positive/negative examples for each label.
\ie we perform a classification/bipartite ranking task on each label which forms a binary dataset:
$\{\x_n, y_k\}_{n=1}^N$ for the $k$-th label.

\begin{equation*}
%\resizebox{\linewidth}{!}{$
\RCal_\textsc{col} 
= \displaystyle \sum_{pl}
  \frac{1}{N_+^{pl}} \sum_{s \in pl} e^{-(\w_{pl}^\top \phi(s) + b)} +
  \frac{1}{N_-^{pl}} \sum_{s \notin pl} \frac{1}{p} e^{p \w_{pl}^\top \phi(s)}.
%$}
\end{equation*}
where normalising factor $N_+^{pl}$ is the number of songs in playlist $pl$,
and $N_-^{pl}$ is the number of songs in a music library that playlist $pl$ does not include.


{\bf Row-wise + column-wise loss}: the summation of both: $\RCal_\textsc{row} + C \RCal_\textsc{col}$ 
where $C$ is a trade-off parameter.

The binary relevance baseline is learning a logistic regression for each playlist independently.


\begin{table}[!h]
\centering
\caption{Empirical results}
%\resizebox{\linewidth}{!}{
\begin{tabular}{l|ccccc}
\toprule
{}            & $\RCal_\textsc{example}$ & $\RCal_\textsc{label}$ & $\RCal_\textsc{both}$ & Independent L.R. & Pop-rank \\
\midrule
%AotM-2011     & 0.6827396 & 0.743770 & 0.7385298 & 0.6924 & 0.80199 \\
AotM-2011     & 0.68459 & 0.747755 & 0.7429  & 0.6924 & 0.80199 \\
30Music       & 0.7168  & 0.76867  & 0.76917 & 0.7225 & 0.7165 \\
%30Music       & 0.56766179 & 0.6350941 & 0.63555 & 0.575567 & 0.80558 \\
\bottomrule
\end{tabular}
%}
\end{table}

\paragraph{Experimental design:}
C: 1, 1, 1, p: 1, no multi-task regularisation

\paragraph{Is multi-task regularisation helpful?}

multi-task regularisation: we regularise the difference of playlist parameters 
such that $\|\w_j - \w_k\|_2$ is small if playlist $j$ and $k$ belong to the same user.

\begin{equation*}
\RCal_\textsc{reg} = \frac{1}{\sum_u N_u (N_u - 1)} \sum_u \sum_{j, k \in u} (\w_j - \w_k)^\top (\w_j - \w_k)
\end{equation*}
where $N_u$ is the number of playlist user $u$ has.

\begin{table}[!h]
\centering
\caption{Empirical results}
%\resizebox{\linewidth}{!}{
\begin{tabular}{l|ccc}
\toprule
{}            & Multi-task Reg. + $\RCal_\textsc{example}$ & Multi-task reg. + $\RCal_\textsc{label}$ \\
\midrule
%AotM-2011     & 0.6882099 & 0.7615590 \\
AotM-2011     & 0.69167 & 0.7819 \\
30Music       & 0.7177  & 0.7840 \\
%30Music       & 0.581426 & 0.6597  \\
%30Music       & 0.574175 & 0.667804  \\
%AUC           & 0.66583  & 0.68517   \\
\bottomrule
\end{tabular}
%}
\end{table}



\TODO
measure performance by AUC and HitRate@K,
compare with baselines such as independent logistic regression (\ie binary relevance), popularity based recommendation,
and matrix factorisation.



\subsection{Discussion}

{\it the choice of playlist dataset?}

%\section{Related work}
%\label{sec:related}

%\cheng{Make this a subsection of the multitask learning section.}

We first summarise recent work of recommending music to form playlists
according the problem settings, the recommendation methods,
as well as the information being utilised,
and then music recommendation techniques in cold-start scenarios.

%\cheng{Reduce related work subsection to 1 column}

{\bf Problem settings}.
There are three typical settings:
playlist generation, next song recommendation, and playlist continuation.
%
There is rich collection of recent literature on the playlist generation or prediction
problem~\cite{platt2002learning,mcfee2011natural,mcfee2012hypergraph,chen2012playlist,ben2017groove}.
%
Typical work is to generate a complete playlist given some seed,
for example, the AutoDJ system~\cite{platt2002learning} generates playlists given one or more seed songs,
a playlist for a specific user can be generated by Groove Radio given a seed artist~\cite{ben2017groove},
or a seed location in hidden space, where all songs are embedded,
should be specified in order to generate a complete playlist~\cite{chen2012playlist}.
%
There are also work that focus on evaluating the learned playlist model,
without concretely generate playlist~\cite{mcfee2011natural,mcfee2012hypergraph}.
More details of playlist generation can be found in this recent survey~\cite{bonnin2015automated}.


Next song recommendation~\cite{hariri2012context,bonnin2013evaluating,jannach2015beyond}
is to predict the next song a user might play after observing some context,
for example, the most recent sequence of songs a user interacted with the system was used to
infer the contextual information, which was further used to rank the next possible song
with regards to a topic-based sequential patterns learned from user playlists~\cite{hariri2012context}.
%
The artists appeared in user's listening history can be used as context,
which, together with the popularity of song or frequency of artists collocations,
were further used to score the next song~\cite{mcfee2012million,bonnin2013evaluating}.
%
It is obvious that next song recommendation techniques can also be used to generate a
complete playlist by picking the next song sequentially~\cite{bonnin2013evaluating,ben2017groove}.


Playlist continuation is to add one or more songs to a playlist,
given the added songs serve the same purpose of the original playlist~\cite{schedl2017,recsysch2018}.
One can immediately notice that playlist generation and next song recommendation are special cases
of playlist continuation, which means techniques developed for playlist generation and
next song recommendation can both be used for playlist continuation.

In the collaborative filtering literature,
the cold-start setting has primarily been addressed through
suitable regularisation of matrix factorisation parameters
based on exogenous user- or item-features~\cite{Ma:2008,Agarwal:2009,Cao:2010}.
Another popular approach involves explicitly mapping such features to the latent embeddings~\cite{Gantner:2010}.

{\bf Methods}.
Ranking approaches such as popularity-based ranking have been shown to
work surprisingly well~\cite{mcfee2012million,bonnin2013evaluating,bonnin2015automated}.
The reason is believed to be the long-tail distribution of songs in
playlists~\cite{cremonesi2010performance,bonnin2013evaluating}.
%
This approach has been further improved by taking into account artist information in addition to
song popularity, which creates a comparably strong baseline that outperforms many sophisticated
approaches such as Bayesian Personalised Ranking based approach, neighbourhood methods, and approaches
making use of association rules and sequential patterns~\cite{mcfee2012million,bonnin2013evaluating}.
%
Ranking method has also been used as a post-processing component in more sophisticated approaches,
where a subset of songs were selected by scoring~\cite{jannach2015beyond} or matching~\cite{hariri2012context}
before ranking which optimises specific characteristics of the generated playlists.


Markov chains and related approaches were also widely used for playlist generation by casting the task
as language modelling problem~\cite{mcfee2011natural},
random walks in a hyper-graph~\cite{mcfee2012hypergraph} where songs were first grouped (by genre,
year of release, or social tags etc.) to serve as edges in the hyper-graph, or samples of Markov chains
in latent space where songs are embedded as (pairs of) points~\cite{chen2012playlist}.
%
Other techniques including playlist generation as sequential classification based on context~\cite{ben2017groove},
Gaussian process regression for user preference learning~\cite{platt2002learning},
topic models for sequential pattern mining and the well-known matrix factorisation techniques for learning
latent representations of songs, artists and users~\cite{mcfee2012hypergraph,chen2012playlist,ben2017groove}.


{\bf A diverse set of information} has been used for music recommendation,
such as song metadata (\eg song title, artist name, era, genre, albums etc.)~\cite{hariri2012context,platt2002learning},
content data (\eg lyrics, low level audio features etc.)~\cite{mcfee2011natural,mcfee2012hypergraph,jannach2015beyond,ben2017groove},
artists information (\eg artist popularity, collocation of artists etc.)~\cite{bonnin2013evaluating,ben2017groove},
user listening history and social interactions (\eg social tags) as well as usage statistics (\eg song popularity,
song co-occurrence etc.)~\cite{mcfee2012hypergraph,hariri2012context,bonnin2013evaluating,jannach2015beyond,ben2017groove}.
There are a few work that make use of latent representations of song, user and artist which are learned from existing playlists,
or user-song and user-artist interactions~\cite{chen2012playlist,ben2017groove}.

The sequential order of songs in playlist has not been well understood~\cite{schedl2017},
some work suggest that song order and song-to-song transitions are important
for playlist quality~\cite{mcfee2012hypergraph,kamehkhosh2018automated},
while other work have shown that the ensemble of songs in playlist do matter,
but the song order seems to be negligible~\cite{tintarev2017sequences,vall2017importance}.
In this work, we treat a playlist as a set of songs by discarding the sequential order,
and leave the investigation of using song order to assist playlist generation as future work.

%\cheng{In 2-3 sentences, re-define the problem we are attacking in this paper.}



%It has been known that binary classification and bipartite ranking are
%closely related~\cite{ertekin2011equivalence,menon2016bipartite}.
%In particular, \citet{ertekin2011equivalence} have shown that the P-Norm Push bipartite ranking loss~\cite{rudin2009p}
%is equivalent to the P-Classification loss~\cite{ertekin2011equivalence} when using the exponential surrogate.
%Further, the P-Norm Push loss is an approximation of the Infinite-Push loss~\cite{agarwal2011infinite},
%or equivalently, the Top-Push loss~\cite{li2014top}, which focuses on the highest ranked negative example instead of
%of the lowest ranked positive example as in~(\ref{eq:bprisk}).
%
%Inspired by these connections, we seek a classification loss that is equivalent to a bipartite ranking loss (under a few assumptions),
%which can approximate the risk with Bottom-Push loss in~(\ref{eq:bprisk}).
%This will make it possible to formulate an unconstrained objective that approximates the empirical loss $\RCal_\textsc{rank}$.


{\bf Cold-start music recommendation}.
Content based recommendation approaches~\cite[Chapter~4]{aggarwal2016recommender}
can be adopted to recommend {\it cold songs} (\ie new songs),
typically by making use of content features of songs extracted either automatically~\cite{seyerlehner2010automatic,eghbal2015vectors}
or manually by musical experts~\cite{john2006pandora}.
Further, content features can also be combined with other approaches, such as those based on 
collaborative filtering~\cite{yoshii2006hybrid,donaldson2007hybrid,shao2009music}.
This is known as the hybrid recommendation approaches~\cite{burke2002hybrid}, 
see~\cite[Chapter~6]{aggarwal2016recommender} for a general description besides music recommendation.
The problem of recommending music for {\it cold users} (\ie new users) 
has also been tackled by a number of approaches, such as transferring user preferences learned 
from related domains~\cite{hu2010study,aizenberg2012build},
or techniques that balance the exploration-exploitation trade-off~\cite{wang2014exploration,liebman2015dj}.

\section{Conclusion}

We investigate the problem of recommending playlists to users in cold-start settings.
In the cold songs setting, we recommend newly released songs to extend existing playlists;
in the cold playlists setting, we recommend a set of songs to form a new playlist for an existing user;
and in the cold users setting, we recommend a set of songs to form a new playlist for a new user.
We deal with all three settings using a multitask learning method which encourages songs in playlist 
to be ranked higher than those are not by minimising a bipartite ranking loss. 
We formulate the objective as a constrained convex optimisation problem, and further approximates it 
by an unconstrained objective inspired by an equivalence relationship between bipartite ranking and
binary classification. 
Empirical results on two real music playlist datasets show the proposed approach 
has good performance for playlist recommendation in cold-start settings.

We are aware of a few limitations of the proposed approach, which we leave as future work.
Specifically, additional data sources (\eg music information shared on social media) or song/user 
features (\eg lyrics, user profile), as well as the sequential order of songs which could provide 
additional information to help make better recommendations.
Further, non-linear models such as deep neural networks have shown strong performance in a wide arrange of tasks,
and the linear model with sparse parameters in this work could potentially be more compact if non-linear objective were employed.

Finally, as a remark, we want to mention the challenge of evaluating the recommended results.
While metrics in information retrieval are commonly used, recommender system is more like a generative process
than a information retrieval task. Fortunately, this challenge has been noticed and been attacked in many 
ways~\cite{mcfee2011natural,mcfee2012hypergraph,schedl2017}, 
we believe that promising automatic evaluation methods that accepted by the (majority of) 
community is one premise of significant progress in music recommendation.


% Acknowledgements should only appear in the accepted version.
%\section*{Acknowledgements}
%
%\textbf{Do not} include acknowledgements in the initial version of
%the paper submitted for blind review.
%\clearpage

%\setlength\bibsep{2pt} % from natbib

\fontsize{9.5pt}{10.5pt} \selectfont
\bibliographystyle{aaai}
\bibliography{ref,ref_aditya}

\clearpage
\onecolumn
% !TEX root = ./main.tex

\appendix
%\section{Supplementary material to Structured Recommendation}
%\label{sec:supplement}
%{\Large\bf Supplementary material to Structured Recommendation}
\begin{center}
  {\Large\bf Supplementary material for ``Trajectory Recommendation via Structured Prediction''}
\end{center}

%\setlength{\belowdisplayskip}{2pt} \setlength{\belowdisplayshortskip}{1pt}
%\setlength{\abovedisplayskip}{2pt} \setlength{\abovedisplayshortskip}{1pt}

\section{Model learning and prediction}
\label{sec:supplement}

In this section, we describe the 1-slack formulation for the proposed SR model 
and the details of the list Viterbi algorithm.

\subsection{1-slack formulation for the SR model}
\label{ssec:1slack_sr}

We can use \emph{one} slack variable to represent the sum of the $N$ hinge losses:
\begin{equation*}
%%\label{eq:hingeloss}
%%\resizebox{1.1\linewidth}{!}{
%%\begin{minipage}{\linewidth}
%\begin{align*}
\xi_i = \max \left( 0, \, 
        \max_{\bar{\y} \in \mathcal{Y}}
        \left\{ \Delta(\y^{(i)}, \bar{\y}) + \w^\top \Psi(\x^{(i)}, \bar{\y}) \right\} - \w^\top \Psi(\x^{(i)}, \y^{(i)}) \right).
%\end{align*}
%%\end{minipage}
%%}
\end{equation*}
Which results in the 1-slack formulation for the SR model:
\begin{equation*}
%%\label{eq:1slack_ml}
%\resizebox{0.9\linewidth}{!}{$
\begin{aligned}
\min_{\w, \, \xi \ge 0} ~\frac{1}{2} \w^\top \w + C \xi, ~~s.t.~ \frac{1}{N} \left( \sum_{i,j} \w^\top \Psi(\x^{(i)}, \y^{(ij)}) - \w^\top \Psi(\x^{(i)}, \bar{\y}^{(i)}) \right) 
  \ge \frac{1}{N} \sum_{i,j} \Delta(\y^{(ij)}, \bar{\y}^{(i)}) - \xi.
\end{aligned}
%$}
\end{equation*}


\subsection{The list Viterbi algorithm}
\label{sec:listviterbi-supp}

We make use of the (serial) list Viterbi in four situations:
\begin{enumerate}
  \item To avoid sequence with loops during the prediction phase of the SP and SR models
  \item To make top-$k$ prediction using the SP and SR models
  \item To eliminate known ground truths during the training phase (\ie loss augmented inference) of the SR and \textsc{SRpath} models
  \item To avoid sequence with loops during the training phase (\ie loss augmented inference) of the \textsc{SPpath} and \textsc{SRpath} models
\end{enumerate}

Here, we describe two well-known proposals of the serial list Viterbi algorithm in the context of hidden Markov models (HMMs).
Suppose we have an HMM with states $\SSf_{t}$, observations $\OSf_{t}$, transitions $a(i, j) = \Pr( \SSf_{t+1} = j \mid \SSf_t = i )$, and emissions $b(i, k) = \Pr( \OSf_{t} = k \mid \SSf_t = i )$.
Suppose $s^*_{1:T}$ is the most likely length $T$ sequence given observations $\OSf_{1:T}$, and
$\delta(j, t)$ is the value of the best sequence up till position $t$ ending at state $j$ as computed by the Viterbi algorithm.

To find the second-best sequence with value $ M \defEq \max_{\SSf_{1:T} \neq s^*_{1 : T}}{\Pr( \SSf_{1:T}, \OSf_{1:T} )}. $
\citet{seshadri1994list} observed that $M = \bar{\delta}_{T+1}$, where $\bar{\delta}_t$ has a Viterbi-like recurrence
\begin{equation}
    \label{eqn:att-recurrence}
    %\resizebox{0.9\linewidth}{!}{$
    \begin{aligned}
        \bar{\delta}_t &\defEq 
        \indicator{t > 0} \cdot
        \max
        \begin{cases}
        \max_{i \neq s^*_{t-1}} \delta(i, t-1) \cdot a( i, s^*_{t} ) \cdot b( s^*_{t}, \OSf_{t} ) \\
        \bar{\delta}_{t - 1} \cdot a( s^*_{t - 1}, s^*_{t} ) \cdot b( s^*_{t}, \OSf_{t} ).
        \end{cases}
    \end{aligned}
    %$}
\end{equation}
Intuitively, $\bar{\delta}_t$ finds the value of the second-best sequence that merges with the best sequence by at least time $t$,
to find the third, \dots, k-th best sequences, this algorithm keeps track of the ``next-best'' sequence terminating at each state in the current list of best sequences.

On the other hand, \citet{nilsson2001sequentially} observed that $M = \max_t \widehat{\rho}_t$, where
\begin{align*}
	\widehat{\rho}_{t} &\defEq \max_{i \neq s^*_{t}} \max_{S_{t+1:T}} {\Pr( \SSf_{1:t-1} = s^*_{1:t-1}, \SSf_t = i, \SSf_{t+1:T}, \OSf_{1:T} )}.
\end{align*}
Intuitively, $\widehat{\rho}_t$ finds the value of the second-best sequence that first deviates from the best sequence exactly at time $t$,
We note that one can compute $\widehat{\rho}_{t}$ using $\eta_{i, j, t} \defEq \max_{\SSf \colon S_{t - 1} = i, \SSf_{t} = j} \Pr( \SSf_{1:T}, \OSf_{1:T} )$ \citep{nilsson2001sequentially},
which in turn can be computed from the ``backward'' analogue of $\delta$.
In the original work, \citeauthor{nilsson2001sequentially} illustrated this approache as cleverly partitions the search space into subsets of sequences that share a prefix with the current list of best sequences.

If we unroll the recurrence in Equation \ref{eqn:att-recurrence}, and note the definition of $\delta$, we have
%\resizebox{\linewidth}{!}{
%    \begin{minipage}{\linewidth}
        \begin{align*}
        	M &= \max_t \widehat{\mu}_t \\
        	\widehat{\mu}_{t} &\defEq \left[ \prod_{k = t+2}^T a( s^*_{k-1}, s^*_{k} ) \cdot b( s^*_{k}, \OSf_{k} ) \right] \cdot \max_{i \neq s^*_{t}} \delta(i, t) \cdot a( i, s^*_{t+1} ) \cdot b( s^*_{t+1}, \OSf_{t+1} ) \\
        	&= \max_{i \neq s^*_{t}} \max_{S_{1:t-1}} {\Pr( \SSf_{1:t-1}, \SSf_t = i, \SSf_{t+1:T} = s^*_{t+1:T}, \OSf_{1:T} )};
        \end{align*}
%    \end{minipage}
%}%
\ie, the two approaches computed the same quantities, 
while the former fixes the suffix of the candidate sequence, and the latter fixes the prefix. 
The case of finding the $K$-th best sequence can be shown in a similar manner.

For sequence recommendation, as the underlying structure of SSVM is a Markov chain, which means we can use the same algorithms described above, 
except that the emissions are uniform distributions since all states are observed.

%%There are two general approaches for generalising Viterbi decoding to when we require $k$
%%sequences to be decoded: by maintaining $k$ paths through the trellis while decoding; or by
%%careful book keeping of the best $k-1$ paths through the trellis found so far and avoiding them.
%%We choose the latter approach as it is more memory efficient.

%The serial list Viterbi algorithm~\cite{nilsson2001sequentially,seshadri1994list} maintains
%a heap (\ie priority queue) of potential solutions, which are then checked for the desired property (for example
%whether there are loops). Once the requisite number of trajectories with the desired
%property are found, the algorithm terminates (for example once $k$ trajectories without loops are found when performing top-k prediction)
%The heap is initialised by running forward-backward (Algorithm~\ref{alg:forward-backward}) followed by the vanilla Viterbi (Algorithm~\ref{alg:viterbi}).
%
%\begin{algorithm}[htbp]
%\caption{Forward-backward procedure~\cite{rabiner1989tutorial}}
%\label{alg:forward-backward}
%\begin{algorithmic}[1]
%  \STATE $\forall p_j \in \mathcal{P},~ \alpha_t(p_j) =
%          \begin{cases}
%          0,~ t = 1 \\
%          \max_{p_i \in \mathcal{P}} \left\{ \alpha_{t-1}(p_i) + \mathbf{w}_{ij}^\top \psi_{ij}(\mathbf{x}, p_i, p_j) +
%          \mathbf{w}_j^\top \psi_j(\mathbf{x}, p_j) \right\},~ t=2,\dots,l
%          \end{cases}$
%
%  \STATE $\forall p_i \in \mathcal{P},~ \beta_t(p_i) =
%          \begin{cases}
%          0,~ t = l \\
%          \max_{p_j \in \mathcal{P}} \left\{ \mathbf{w}_{ij}^\top \psi_{ij}(\mathbf{x}, p_i, p_j) +
%          \mathbf{w}_j^\top \psi_j(\mathbf{x}, p_j) + \beta_{t+1}(p_j) \right\},~ t = l-1,\dots,1
%          \end{cases}$
%
%  %\STATE $\forall p_i \in \mathcal{P},~ f_t(p_i) = \alpha_t(p_i) + \beta_t(p_i),~ t = 1,\dots,K$
%  \STATE $\forall p_i, p_j \in \mathcal{P},~ f_{t,t+1}(p_i, p_j) = \alpha_t(p_i) + \mathbf{w}_{ij}^\top \psi_{ij}(\mathbf{x}, p_i, p_j) +
%                                \mathbf{w}_j^\top \psi_j(\mathbf{x}, p_j) + \beta_{t+1}(p_j),~ t = 1,\dots,l-1$
%\end{algorithmic}
%\end{algorithm}
%
%\begin{algorithm}[htbp]
%\caption{Viterbi}
%\label{alg:viterbi}
%\begin{algorithmic}[1]
%  \STATE $y_t^1 = \begin{cases}
%                  s,~ t = 1 \\
%  %                \argmax_{p \in \mathcal{P}} \left\{ f_{1,2}(s, p) \right\},~ t = 2, \\
%                  \argmax_{p \in \mathcal{P}} \left\{ f_{t-1,t}(y_{t-1}^1, p) \right\},~ t = 2,\dots,l
%                  \end{cases}$
%
%  %\STATE $r^1 = \max_{p \in \mathcal{P}} \left\{ f_K(p) \right\}~~~ \triangleright$ $r^1$ is the score/priority of $\mathbf{y}^1$
%  \STATE $r^1 = \max_{p \in \mathcal{P}} \left\{ \alpha_{l}(p) \right\}~~~ \triangleright$ $r^1$ is the score/priority of $\mathbf{y}^1$
%\end{algorithmic}
%\end{algorithm}
%
%Given an existing heap containing potential trajectories,
%list Viterbi maintains a set of POIs $S$ to exclude, which is updated
%sequentially by considering the front sequence of the heap.
%
%Recall that for trajectory recommendation we are given the query $\mathbf{x}=(s, l)$, where
%$s$ is the starting POI from the set of POIs $\mathcal{P}$,
%and $l$ is the desired length of the trajectory.
%We assume the score function is of the form $\mathbf{w}^\top \Psi$ where $\Psi$ is the joint
%feature vector. $\mathbf{w}$ could be the value of the weight in the current iteration in training,
%or the learned weight vector during prediction.
%
%The list Viterbi algorithm for performing top-$k$ prediction is described in Algorithm~\ref{alg:listviterbi}.
%To eliminating known ground truths in loss augmented inference,
%we modify the forward-backward procedure (Algorithm~\ref{alg:forward-backward}) to account for the loss term $\Delta(\cdot,\cdot)$,
%and Algorithm~\ref{alg:viterbi} and Algorithm~\ref{alg:listviterbi} can be used without modification.

%\begin{algorithm}[htbp]
%\caption{The list Viterbi algorithm for top-$K$ prediction~\cite{nilsson2001sequentially,seshadri1994list}}
%\label{alg:listviterbi}
%\begin{algorithmic}[1]
%\STATE \textbf{Input}: $\mathbf{x}=(s, l),~ \mathcal{P},~ \mathbf{w},~ \Psi, ~K$
%%\STATE Initialise score matrices $\alpha,~ \beta,~ f_t,~ f_{t, t+1}$, a max-heap $H,~ k=0$.
%\STATE Initialise score matrices $\alpha,~ \beta,~ f_{t, t+1}$, a max-heap $H$, result set $R$, $k=0$.
%\STATE $\triangleright$ Do the forward-backward procedure (Algorithm~\ref{alg:forward-backward})
%\STATE $\triangleright$ Identify the best (scored) trajectory $\mathbf{y}^1=(y_1^1,\dots,y_l^1)$
%       with score $r^1$ using Viterbi (Algorithm~\ref{alg:viterbi}). $\y^1$ may be a trajectory that violates the desired condition.
%\STATE $H.\textit{push}\left(r^1,~ (\mathbf{y}^1, \textsc{nil}, \emptyset) \right)$
%\STATE Set $R=\emptyset$, the list of trajectories to be returned.
%\WHILE{$H \ne \emptyset$ \textbf{and} $k < \,|\mathcal{P}|^{l-1} - \prod_{t=2}^l (|\mathcal{P}|-t+1) + K$}
%    \STATE $r^k,~ (\mathbf{y}^k, I, S) = H.\textit{pop}()~~~ \triangleright$
%           $r^k$ is the score of $\mathbf{y}^k=(y_1^k,\dots,y_l^k)$, $I$ is the partition index, and $S$ is the exclude set
%    \STATE $k = k + 1$
%    \STATE Add $\mathbf{y}^k$ to $R$ if it satisfies the desired property
%    \RETURN $R$ if it contains the required number of trajectories
%    \STATE $I' = \begin{cases}
%                  2, & I = \textsc{nil} \\
%                  I, & \text{otherwise}
%                 \end{cases}$
%
%    \FOR{$t = I',\dots,l$}
%        \STATE $S' = \begin{cases}
%                      S \cup \{ y_t^k \}, & t = I' \\
%                      \{ y_t^k \},        & \text{otherwise}
%                     \end{cases}$
%
%        \STATE $y'_j = \begin{cases}
%                            y_j^k,                                                                             & j=1,\dots,t-1 \\
%                            \argmax_{p \in \mathcal{P} \setminus S'} \left\{ f_{t-1,t}(y_{t-1}^k, p) \right\}, & j=t \\
%                            \argmax_{p \in \mathcal{P}} \left\{ f_{j-1, j}(y'_{j-1}, p) \right\},              & j=t+1,\dots,l
%                \end{cases}$
%        %\STATE $\bar{r} = \begin{cases}
%        %                  f_{t-1,t}(y_{t-1}^k, \bar{y}_t),&I = \textsc{nil} \\
%        %                  r^k + f_{t-1,t}(y_{t-1}^k, \bar{y}_t) - f_{t-1,t}(y_{t-1}^k, y_t^k), &\text{otherwise}
%        %                  \end{cases}$
%        \STATE \vspace{3pt}$r' = r^k + f_{t-1,t}(y_{t-1}^k, y'_t) - f_{t-1,t}(y_{t-1}^k, y_t^k)$
%
%        \STATE \vspace{3pt}$H.\textit{push}(r', (\y', t, S') )$ \vspace{2pt}
%    \ENDFOR
%\ENDWHILE
%\end{algorithmic}
%\end{algorithm}

%\clearpage

\section{Experiment}

In this section, we describe trajectory dataset used in experiments as well as features for each methods.
In addition, the details of evaluation and empirical results are also described.

\subsection{Photo trajectory dataset and features}
\label{sec:feature}

%\textbf{Dataset}.
\subsubsection{Dataset}
In the interests of reproducibility we present further details of our empirical experiment.
The histogram of the number of trajectories per query is shown in Figure~\ref{fig:hist_query},
where we can see each query has 4-9 ground truths (\ie trajectories) on average, and 30-60 trajectories at most.
The histogram of trajectory length (\ie the number of POIs in a trajectory) is shown in Figure~\ref{fig:hist_length},
where we can see the majority are short trajectories (\ie length $\le$ 5).

% % histogram of #ground truth
% \begin{figure}[t]
% 	\centering
% 	\includegraphics[width=.7\linewidth]{hist_query.pdf}
% 	\caption{Histograms of the number of trajectories per query.}
% 	\label{fig:hist_query}
% \end{figure}
% %
% %
% %% histogram of trajectory length
% \begin{figure}[t]
% 	\centering
% 	\includegraphics[width=.7\linewidth]{hist_length.pdf}
% 	\caption{Histograms of trajectory length.}
% 	\label{fig:hist_length}
% \end{figure}


%\textbf{Features}.
\subsubsection{Features}
The POI-query features used by \textsc{PoiRank}, SP and SR methods and their extensions 
(\ie the \textsc{SPpath} and \textsc{SRpath} models) are shown in Table~\ref{tab:poifeature},
pairwise features used in SP and SR methods and their extensions are shown in Table~\ref{tab:tranfeature}.

\begin{table}[!h]
\small
\setlength{\tabcolsep}{10pt} % tweak the space between columns
\begin{tabular}{l|l} \hline
\textbf{Feature}       & \textbf{Description} \\ \hline
\texttt{category}      & one-hot encoding of the category of $p$ \\
\texttt{neighbourhood} & one-hot encoding of the POI cluster that $p$ resides in \\
\texttt{popularity}    & logarithm of POI popularity of $p$ \\
\texttt{nVisit}        & logarithm of the total number of visit by all users at $p$ \\
\texttt{avgDuration}  & logarithm of the average visit duration at $p$ \\
\hline
%\texttt{nOccurrence}            & the number of times $p$ occurred in a trajectory that satisfies the query \\ DON'T know given new query

\texttt{trajLen}           & trajectory length $l$, i.e., the number of POIs required \\
\texttt{sameCatStart}      & $1$ if the category of $p$ is the same as that of $s$, $-1$ otherwise \\
\texttt{sameNeighbourhoodStart} & $1$ if $p$ resides in the same POI cluster as $s$, $-1$ otherwise \\
\texttt{diffPopStart}    & real-valued difference in POI popularity of $p$ from that of $s$ \\
\texttt{diffNVisitStart}        & real-valued difference in the total number of visit at $p$ from that at $s$ \\
\texttt{diffDurationStart}  & real-valued difference in average duration at $p$ from that at $s$ \\
\texttt{distStart}          & distance between $p$ and $s$, calculated using the Haversine formula \\
\hline
\end{tabular}
\caption{POI-query features: features of POI $p$ with respect to query $(s,l)$}
\label{tab:poifeature}
\centering
\end{table}



\begin{table}[!h]
\centering
\small
\setlength{\tabcolsep}{2pt} % tweak the space between columns
\begin{tabular}{l|l} \hline
\textbf{Feature}       & \textbf{Description} \\ \hline
\texttt{category}      & category of POI \\
\texttt{neighbourhood} & the cluster that a POI resides in \\
\texttt{popularity}    & (discretised) popularity of POI \\
\texttt{nVisit}        & (discretised) total number of visit at POI \\
\texttt{avgDuration}  & (discretised) average duration at POI \\ \hline
\end{tabular}
\caption{Pairwise POI features}
\label{tab:tranfeature}
\end{table}


%\clearpage
\subsection{Evaluation settings}
\label{sec:metric}

%\textbf{Top-k prediction for baselines}.
\subsubsection{Top-k prediction for baselines}
\begin{itemize}
\item To perform top-$k$ prediction with \textsc{Random} baseline, we simply repeat the \textsc{Random} method $k$ times.
\item To perform top-$k$ prediction with \textsc{Popularity} and \textsc{PoiRank}, we make use of the list Viterbi algorithm 
      %(Algorithm~\ref{alg:listviterbi} 
      to get $k$ best scored paths, in particular, 
      for \textsc{Popularity}, the score of a path is the accumulated popularity of all POIs in the path; 
      for \textsc{PoiRank}, the score of a path is the likelihood 
      (the ranking scores for POIs are first transformed into a probability distribution using the softmax function, as described in~\cite{cikm16paper}).
\end{itemize}

To evaluate the performance of a certain recommendation algorithm,
we need to measure the similarity (or loss) given prediction $\hat{\mathbf{y}}$
and ground truth $\mathbf{y}$.


%\textbf{F$_1$ score on points}.
\subsubsection{F$_1$ score on points}
F$_1$ score on points~\cite{ijcai15} cares about only the set of correctly recommended POIs.
%%Let $\texttt{set}(\mathbf{y})$ denote the set of POIs in trajectory $\mathbf{y}$, F$_1$ score on points is defined as
\begin{equation*}
F_1(\mathbf{y}, \hat{\mathbf{y}}) = \frac{2  P_{\textsc{point}}  R_{\textsc{point}}}{P_{\textsc{point}} + R_{\textsc{point}}}
%~~\text{where}~
%P_{\textsc{point}} = \frac{| \texttt{set}(\hat{\mathbf{y}}) \cap \texttt{set}(\mathbf{y}) |}{| \texttt{set}(\hat{\mathbf{y}}) |}~\text{and}~
%R_{\textsc{point}} = \frac{| \texttt{set}(\hat{\mathbf{y}}) \cap \texttt{set}(\mathbf{y}) |}{| \texttt{set}(\mathbf{y}) |}.
\end{equation*}
where $P_\textsc{point}$, $R_\textsc{point}$ are respectively the precision and recall for points in $\hat\y$ and $\y$.
If $| \hat{\mathbf{y}} | = | \mathbf{y} |$, this metric is just the unordered Hamming loss,
i.e., Hamming loss between two binary indicator vectors of size $| \mathcal{P} |$.

%\textbf{F$_1$ score on pairs}.
\subsubsection{F$_1$ score on pairs}
To take into account the orders in recommended sequence, 
we also use the F$_1$ score on pairs~\cite{cikm16paper} measure, which considers the set of correctly predicted POI pairs,
\begin{equation*}
\text{pairs-F}_1(\mathbf{y}, \hat{\mathbf{y}}) = \frac{2 P_{\textsc{pair}} R_{\textsc{pair}}}{P_{\textsc{pair}} + R_{\textsc{pair}}}
%~~\text{where}~
%P_{\textsc{pair}} = \frac{N_c} {| \texttt{set}(\hat{\mathbf{y}}) | (| \texttt{set}(\hat{\mathbf{y}}) | - 1) / 2}~\text{and}~
%R_{\textsc{pair}} = \frac{N_c} {| \texttt{set}(\mathbf{y}) | (| \texttt{set}(\mathbf{y}) | - 1) / 2},
\end{equation*}
where $P_\textsc{point}$, $R_\textsc{point}$ are respectively the precision and recall for all possible pairs of $\hat\y$ and $\y$.
%%\eat{
%%and $N_c = \sum_{j=1}^{| \mathbf{y} | - 1} \sum_{k=j+1}^{| \mathbf{y} |} \llb y_j \prec_{\bar{\mathbf{y}}} y_k \rrb$,
%%here $y_j \prec_{\bar{\mathbf{y}}} y_k$ denotes that POI $y_j$ appears before POI $y_k$ in trajectory $\bar{\mathbf{y}}$.
%%We define pairs-F$_1 = 0$ when $N_c = 0$.
%%}


%\textbf{Kendall's $\tau$ with ties}
\subsubsection{Kendall's $\tau$ with ties}
Alternatively, we can cast a trajectory $\y = y_{1:l}$ as a ranking of POIs in $\mathcal{P}$,
where $y_j$ has a rank $| \mathcal{P} | - j + 1$ and any other POI $p \notin \mathbf{y}$ has a rank $0$ ($0$ is an arbitrary choice),
then we can make use of ranking evaluation metrics such as Kendall's $\tau$ by taking care of ties in ranks.
In particular, given a prediction $\hat\y = \hat{y_{1:l}}$ and ground truth $\y = y_{1:l}$,
we produce two ranks for $\mathbf{y}$ and $\hat{\mathbf{y}}$ with respect to 
a specific ordering of POIs $(p_1, p_2, \dots, p_{|\mathcal{P}|})$:
\begin{align*}
r_i       &= \sum_{j=1}^l (| \mathcal{P} | - j + 1)  \llb p_i = y_j \rrb,~
i = 1, \dots, | \mathcal{P} | \\
\hat{r}_i &= \sum_{j=1}^l (| \mathcal{P} | - j + 1)  \llb p_i = \hat{y}_j \rrb,~
i = 1, \dots, | \mathcal{P} |
\end{align*}
where POIs not in $\mathbf{y}$ will have a rank of $0$.
Then we compute the following metrics:
\begin{itemize}
\item the number of concordant pairs \(
      C = \frac{1}{2} \sum_{i,j} \left(\llb r_i < r_j \rrb  \llb \hat{r}_i < \hat{r}_j \rrb +
                      \llb r_i > r_j \rrb  \llb \hat{r}_i > \hat{r}_j \rrb \right) \)
\item the number of discordant pairs \(
      D = \frac{1}{2} \sum_{i,j} \left(\llb r_i < r_j \rrb  \llb \hat{r}_i > \hat{r}_j \rrb +
                      \llb r_i > r_j \rrb  \llb \hat{r}_i < \hat{r}_j \rrb \right) \)
\item the number of ties in ground truth $\y$: \(
      T_{\mathbf{y}} = \frac{1}{2} \sum_{i \ne j} \llb r_i = r_j \rrb 
%                     = \frac{1}{2} \sum_{i \ne j} \llb r_i = 0 \rrb  \llb r_j = 0 \rrb 
                     = \frac{1}{2} \left( |\mathcal{P}| - l \right) \left( |\mathcal{P}| - l - 1 \right) \)
\item the number of ties in prediction $\hat\y$: \(
      T_{\hat{\mathbf{y}}} = \frac{1}{2} \sum_{i \ne j} \llb \hat{r}_i = \hat{r}_j \rrb 
%                           = \frac{1}{2} \sum_{i \ne j} \llb \hat{r}_i = 0 \rrb  \llb \hat{r}_j = 0 \rrb 
                           = \frac{1}{2} \left( |\mathcal{P}| - l \right) \left( |\mathcal{P}| - l - 1 \right) \)
\item the number of ties in both $\y$ and $\hat\y$: \(
      T_{\mathbf{y},\hat{\mathbf{y}}} = \frac{1}{2} \sum_{i \ne j} \llb r_i = r_j \rrb  \llb \hat{r}_i = \hat{r}_j \rrb \)
%                                      = \frac{1}{2} \sum_{i \ne j} \llb r_i = 0 \rrb  \llb r_j = 0 \rrb
%                                        \llb \hat{r}_i = 0 \rrb  \llb \hat{r}_j = 0 \rrb \)
\end{itemize}
Kendall's $\tau$ (version $b$)~\cite{agresti2010analysis} is %,kendall1945} is
\begin{equation*}
\tau_b(\mathbf{y}, \hat{\mathbf{y}}) = \frac{C - D}{\sqrt{(C + D + T) (C + D + U)}},
\end{equation*}
where $T = T_{\mathbf{y}} - T_{\mathbf{y},\hat{\mathbf{y}}}$ and $U = T_{\hat{\mathbf{y}}} - T_{\mathbf{y},\hat{\mathbf{y}}}$.

%Furthermore, F$_1$ score on points can be written as
%\begin{equation*}
%F_1(\mathbf{y}, \hat{\mathbf{y}}) = \frac{1}{l} \sum_i \llb r_i > 0 \rrb  \llb \hat{r}_i > 0 \rrb,
%\end{equation*}
%and F$_1$ score on pairs can be written as
%\begin{align*}
%& \text{pairs-F}_1(\mathbf{y}, \hat{\mathbf{y}}) \\
%&= \left( \frac{1}{2} \sum_{i,j} \llb r_i < r_j \rrb  \llb r_i > 0 \rrb \llb \hat{r}_i < \hat{r}_j \rrb  \llb \hat{r}_i > 0 \rrb \right. \\
%&  \left. ~+ \frac{1}{2} \sum_{i,j} \llb r_i > r_j \rrb  \llb r_j > 0 \rrb \llb \hat{r}_i > \hat{r}_j \rrb  \llb \hat{r}_j > 0 \rrb \right)
%   \cdot \frac{1}{l(l-1)/2} \\
%&= \frac{\sum_{i,j} \llb r_i < r_j \rrb  \llb r_i > 0 \rrb \llb \hat{r}_i < \hat{r}_j \rrb  \llb \hat{r}_i > 0 \rrb +
%         \sum_{i,j} \llb r_i > r_j \rrb  \llb r_j > 0 \rrb \llb \hat{r}_i > \hat{r}_j \rrb  \llb \hat{r}_j > 0 \rrb}
%        {l(l-1)}
%\end{align*}

%\clearpage
\subsection{Empirical results}

%\textbf{The effects of $k$ for top-$k$ prediction}.
\subsubsection{The effects of $k$ for top-$k$ prediction}
The performance of baselines and structured recommendation algorithms for top-$k$ ($k=1,3,5,10$) 
prediction are shown in the following tables.
Dark gray entries: best performing method for each metric; light gray entries: the \textit{next best}. 

% topk evaluation table
\begin{table*}[!h]
\centering
\small
\setlength{\tabcolsep}{3pt} % tweak the space between columns
\begin{tabular}{l|cc|ccc|ccc} \hline
& \multicolumn{8}{c}{\bf Kendall's $\tau$} \\ \hline
 & \textsc{Random} & \textsc{Popularity} & \textsc{PoiRank} & \textsc{Markov} & \textsc{SP} & \textsc{SPpath} & \textsc{SR} & \textsc{SRpath} \\ \hline
Osaka & $0.420\pm0.030$ & $0.566\pm0.034$ & $\firstBest{0.644\pm0.040}$ & $0.600\pm0.036$ & $0.525\pm0.037$ & $0.525\pm0.039$ & $0.608\pm0.042$ & $\secondBest{0.613\pm0.044}$ \\
Glasgow & $0.430\pm0.031$ & $0.644\pm0.036$ & $\firstBest{0.733\pm0.030}$ & $0.623\pm0.030$ & $0.564\pm0.029$ & $0.615\pm0.034$ & $0.708\pm0.031$ & $\secondBest{0.712\pm0.031}$ \\
Toronto & $0.394\pm0.025$ & $0.626\pm0.023$ & $\secondBest{0.714\pm0.024}$ & $0.629\pm0.023$ & $0.543\pm0.026$ & $0.572\pm0.026$ & $0.714\pm0.026$ & $\firstBest{0.717\pm0.026}$ \\
\hline
& \multicolumn{8}{c}{\bf F$_1$ score on points} \\ \hline
Osaka & $0.459\pm0.027$ & $0.601\pm0.031$ & $\firstBest{0.678\pm0.037}$ & $0.630\pm0.034$ & $0.555\pm0.034$ & $0.558\pm0.036$ & $0.638\pm0.039$ & $\secondBest{0.645\pm0.040}$ \\
Glasgow & $0.478\pm0.027$ & $0.681\pm0.032$ & $\firstBest{0.764\pm0.027}$ & $0.654\pm0.027$ & $0.604\pm0.026$ & $0.653\pm0.031$ & $0.741\pm0.028$ & $\secondBest{0.743\pm0.028}$ \\
Toronto & $0.461\pm0.020$ & $0.671\pm0.021$ & $\secondBest{0.756\pm0.021}$ & $0.676\pm0.021$ & $0.594\pm0.023$ & $0.623\pm0.023$ & $0.753\pm0.023$ & $\firstBest{0.757\pm0.022}$ \\
\hline
& \multicolumn{8}{c}{\bf F$_1$ score on pairs} \\ \hline
Osaka & $0.104\pm0.037$ & $0.281\pm0.051$ & $\firstBest{0.428\pm0.059}$ & $0.331\pm0.053$ & $0.243\pm0.052$ & $0.254\pm0.055$ & $0.375\pm0.059$ & $\secondBest{0.401\pm0.060}$ \\
Glasgow & $0.154\pm0.035$ & $0.426\pm0.051$ & $\firstBest{0.545\pm0.046}$ & $0.368\pm0.045$ & $0.289\pm0.042$ & $0.389\pm0.048$ & $0.506\pm0.048$ & $\secondBest{0.516\pm0.048}$ \\
Toronto & $0.143\pm0.025$ & $0.381\pm0.034$ & $0.503\pm0.036$ & $0.391\pm0.034$ & $0.299\pm0.033$ & $0.340\pm0.035$ & $\secondBest{0.530\pm0.037}$ & $\firstBest{0.533\pm0.037}$ \\
\hline
\end{tabular}
\caption{Results on trajectory recommendation datasets on best of top-1.}
\end{table*}


\begin{table*}[!h]
\centering
\small
\setlength{\tabcolsep}{3pt} % tweak the space between columns
\begin{tabular}{l|cc|ccc|ccc} \hline
& \multicolumn{8}{c}{\bf Kendall's $\tau$} \\ \hline
 & \textsc{Random} & \textsc{Popularity} & \textsc{PoiRank} & \textsc{Markov} & \textsc{SP} & \textsc{SPpath} & \textsc{SR} & \textsc{SRpath} \\ \hline
Osaka & $0.556\pm0.037$ & $0.666\pm0.039$ & $\firstBest{0.726\pm0.042}$ & $\secondBest{0.718\pm0.039}$ & $0.630\pm0.044$ & $0.698\pm0.040$ & $0.711\pm0.042$ & $0.697\pm0.042$ \\
Glasgow & $0.563\pm0.031$ & $0.693\pm0.036$ & $0.781\pm0.030$ & $0.684\pm0.032$ & $0.666\pm0.033$ & $0.688\pm0.032$ & $\secondBest{0.803\pm0.029}$ & $\firstBest{0.808\pm0.030}$ \\
Toronto & $0.521\pm0.026$ & $0.670\pm0.025$ & $0.746\pm0.023$ & $0.712\pm0.023$ & $0.629\pm0.027$ & $0.650\pm0.027$ & $\firstBest{0.753\pm0.025}$ & $\secondBest{0.749\pm0.024}$ \\
\hline
& \multicolumn{8}{c}{\bf F$_1$ score on points} \\ \hline
Osaka & $0.587\pm0.034$ & $0.691\pm0.035$ & $\firstBest{0.750\pm0.039}$ & $\secondBest{0.740\pm0.037}$ & $0.656\pm0.040$ & $0.724\pm0.037$ & $0.735\pm0.038$ & $0.723\pm0.039$ \\
Glasgow & $0.598\pm0.028$ & $0.722\pm0.033$ & $0.803\pm0.027$ & $0.711\pm0.029$ & $0.698\pm0.030$ & $0.716\pm0.029$ & $\secondBest{0.825\pm0.026}$ & $\firstBest{0.829\pm0.026}$ \\
Toronto & $0.577\pm0.022$ & $0.704\pm0.023$ & $0.776\pm0.021$ & $0.748\pm0.021$ & $0.674\pm0.023$ & $0.693\pm0.023$ & $\firstBest{0.784\pm0.022}$ & $\secondBest{0.780\pm0.021}$ \\
\hline
& \multicolumn{8}{c}{\bf F$_1$ score on pairs} \\ \hline
Osaka & $0.288\pm0.055$ & $0.448\pm0.058$ & $\firstBest{0.578\pm0.060}$ & $0.538\pm0.060$ & $0.425\pm0.062$ & $0.511\pm0.059$ & $\secondBest{0.549\pm0.060}$ & $0.520\pm0.059$ \\
Glasgow & $0.300\pm0.043$ & $0.524\pm0.053$ & $0.625\pm0.046$ & $0.465\pm0.048$ & $0.464\pm0.049$ & $0.481\pm0.048$ & $\secondBest{0.666\pm0.045}$ & $\firstBest{0.678\pm0.045}$ \\
Toronto & $0.281\pm0.032$ & $0.473\pm0.036$ & $0.572\pm0.035$ & $0.517\pm0.035$ & $0.429\pm0.037$ & $0.461\pm0.037$ & $\firstBest{0.592\pm0.036}$ & $\secondBest{0.583\pm0.036}$ \\
\hline
\end{tabular}
\caption{Results on trajectory recommendation datasets on best of top-3.}
\end{table*}


\begin{table*}[!h]
\centering
\small
\setlength{\tabcolsep}{3pt} % tweak the space between columns
\begin{tabular}{l|cc|ccc|ccc} \hline
& \multicolumn{8}{c}{\bf Kendall's $\tau$} \\ \hline
 & \textsc{Random} & \textsc{Popularity} & \textsc{PoiRank} & \textsc{Markov} & \textsc{SP} & \textsc{SPpath} & \textsc{SR} & \textsc{SRpath} \\ \hline
 & \textsc{Random} & \textsc{Popularity} & \textsc{PoiRank} & \textsc{SP} & \textsc{SPpath} & \textsc{SR} & \textsc{SRpath} \\ \hline
Osaka & $0.618\pm0.038$ & $0.674\pm0.038$ & $\secondBest{0.750\pm0.040}$ & $\firstBest{0.769\pm0.036}$ & $0.678\pm0.045$ & $0.735\pm0.039$ & $0.741\pm0.039$ & $0.729\pm0.041$ \\
Glasgow & $0.623\pm0.029$ & $0.727\pm0.037$ & $0.801\pm0.030$ & $0.712\pm0.032$ & $0.727\pm0.033$ & $0.743\pm0.031$ & $\secondBest{0.826\pm0.028}$ & $\firstBest{0.832\pm0.028}$ \\
Toronto & $0.574\pm0.025$ & $0.687\pm0.025$ & $0.754\pm0.023$ & $0.749\pm0.024$ & $0.662\pm0.027$ & $0.683\pm0.026$ & $\firstBest{0.778\pm0.023}$ & $\secondBest{0.769\pm0.024}$ \\
\hline
& \multicolumn{8}{c}{\bf F$_1$ score on points} \\ \hline
Osaka & $0.646\pm0.035$ & $0.699\pm0.034$ & $\secondBest{0.772\pm0.037}$ & $\firstBest{0.789\pm0.033}$ & $0.700\pm0.041$ & $0.757\pm0.036$ & $0.761\pm0.036$ & $0.751\pm0.037$ \\
Glasgow & $0.655\pm0.026$ & $0.754\pm0.033$ & $0.821\pm0.026$ & $0.736\pm0.029$ & $0.755\pm0.030$ & $0.770\pm0.027$ & $\secondBest{0.847\pm0.024}$ & $\firstBest{0.850\pm0.025}$ \\
Toronto & $0.624\pm0.022$ & $0.719\pm0.023$ & $0.781\pm0.021$ & $0.783\pm0.021$ & $0.705\pm0.023$ & $0.724\pm0.022$ & $\firstBest{0.808\pm0.021}$ & $\secondBest{0.798\pm0.021}$ \\
\hline
& \multicolumn{8}{c}{\bf F$_1$ score on pairs} \\ \hline
Osaka & $0.375\pm0.058$ & $0.459\pm0.057$ & $\secondBest{0.607\pm0.058}$ & $\firstBest{0.621\pm0.055}$ & $0.507\pm0.064$ & $0.568\pm0.058$ & $0.584\pm0.058$ & $0.575\pm0.058$ \\
Glasgow & $0.377\pm0.044$ & $0.590\pm0.052$ & $0.670\pm0.045$ & $0.507\pm0.048$ & $0.563\pm0.048$ & $0.573\pm0.047$ & $\secondBest{0.701\pm0.043}$ & $\firstBest{0.715\pm0.044}$ \\
Toronto & $0.343\pm0.034$ & $0.500\pm0.036$ & $0.590\pm0.034$ & $0.581\pm0.034$ & $0.483\pm0.037$ & $0.509\pm0.037$ & $\firstBest{0.624\pm0.035}$ & $\secondBest{0.609\pm0.035}$ \\
\hline
\end{tabular}
\caption{Results on trajectory recommendation datasets on best of top-5.}
\end{table*}


\begin{table*}[!h]
\centering
\small
\setlength{\tabcolsep}{3pt} % tweak the space between columns
\begin{tabular}{l|cc|ccc|ccc} \hline
& \multicolumn{8}{c}{\bf Kendall's $\tau$} \\ \hline
 & \textsc{Random} & \textsc{Popularity} & \textsc{PoiRank} & \textsc{Markov} & \textsc{SP} & \textsc{SPpath} & \textsc{SR} & \textsc{SRpath} \\ \hline
Osaka & $0.685\pm0.035$ & $0.768\pm0.038$ & $0.787\pm0.037$ & $\firstBest{0.824\pm0.031}$ & $0.749\pm0.043$ & $0.791\pm0.036$ & $0.777\pm0.036$ & $\secondBest{0.803\pm0.034}$ \\
Glasgow & $0.703\pm0.029$ & $0.748\pm0.036$ & $0.830\pm0.029$ & $0.781\pm0.031$ & $0.790\pm0.030$ & $0.787\pm0.029$ & $\firstBest{0.868\pm0.026}$ & $\secondBest{0.853\pm0.026}$ \\
Toronto & $0.652\pm0.024$ & $0.719\pm0.024$ & $0.784\pm0.023$ & $0.789\pm0.022$ & $0.697\pm0.027$ & $0.719\pm0.026$ & $\firstBest{0.802\pm0.022}$ & $\secondBest{0.797\pm0.022}$ \\
\hline
& \multicolumn{8}{c}{\bf F$_1$ score on points} \\ \hline
Osaka & $0.703\pm0.032$ & $0.786\pm0.034$ & $0.804\pm0.034$ & $\firstBest{0.840\pm0.029}$ & $0.770\pm0.039$ & $0.809\pm0.033$ & $0.793\pm0.033$ & $\secondBest{0.820\pm0.031}$ \\
Glasgow & $0.731\pm0.026$ & $0.771\pm0.033$ & $0.847\pm0.025$ & $0.800\pm0.028$ & $0.810\pm0.027$ & $0.807\pm0.026$ & $\firstBest{0.883\pm0.023}$ & $\secondBest{0.868\pm0.023}$ \\
Toronto & $0.696\pm0.021$ & $0.746\pm0.022$ & $0.807\pm0.020$ & $0.819\pm0.019$ & $0.733\pm0.023$ & $0.755\pm0.022$ & $\firstBest{0.828\pm0.019}$ & $\secondBest{0.823\pm0.020}$ \\
\hline
& \multicolumn{8}{c}{\bf F$_1$ score on pairs} \\ \hline
Osaka & $0.451\pm0.057$ & $0.626\pm0.055$ & $0.661\pm0.056$ & $\firstBest{0.693\pm0.051}$ & $0.620\pm0.061$ & $0.664\pm0.055$ & $0.637\pm0.055$ & $\secondBest{0.671\pm0.053}$ \\
Glasgow & $0.495\pm0.046$ & $0.623\pm0.051$ & $0.726\pm0.043$ & $0.635\pm0.048$ & $0.658\pm0.046$ & $0.648\pm0.045$ & $\firstBest{0.770\pm0.039}$ & $\secondBest{0.746\pm0.041}$ \\
Toronto & $0.438\pm0.034$ & $0.546\pm0.036$ & $0.646\pm0.034$ & $0.644\pm0.033$ & $0.530\pm0.037$ & $0.552\pm0.036$ & $\firstBest{0.660\pm0.033}$ & $\secondBest{0.656\pm0.034}$ \\
\hline
\end{tabular}
\caption{Results on trajectory recommendation datasets on best of top-10.}
\end{table*}


%\clearpage
%\textbf{Performance on short and long trajectories}.
\subsubsection{Performance on short and long trajectories}
The performance of baselines and structured recommendation algorithms 
for short (length $<$ 5) and long (length $\ge$ 5) trajectories 
with top-$k$ ($k=1:10$) prediction are shown in the following figures.

\clearpage
% topk evaluation plots
%\includepdf[pages={1-},scale=0.75]{plots.pdf}
\includepdf[pages={1-}]{plot_topk.pdf}


\end{document}
