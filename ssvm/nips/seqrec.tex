%!TEX root = main.tex

%\section{Recommending sequences}
\secmoveup
\section{The sequence recommendation problem}
\label{sec:recseq}
\textmoveup

We introduce the sequence recommendation problem that is the focus of this paper.
%To do so, we first
We then provide some motivating examples, in particular the problem of trajectory recommendation.
%and show how it may be viewed as a kind of structured prediction problem.


%
%\subsection{Trajectory recommendation as a structured prediction problem}
%\section{Trajectory Recommendation as Structured Prediction}
\label{sec:trajrec}

The trajectory recommendation problem is: given a set of points-of-interest (POI) $\mathcal{P}$ and a trajectory query $\mathbf{x} = (s, K)$,
where $s \in \mathcal{P}$ is the desired start POI and $K > 1$ is the number of POIs in the desired trajectory (including the start location $s$).
We want to recommend a sequence of POIs $\mathbf{y}^*$ that maximises utility, i.e., for a suitable function $f(\cdot,\cdot)$,
\begin{equation*}
\mathbf{y}^* = \argmax_{\mathbf{y} \in \mathcal{Y}}~f(\mathbf{x}, \mathbf{y}),
\end{equation*}
where $\mathcal{Y}$ is the set of all possible trajectories with POIs in $\mathcal{P}$ and conform to query $\mathbf{x}$.
$\mathbf{y} = (y_1 = s,~ y_2, \dots, y_K)$ is a trajectory with $K$ POIs, and $y_j \ne y_k$ if $j \ne k$, 
i.e.,
there is no sub-tours in trajectory $\mathbf{y}$.


Instead of specifying the number of desired POIs, we can constrain the trajectory with a total time budget $T$.
In this case, the number of POIs $K$ can be treated as a \emph{hidden} variable, with additional constraint $\sum_{k=1}^K t_k \le T$
where $t_k$ is the time spent at POI $y_k$.


This problem is related to automatic playlist generation,
where we recommend a sequence of songs given a specified song (a.k.a. the seed) and the number of new songs.
Formally, given a library of songs and a query $\mathbf{x} = (s, K)$, where $s$ is the seed and $K$ is the number of songs in playlist,
we produce a list with $K$ songs (without duplication) by maximising the likelihood~\cite{chen2012playlist},
\begin{equation*}
%\max_{(y_1,\dots,y_K)} \prod_{k=2}^K \mathbb{P}(y_{k-1} \given y_k),~ y_1 = s ~\text{and}~ y_j \ne y_k,~ j \ne k.
\mathbf{y}^* = \argmax_{\mathbf{y} \in \mathcal{P}_\mathbf{x}}~ \mathbb{P}(\mathbf{y} \given \mathbf{x}),~ \mathbf{y} = (y_1=s,\dots,y_K)
~\text{and}~ y_j \ne y_k ~\text{if}~ j \ne k.
\end{equation*}

Another similar problem is choosing a small set of photos from a large photo library and compiling them into a slideshow or movie.


% standard SSVM
We can learn a recommender by training a SSVM on the set of observed trajectories $\{\mathbf{x}^{(i')}, \mathbf{y}^{(i')}\}_{i'=1}^{N'}$,
However, we ignore the fact that for the same query, we normally observed more than one trajectory,
we would like to exploit this fact to better modelling the observed trajectories.


\subsection{Query Aggregation}
\label{sec:query}

To modelling the fact that a given query has multiple observed trajectories, 
we firstly group trajectories according to queries, in other words,
we now have a dataset $\{\mathbf{x}^{(i)}, \{\mathbf{y}^{(ij)}\}_{j=1}^{N_i}\}_{i=1}^N$
with $N$ queries and queries $\mathbf{x}^{(i)}$ has $N_i$ trajectories observed.


\subsection{Recommendation with Multiset SSVM}
\label{sec:trajrec-ssvm}

We can learn to recommend trajectories by training a multiset SSVM described in Section~\ref{sec:ssvm-ms}

% multiset SSVM

% multiset SSVM: training, prediction




%
%\subsection{From trajectory to sequence recommendation}
\subsection{Structured and sequence recommendation}
\label{sec:seqrec-defn}

%We now generalise the previous discussion to cover a broad class of problems.
Consider the following abstract
\emph{structured recommendation} problem:
given an input query $\x \in \mathcal{X}$ (representing \eg a user, a location, or some ``seed'' song)
we wish to recommend one or more \emph{structured outputs} $\y \in \mathcal{Y}$ (representing \eg a sequence of locations, or songs)
according to a learned \emph{score function} $f(\x,\y)$.
To learn $f$,
we are provided as input a training set
%$(\x\pb{i}, \{ \y\pb{ij} \}_{j=1:n^i})$, $i=1:n$,
$\{ ( \x\pb{i}, \{ \y\pb{ij} \}_{j=1}^{n_i} ) \}_{i=1}^{n}$,
comprising a collection of inputs $\x\pb{i}$ with an associated \emph{set} of $n_i$ output structures $\{ \y\pb{ij} \}$.

For this work, we assume the output $\y$ is a \emph{sequence} of $l$ points, denoted $y_{1:l}$
where each $y_i$ belongs to some fixed set (e.g.\ places of interest in a city, or songs).
%For example, each $\y$ may be a sequences of places in a city, or a playlist of songs.
%Thus, for example, the training set might represent a collection of users in a city, along with a set of trajectories () they have visited.
We call the resulting specialisation the \emph{sequence recommendation} problem,
and this shall be our primary interest in this paper.
In many settings, one further requires the sequences to be \emph{paths} \ie not contain any repetitions.

As a remark, we note that the assumption that $\y$ is a sequence does not limit the generality of our approach,
as inferring $\y$ of other structure can be achieved using corresponding inference and loss-augmented inference algorithms~\cite{joachims2009predicting}.  %LX - this sentence can be cut or merged above


%
\subsection{Sequence recommendation versus existing problems}

There are key differences between sequence recommendation and %what is being solved in
standard problems in structured prediction and recommender systems;
%This setting generalises from structured prediction and recommendation problems in the following ways.
this brings unique challenges for both inference and learning.

In a structured prediction problem, the goal is to learn from a set of
input vector and output sequence tuples %$(\x\pb{i}, \y\pb{i})$, $i=1:n$.
$\{ (\x\pb{i}, \y\pb{i}) \}_{i = 1}^n$, where
for each input $\x\pb{i}$ there is usually one \emph{unique} output sequence $\y\pb{i}$.
In a sequence recommendation problem, however, we expect that %learn from
%tuples $(\x\pb{i}, \{ y\pb{ij} \}_{j=1:n^i})$, $i=1:n$. That is to say,
for each input $\x\pb{i}$ (\eg users),
there %is %have not one, but a set of
are \emph{multiple} associated outputs %$\{ y\pb{ij} \}_{j=1:n^i}$ (\eg movies).
$\{ \y\pb{ij} \}_{j=1}^{n_i}$ (\eg trajectories they have visited).
%Indeed, the existence of multiple outputs is the basis on which even non-structured recommendation systems are built, as one looks to exploit signal embedded in the aggregate information.
Structured prediction approaches do not have a standard way to handle such multiple output sequences.
%$\{ \y\pb{ij} \}_{j=1:n^i}$
%for each input %$\x\pb{i}$
%yet.

In a typical recommender systems problem, the outputs are non-structured; canonically, one works with {static} content such as books or movies~\citep{Goldberg:1992,Sarwar:2001,Netflix}.
Thus, making a prediction involves enumerating all {\em non-structured} items $y$ in order to compute $\argmax_y f(\x,y)$ for suitable score function $f$ \eg some form of matrix factorisation~\citep{Koren:2009}.
For sequence recommendation, computing $\argmax_\y f(\x,\y)$ is harder since it is often impossible to efficiently enumerate $\y$ (\eg all possible trajectories in a city).
This inability to enumerate $\y$ also poses a challenge in designing a suitable $f(\x,\y)$;
\eg
%the standard matrix factorisation approach to recommender systems~\citep{Koren:2009}
matrix factorisation
would require associating a latent feature with each $\y$, which will be infeasible.


%
\subsection{Examples of sequence recommendation}
\label{sec:trajrec}

To make the sequence recommendation problem more concrete,
we provide three specific examples,
starting with the problem of trajectory recommendation
that shall serve as a recurring motivation.
%we explicate how a recently studied problem may be viewed as a special case.
Note that in all these problems, one is specifically interested in sequences that are paths.

\section{Trajectory Recommendation as Structured Prediction}
\label{sec:trajrec}

The trajectory recommendation problem is: given a set of points-of-interest (POI) $\mathcal{P}$ and a trajectory query $\mathbf{x} = (s, K)$,
where $s \in \mathcal{P}$ is the desired start POI and $K > 1$ is the number of POIs in the desired trajectory (including the start location $s$).
We want to recommend a sequence of POIs $\mathbf{y}^*$ that maximises utility, i.e., for a suitable function $f(\cdot,\cdot)$,
\begin{equation*}
\mathbf{y}^* = \argmax_{\mathbf{y} \in \mathcal{Y}}~f(\mathbf{x}, \mathbf{y}),
\end{equation*}
where $\mathcal{Y}$ is the set of all possible trajectories with POIs in $\mathcal{P}$ and conform to query $\mathbf{x}$.
$\mathbf{y} = (y_1 = s,~ y_2, \dots, y_K)$ is a trajectory with $K$ POIs, and $y_j \ne y_k$ if $j \ne k$, 
i.e.,
there is no sub-tours in trajectory $\mathbf{y}$.


Instead of specifying the number of desired POIs, we can constrain the trajectory with a total time budget $T$.
In this case, the number of POIs $K$ can be treated as a \emph{hidden} variable, with additional constraint $\sum_{k=1}^K t_k \le T$
where $t_k$ is the time spent at POI $y_k$.


This problem is related to automatic playlist generation,
where we recommend a sequence of songs given a specified song (a.k.a. the seed) and the number of new songs.
Formally, given a library of songs and a query $\mathbf{x} = (s, K)$, where $s$ is the seed and $K$ is the number of songs in playlist,
we produce a list with $K$ songs (without duplication) by maximising the likelihood~\cite{chen2012playlist},
\begin{equation*}
%\max_{(y_1,\dots,y_K)} \prod_{k=2}^K \mathbb{P}(y_{k-1} \given y_k),~ y_1 = s ~\text{and}~ y_j \ne y_k,~ j \ne k.
\mathbf{y}^* = \argmax_{\mathbf{y} \in \mathcal{P}_\mathbf{x}}~ \mathbb{P}(\mathbf{y} \given \mathbf{x}),~ \mathbf{y} = (y_1=s,\dots,y_K)
~\text{and}~ y_j \ne y_k ~\text{if}~ j \ne k.
\end{equation*}

Another similar problem is choosing a small set of photos from a large photo library and compiling them into a slideshow or movie.


% standard SSVM
We can learn a recommender by training a SSVM on the set of observed trajectories $\{\mathbf{x}^{(i')}, \mathbf{y}^{(i')}\}_{i'=1}^{N'}$,
However, we ignore the fact that for the same query, we normally observed more than one trajectory,
we would like to exploit this fact to better modelling the observed trajectories.


\subsection{Query Aggregation}
\label{sec:query}

To modelling the fact that a given query has multiple observed trajectories, 
we firstly group trajectories according to queries, in other words,
we now have a dataset $\{\mathbf{x}^{(i)}, \{\mathbf{y}^{(ij)}\}_{j=1}^{N_i}\}_{i=1}^N$
with $N$ queries and queries $\mathbf{x}^{(i)}$ has $N_i$ trajectories observed.


\subsection{Recommendation with Multiset SSVM}
\label{sec:trajrec-ssvm}

We can learn to recommend trajectories by training a multiset SSVM described in Section~\ref{sec:ssvm-ms}

% multiset SSVM

% multiset SSVM: training, prediction



