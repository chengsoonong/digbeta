% !TEX root = ./main.tex

\section{Trajectory Recommendation as a Structured Prediction Problem}
\label{sec:trajrec}

Travel recommendation problems involve a set of points-of-interest (POIs) $\mathcal{P}$ in a city.
The \emph{trajectory recommendation} problem is: given a \emph{trajectory query} $\mathbf{x} = (s, K)$,
comprising a start POI $s \in \mathcal{P}$ and trip length\footnote{Instead of specifying the number of desired POIs, we can constrain the trajectory with a total time budget $T$.
In this case, the number of POIs $K$ can be treated as a \emph{hidden} variable, with additional constraint $\sum_{k=1}^K t_k \le T$
where $t_k$ is the time spent at POI $y_k$.} $K > 1$ (including the start location $s$),
we want to recommend a sequence of POIs $\mathbf{y}^*$ that maximises some notion of utility.
That is, for a suitable function $f(\cdot,\cdot)$, we wish to find
\begin{equation*}
\mathbf{y}^* = \argmax_{\mathbf{y} \in \mathcal{Y}}~f(\mathbf{x}, \mathbf{y}),
\end{equation*}
where $\mathcal{Y}$ is the set of all possible trajectories with POIs in $\mathcal{P}$ that conform to the constraints imposed by the query $\mathbf{x}$.
In particular,
$\mathbf{y} = (y_1 = s,~ y_2, \dots, y_K)$ is a trajectory with $K$ POIs, which has no sub-tours i.e. $y_j \ne y_k$ if $j \ne k$.

% AKM: not sure these are suitable here
%
% This problem is related to automatic playlist generation,
% where we recommend a sequence of songs given a specified song (a.k.a. the seed) and the number of new songs.
% Formally, given a library of songs and a query $\mathbf{x} = (s, K)$, where $s$ is the seed and $K$ is the number of songs in playlist,
% we produce a list with $K$ songs (without duplication) by maximising the likelihood~\cite{chen2012playlist},
% \begin{equation*}
% %\max_{(y_1,\dots,y_K)} \prod_{k=2}^K \mathbb{P}(y_{k-1} \given y_k),~ y_1 = s ~\text{and}~ y_j \ne y_k,~ j \ne k.
% \mathbf{y}^* = \argmax_{\mathbf{y} \in \mathcal{P}_\mathbf{x}}~ \mathbb{P}(\mathbf{y} \given \mathbf{x}),~ \mathbf{y} = (y_1=s,\dots,y_K)
% ~\text{and}~ y_j \ne y_k ~\text{if}~ j \ne k.
% \end{equation*}

% Another similar problem is choosing a small set of photos from a large photo library and compiling them into a slideshow or movie.

To learn a suitable $f$ for trajectory recommendation,
we are provided as input a training set
%$(\x\pb{i}, \{ \y\pb{ij} \}_{j=1:n^i})$, $i=1:n$,
$\{ ( \x\pb{i}, \{ \y\pb{ij} \}_{j=1}^{n^i} ) \}_{i=1}^{n}$.
Here, each $\x\pb{i}$ is a distinct query,
while the corresponding $\{ \y\pb{ij} \}_{j=1}^{n^i}$ is the set of trajectories observed for that query.
Note that we expect most queries to have several distinct trajectories;
minimally,
for example,
there may two nearby POIs that are visited in interchangeable order by different travellers.

Comparing the above to the discussion in the previous section, it is clear that
we can cast trajectory recommendation as a special case of the structured recommendation problem.
Consequently, we may approach it using structured prediction methods such as the SSVM,
as well as the extensions proposed to account for multiple ground truths and eliminate loops during training.

% AKM: repetitive
%
% standard SSVM
% We can learn a recommender by training a SSVM on the set of observed trajectories $\{\mathbf{x}^{(i')}, \mathbf{y}^{(i')}\}_{i'=1}^{N'}$,
% However, we ignore the fact that for the same query, we normally observed more than one trajectory,
% we would like to exploit this fact to better modelling the observed trajectories.

% AKM: repetitive
%
% \subsection{Query Aggregation}
% \label{sec:query}

% To modelling the fact that a given query has multiple observed trajectories, 
% we firstly group trajectories according to queries, in other words,
% we now have a dataset $\{\mathbf{x}^{(i)}, \{\mathbf{y}^{(ij)}\}_{j=1}^{N_i}\}_{i=1}^N$
% with $N$ queries and queries $\mathbf{x}^{(i)}$ has $N_i$ trajectories observed.


% \subsection{Recommendation with Multiset SSVM}
% \label{sec:trajrec-ssvm}

% We can learn to recommend trajectories by training a multiset SSVM described in Section~\ref{sec:ssvm-ms}

% multiset SSVM

% multiset SSVM: training, prediction


