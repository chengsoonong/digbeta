% !TEX root = ./main.tex

\section{Experimental results}
\label{sec:experiment}

We now present experiments assessing the viability of our structured prediction approach
for the concrete sequential recommendation task of
trajectory recommendation.
We compare our methods to a number of non-structured baselines, over which we demonstrate significant improvements.


%
\subsection{Description of datasets}
\label{sec:dataset}

% experiment protocol: Nested cross-validation with Monte-Carlo cross-validation for the inner loop
We assess the recommendation performance %methods developed in Section~\ref{sec:trajrec}
of various methods
on trajectories extracted from Flickr photos
for the cities of Glasgow and Osaka~\cite{thomee2016yfcc100m}.
Each dataset comprises a
list of trajectories, being a sequence of points of interest (POI), as visited by various Flickr users.
The statistics of datasets are shown in Table~\ref{tab:data}.
\TODO{these seem to include a lot of irrelevant stats}
From this data, we consider the recommendation problem as posed in \S\ref{sec:trajrec}:
given a query comprising start POI and a desired trip length, can we recommend a trajectory of POIs that a visitor will enjoy?

A characteristic of these datasets is that many queries are associated with multiple trajectories, or ground truths.
The histograms of the number of ground truths for queries are shown in Figure~\ref{fig:hist}.

% dataset stats
\begin{table}[t]
\caption{Statistics of trajectory dataset}
\label{tab:data}
\centering
\setlength{\tabcolsep}{4pt} % tweak the space between columns
\small
\begin{tabular}{l*{5}{c}} \hline
\textbf{Dataset} & \textbf{\#Traj.} & \textbf{\#POIs} & \textbf{\#Users} & \textbf{\#Queries} & \textbf{AvgLenth} \\ \hline
Osaka            & 186              & 26              & 130              & 47                 & 2.4 \\
Glasgow          & 351              & 25              & 219              & 64                 & 2.5 \\
\hline
\end{tabular}
\end{table}


% histogram of #ground truth
\begin{figure*}[t]
	\centering
	\includegraphics[width=\columnwidth]{hist.pdf}
	\caption{Histogram of the number of ground truths per query on our datasets.}
	\label{fig:hist}
\end{figure*}


%
\subsection{Methods compared}

We compare the performance of our methods to the following three baselines:
\begin{itemize}[leftmargin=0.125in]
	\item The \textsc{Random} baseline simply recommends a sequence of POIs by sampling uniformly at random from the whole set of POIs (without replacement).
	
	\item The stronger \textsc{Popularity} baseline recommends the top-$k$ most popular POIs i.e. the POIs that have been visited the most in the training set.

	\item \textsc{PoiRank}~\cite{cikm16paper} is a generalisation of \textsc{Popularity} which considers a number of POI features in addition to the popularity,
and trains a RankSVM model to learn a score for each POI. The top-$k$ scoring POIs are then used to construct a trajectory.
\end{itemize}

To assess the viability of our structured prediction approach, and the necessity of our two extensions (normalising the loss per query and disallowing loops), we implemented the following versions of our structured prediction methods:
\begin{itemize}[leftmargin=0.125in]
	\item The structured prediction ({\sc SP}) method employs the vanilla structured SVM framework in order to learn a score for trajectories given a query.

	\item The structured recommendation ({\sc SR}) method extends the {\sc SP} method by additionally performing normalisation of the loss function per query,
	so that we do not attempt to distinguish between multiple ground truths for the same query.

	\item The variants {\sc SPpath} and {\sc SRpath} extend the above methods by enforcing the constraint during training that sequences with loops are disallowed.
\end{itemize}


%
\subsection{Evaluation procedure}

% leave-one-out evaluation (with query aggregation)
%As described in Section~\ref{sec:queryagg}, 
To evaluate the performance of the various methods under comparison,
we first group the trajectories according to queries that they conform to.
We then evaluate the performance of each algorithm using leave-one-out cross validation,
where in each iteration of this procedure,
one query and its associated trajectories serves as a test point, with all other trajectories for training.
(Note that without this query aggregation, there will be considerable overlap between the train and test set, and simple nearest neighbour methods will be hard to outperform.)
% model selection (Monte Carlo CV) (with query aggregation): 90/10 random split for 5 times
Furthermore, 
all relevant hyper-parameters (e.g.\ the regularisation constant $C$) are tuned using Monte Carlo cross validation~\cite{burman1989comparative} on the training set.

% evaluation metric: kendall's Tau (mention F1, pF1)
We use three performance measures to assess the test fold performance of each algorithm:
point-F$_1$ score~\cite{ijcai15},
pairs-F$_1$ score~\cite{cikm16paper},
and
Kendall's $\tau$ (version $b$)~\cite{kendall1945,agresti2010analysis}.
\TODO{explain point and pairs}

Kendall's $\tau$
essentially measures the fraction of pairs of POIs that are correctly ordered in the prediction and ground truth sequences,
with adjustment for ties.
In particular,
define the rank of POIs $\mathcal{P}$ in a trajectory $\mathbf{y} = (y_1,\dots,y_K)$ as
the position of the POI within the trajectory,
\begin{align*} 
r_\mathbf{y} &= (r_1,\dots,r_j,\dots,r_{|\mathcal{P}|}), \\
r_j &= \sum_{k=1}^K (| \mathcal{P} | - k + 1)  \llb p_j = y_k \rrb, ~ j = 1, \dots, | \mathcal{P} |.
\end{align*}
The Kendall's $\tau$ score is then computed on $r_\mathbf{y}$ versus $r_\mathbf{\hat{y}}$.
Unlike the F$_1$ scores on points and pairs, 
which only care about the set of correctly recommended POIs or POI pairs,
this metric taking both factors into account.

As described previously, our methods are capable of recommending not merely a single trajectory,
but rather a list of trajectories.
While one can simply take the top recommended trajectory as the prediction,
this ignores the fact that there are likely multiple plausible trajectories for any given query.
Thus, for each performance measure $\mathrm{perf}$,
we take the maximum over all trajectories,
i.e.,
\begin{equation*}
%\tau_b^{(i)} = 
\mathrm{perf}^{(i)}( \mathbf{y}, \hat{\mathbf{y}} ) =
\max_{(\mathbf{y}, \hat{\mathbf{y}}) \in \{\mathbf{y}^{(ij)}\}_{j=1}^{N_i} \times \{\hat{\mathbf{y}}^{(ij)}\}_{j=1}^k} 
%\tau_b(r_\mathbf{y}, r_{\hat{\mathbf{y}}}),
\mathrm{perf}(\mathbf{y}, {\hat{\mathbf{y}}}),
\end{equation*}
where $\{\mathbf{y}^{(ij)}\}_{j=1}^{N_i}$ are the ground truths for query $\mathbf{x}^{(i)}$ and
$\{\hat{\mathbf{y}}^{(ij)}\}_{j=1}^k$ are the top-$k$ recommendations.



%
\subsection{Results and discussion}
\label{sec:result}

% !TEX root=main.tex
Osaka percentage of predictions with loops:
11 / 46 = 23.9\% \\
Glasgow percentage of predictions with loops:
20 / 64 = 31.2\% \\
Toronto percentage of predictions with loops:
48 / 99 = 48.5\% 

\begin{table*}[!t]
	\caption{F$_1$ score on points.}
	\label{tab:f1-master}
	\centering	
	\resizebox{0.575\textwidth}{!}{
	\subfloat[Raw scores.]{	
	\begin{tabular}{l|ccc|c}
		\hline
		 & \textsc{List Viterbi} & \textsc{ILP} & \textsc{Heuristic} & \textsc{Viterbi} \\ \hline
        Osaka & $0.638\pm0.039$ & $0.638\pm0.039$ & $0.635\pm0.039$ & $0.622\pm0.040$ \\
        Glasgow & $0.741\pm0.028$ & $0.741\pm0.028$ & $0.718\pm0.030$ & $0.689\pm0.032$ \\
        Toronto & $0.754\pm0.023$ & $0.754\pm0.023$ & $0.699\pm0.027$ & $0.651\pm0.030$ \\
		\hline
	\end{tabular}
	}
	\label{tab:f1}%
	}%	
	\subfloat[Improvement over \textsc{Viterbi}.]{
	\begin{tabular}{l|ccc}
		\hline
		 & \textsc{List Viterbi} & \textsc{ILP} & \textsc{Heuristic} \\ \hline
        Osaka & $2.6\%$ & $2.6\%$ & $2.0\%$ \\
        Glasgow & $7.6\%$ & $7.6\%$ & $4.2\%$ \\
        Toronto & $15.7\%$ & $15.7\%$ & $7.3\%$ \\
		\hline
	\end{tabular}
	\label{tab:f1-up}%	
	}
	%}
\end{table*}

\begin{table*}[!t]
	\caption{F$_1$ score on pairs.}
	\label{tab:pf1-master}	
	\centering
	\resizebox{0.575\textwidth}{!}{
	\subfloat[Raw scores.]{
	\label{tab:pf1}
	\begin{tabular}{l|ccc|c}
		\hline
		 & \textsc{List Viterbi} & \textsc{ILP} & \textsc{Heuristic} & \textsc{Viterbi} \\ \hline
        Osaka & $0.375\pm0.059$ & $0.375\pm0.059$ & $0.369\pm0.059$ & $0.364\pm0.060$ \\
        Glasgow & $0.508\pm0.048$ & $0.506\pm0.048$ & $0.480\pm0.049$ & $0.461\pm0.050$ \\
        Toronto & $0.529\pm0.037$ & $0.530\pm0.037$ & $0.490\pm0.041$ & $0.463\pm0.041$ \\
		\hline
	\end{tabular}
	}
	}
	\subfloat[Improvement over \textsc{Viterbi}.]{
		\label{tab:pf1-up}
		\begin{tabular}{l|ccc}
			\hline
			 & \textsc{List Viterbi} & \textsc{ILP} & \textsc{Heuristic} \\ \hline
            Osaka & $3.0\%$ & $3.0\%$ & $1.4\%$ \\
            Glasgow & $10.1\%$ & $9.7\%$ & $4.1\%$ \\
            Toronto & $14.4\%$ & $14.5\%$ & $5.8\%$ \\
			\hline
		\end{tabular}
	}
\end{table*}

\begin{table*}[!t]
    \caption{F$_1$ score on points of queries that recommendations by \textsc{Viterbi} are \emph{different} from those by \textsc{List Viterbi}.}
    \label{tab:f1-diff}
	\centering	
	\resizebox{0.575\textwidth}{!}{
    \subfloat[Raw scores.]{
	\begin{tabular}{l|ccc|c}
		\hline
		 & \textsc{List Viterbi} & \textsc{ILP} & \textsc{Heuristic} & \textsc{Viterbi} \\ \hline
        Osaka & $0.405\pm0.042$ & $0.405\pm0.042$ & $0.389\pm0.038$ & $0.338\pm0.034$ \\
        Glasgow & $0.644\pm0.040$ & $0.644\pm0.040$ & $0.569\pm0.045$ & $0.476\pm0.040$ \\
        Toronto & $0.654\pm0.025$ & $0.654\pm0.025$ & $0.540\pm0.030$ & $0.443\pm0.028$ \\
		\hline
	\end{tabular}
	}
    \label{tab:f1-diff}
	}%	
	\subfloat[Improvement over \textsc{Viterbi}.]{
	\begin{tabular}{l|ccc}
		\hline
		 & \textsc{List Viterbi} & \textsc{ILP} & \textsc{Heuristic} \\ \hline
        Osaka & $19.7\%$ & $19.7\%$ & $15.1\%$ \\
        Glasgow & $35.2\%$ & $35.2\%$ & $19.4\%$ \\
        Toronto & $47.7\%$ & $47.8\%$ & $22.1\%$ \\
		\hline
	\end{tabular}
	\label{tab:f1-diff-up}%	
	}
\end{table*}

\begin{table*}[!t]
	\caption{F$_1$ score on pairs of queries that recommendations by \textsc{Viterbi} are \emph{different} from those by \textsc{List Viterbi}.}
	\label{tab:pf1-diff}
	\centering
	\resizebox{0.575\textwidth}{!}{
	\subfloat[Raw scores.]{
	\label{tab:pf1-diff}
	\begin{tabular}{l|ccc|c}
		\hline
		 & \textsc{List Viterbi} & \textsc{ILP} & \textsc{Heuristic} & \textsc{Viterbi} \\ \hline
        Osaka & $0.112\pm0.030$ & $0.112\pm0.030$ & $0.088\pm0.026$ & $0.067\pm0.020$ \\
        Glasgow & $0.350\pm0.054$ & $0.345\pm0.055$ & $0.261\pm0.047$ & $0.201\pm0.039$ \\
        Toronto & $0.317\pm0.023$ & $0.318\pm0.024$ & $0.236\pm0.031$ & $0.180\pm0.025$ \\
		\hline
	\end{tabular}
	}
	}
	\subfloat[Improvement over \textsc{Viterbi}.]{
		\label{tab:pf1-diff-up}
		\begin{tabular}{l|ccc}
			\hline
			 & \textsc{List Viterbi} & \textsc{ILP} & \textsc{Heuristic} \\ \hline
            Osaka & $68.2\%$ & $68.2\%$ & $32.7\%$ \\
            Glasgow & $73.9\%$ & $71.4\%$ & $29.9\%$ \\
            Toronto & $76.0\%$ & $76.5\%$ & $30.9\%$ \\
			\hline
		\end{tabular}
	}
\end{table*}


% experimental results
The performance of three baselines and four variants based on structured prediction on two datasets are shown in Table~\ref{tab:result}.
From these results, we can make the following key inferences that validate our contributions.

\textbf{Exploiting the structure of sequences helps}.
We find that all variants of our structured prediction methods achieve better performance than existing baselines.
This indicates that the basis of our approach -- reducing sequence recommendation to a structured prediction problem -- is sensible, and has empirical benefit.

\textbf{Accounting for multiple ground truths helps}.
We find that \textsc{SR} always performs better than \textsc{SP},
and similarly for the {\sc path} variants of both methods. 
This indicates that our first extension -- explicitly modelling multiple ground truths helps recommendation -- is important to achieve good performance.
(We note that even without this correction, our structured methods outperform baselines.)

\textbf{Eliminating loops during training helps}.
We find that {\sc SRpath} improves performance further of the {\sc SR} method,
as indicated by the F$_1$ score on pairs.
This indicates that our second extension -- explicitly performing sub-tour elimination in training -- is important to further improve performance.
Interestingly,
this advantage does \emph{not} take effect if the multiple ground truths are not modelled explicitly,
with the performance of the {\sc SP} method largely unaffected.

Overall, these results indicate that our structured prediction approach to the problem has
benefits over non-structured approaches,
and that our extensions to the vanilla structured approach are important to further improve performance.
