% !TEX root=main.tex

We formalised the problem of eliminating loops when recommending trajectories to visitors in a city,
and surveyed three distinct approaches to the problem -- a heuristic inspired by Christofides' algorithm, list extensions of the Viterbi algorithm, and integer linear programming.
Through experiments on real-world datasets, 
Overall, we find that
all methods offer performance improvements over na\"{i}vely predicting a sequence with loops;
Christofides' algorithm is at least an order of magnitude faster than the other methods, but with a significant sacrifice in accuracy;
and that the list Viterbi is faster than the ILP for short trajectories, but the ILP is superior for longer trajectories.

The basic idea of modifying the standard Viterbi inference problem (Equation \ref{eqn:argmax}) has other applications, such
as ensuring diversity in the predicted ranking.
Such problems have been studied in different contexts such as information retrieval \citep{Carbonell:1998} and computer vision \citep{Park:2011},
and would be interesting for further study in trajectory recommendation.

% {\color{red!75}
% \begin{itemize}
% 	\item Learning
% 	\item diversity, MMR
% 	\item See whether any of the workshop topics might have problems that this paper applies.
% \end{itemize}
% }
