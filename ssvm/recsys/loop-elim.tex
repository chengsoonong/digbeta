% !TEX root=main.tex

%An even simpler approach than the above greedy strategy is the following:
Perhaps the simplest approximate solution to Equation \ref{eqn:argmax-path} is to simply remove the first loop occurring in the standard Viterbi solution (Equation \ref{eqn:argmax}).
Specifically, if the Viterbi solution is $( y_1, \ldots, y_l )$,
and $i$ is the first index where $y_i$ already appears in the subsequence $( y_1, \ldots, y_{i-1} )$,
then simply return the subsequence $( y_1, \ldots, y_{i-1} )$.

From a graph perspective, this approach makes the directed sub-graph induced by the Viterbi solution acyclic
by simply breaking the first cycle-inducing edge.
This is sensible if sequences never escape the first cycle, \ie after the first repeated POI, there is no new POI.
We have indeed found this to be the case in our problem (perhaps owing to cycles being induced by dominant edge scores).
More generally, for sequences %such as $(1,2,1,3)$
where there is a repeated POI followed by a new POI,
the problem of removing cycles can be seen as a special case of the ({\sf NP}-hard) minimum feedback arc-set problem.

This algorithm is appealing in its simplicity,
but has at least two detrimental features.
First, it makes the questionable assumption that we can solve Equation \ref{eqn:argmax-path} from the standard Viterbi solution alone.
Second, it returns a solution that violates the length constraint of the path recommendation problem.
As a remedy to this, we can request the Viterbi algorithm to return a path of longer length $l' > l$;
of course, there is no clear means of choosing $l'$ so that the result is exactly of length $l$.

% Second, if we restrict attention to the POIs $\{ y_1, \ldots, y_{i-1} \}$, it is unclear whether the original ordering of the subsequence is optimal.
% The second point can be remedied, as we now see.
