% !TEX root=main.tex

An even simpler approach than the above greedy strategy is the following:
by default, return the standard Viterbi solution (Equation \ref{eqn:argmax});
if this solution has loops, then simply remove them.
Specifically, if the Viterbi solution is $( y_1, \ldots, y_l )$,
and $i$ is the first index where $y_i$ already appears in the subsequence $( y_1, \ldots, y_{i-1} )$,
then simply return the subsequence $( y_1, \ldots, y_{i-1} )$.

This algorithm is appealing in its simplicity,
but has at least two detrimental features.
First, it is unclear what performance guarantee this method brings.
Second, it returns a solution that violates the length constraint of the trajectory recommendation problem.
As a remedy to this, we can request the Viterbi algorithm to return a path of longer length $l' > l$;
of course, there is no clear means of choosing $l'$ so that the result is exactly of length $l$.

% Second, if we restrict attention to the POIs $\{ y_1, \ldots, y_{i-1} \}$, it is unclear whether the original ordering of the subsequence is optimal.
% The second point can be remedied, as we now see.
