%!TEX root = main.tex

\section{Related Work}
\label{sec:relatedwork}

As geo-tagged travel data become widely available, problems in exploring routes and travel patterns has received increasing attention across several research communities -- including information retrieval and recommender systems, databases, social media, geographic information systems, and artificial intelligence. 
%Exhaustive surveying of the area is beyond the scope of this work, we refer the readers to a number of  
We summarize recently work most related to formulating and solving learning problems on assembling routes 
from points-of-interest (POI), according to the problem setting, the methods for ranking and selecting places and routes, as well as the use of different features and queries. We refer the reader to a number of recent surveys\cite{bao2015recommendations,zheng2015trajectory,zheng2014urban} for detailed profiles of the area.

There are several problem settings for travel patterns and trajectory data. 
The first setting can be called {\em discovery}, such as identifying points of interest~\cite{zheng2009mining} identifying general travel patterns such as the most popular routes~\cite{lu2010photo2trip} or 
quantifying tourist traffic flow between points of interest~\cite{zheng2012patterns}. 
The second setting is POI recommendation. The basic formulation is collaborative filtering model for learning user-location affinity~\cite{shi2011personalized}, with additional ways to incorporate historic information in spatial~\cite{lian2014geomf} and temporal~\cite{gao2013temporal} data. See a detailed survey by Bao et al~\cite{bao2015recommendations} on this topic. 
The third setting is next location recommendation. It is a variant of POI recommendation except given both the user and locations traveled to date. The solutions to this problem include incorporating Markov chains into collaborative filtering~\cite{fpmc10,ijcai13} or with sequence models such as recurrent neural networks~\cite{aaai16}. 
The last problem setting is finding and ranking a whole trip. This usually translate into combinatorial optimisation problems~\cite{ijcai15,lu2012personalized} with the objective derived from user history and constraints designed from travel lengths.


{\bf Method}
MF, recommender system for places;

planning:
% Trip Builder
\cite{tripbuilder15} recommended trajectories by first addressing a Generalized Maximum Coverage problem with respect to user preference and 
time budget to get a set of candidate trajectories, which were then scheduled by solving a variant of Traveling Salesman Problem to form the 
final recommendation.
% WSDM'14
\cite{wsdm14} proposed a framework to recommend trajectories based on user provided constraints such as the visiting order of different 
categories of POIs, time and distance budget as well as the upper/lower bounds of the number of POIs in each category that a user wish 
to visit, user satisfaction at a POI was modeled by either associate a benefit or a set of other POIs and activities to that POI,
and proved the maximization of user satisfaction is NP-hard in both cases. 
They further proposed pseudo polynomial dynamic programming algorithms as well as 
a $1-\epsilon$ approximation algorithm to maximize the user satisfaction.


joint
\cite{lu2012personalized} claims to do both point ranking and trip planning, heuristic combination for point scoring, branch-n-bound search for planning. 
% IJCAI'15
\cite{ijcai15} modeled user's interest on a specific category of POIs based on the observation that if a user is more interested in a certain category of POIs, his/her visit duration would be longer than the average visit duration in general. They then formulated trajectory recommendation as a Orienteering problem and use integer programming to optimize an objective which was a composite of the total POI popularity as well as the total user interest on POIs in the recommended trajectory 
with respect to a number of constraints such as the start/destination POI and time budget.
\cite{kurashima2010geotag} modeled user preferences of POIs and his/her transition patterns 
between POIs using a hybrid of 
Markov and topic models, and recommend a trajectory by search the sequence of POIs with highest probability for this user  with respect to a time budget.
greedy on best next location. 

\cite{chen2015tripplanner}

{\bf queries, features, available data}
clustering space~\cite{hu2013spatialtopic}
travel time~\cite{gao2013temporal}
geo-MF for POI recommendation~\cite{lian2014geomf}
spatial topic model for local rec~\cite{hu2013spatialtopic}
\cite{ht14} aims to recommend a short and pleasant trajectory from the current POI to a destination POI by choosing the best average rank 
of all POIs in a trajectory from the $M$ shortest paths that connect the current location and destination,
POIs were ranked according to the degree of pleasure based on user votes and  crowd-sourced emotion scores on three aspects 
(i.e., beauty, quietness and happiness).
\cite{travel13} recommend places to travelers by extracting traveling patterns based on automatically mined user attributes 
(e.g., age, gender and race) and travel group types (e.g., couple, family and friends) from photos provided by online photo 
sharing sites (e.g., Flickr).

\cite{Zhang2015OOP} opion-based

POI rec from photos~\cite{shi2011personalized}

``Photo2Trip: generating travel routes from geo-tagged photos for trip planning''~\cite{lu2010photo2trip}

Estimation markov chain from regions of interest in photos~\cite{zheng2012patterns}

survey of location-based recommendation~\cite{bao2015recommendations}

correlation between check-in time and location~\cite{gao2013temporal}

LORE, use additive markov chain\cite{zhang2014lore}

{\bf what we do:}
trajectory recommendation, learning the joint preference of ranks routes, on minimum but commonly available set of features. 


