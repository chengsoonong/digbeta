\section{Introduction}
\label{sec:introduction}
Location based information sharing services provided by online social media,
for example Facebook, Twitter and Flickr, generate data with geographical information,
together with time-stamps and user tags.
This enables us to identify the trajectory of a particular user as he or she travels through
a city. By combining trajectories of multiple users in a particular city, we
have an opportunity to explore patterns of tours. Using POI properties such as the type
or category of the site visited,
as well as transition patterns between different POIs,
we obtain a way to recommend high quality trajectories to travellers.

\cheng{Need to describe the problem setting}

We propose using learning to rank to capture the properties of POIs, as well as a Markov chain to
represent the transition patterns between POIs.
We factorise the transition probabilities of the Markov chain between POIs
according to several POI features to deal with data sparsity.
We combine the results of learning to rank and the factorized Markov Chain using both a probabilistic model and a structured
support vector machine, and evaluate the quality of POIs in recommended trajectories in terms of trajectory F$_1$-score\cite{ijcai15}. One problem with the F$_1$-score on POIs is that it ignores
the order in which sites are visited.
We propose a new performance measure by applying F$_1$-score on pairs of POIs.
Experimental results on five trajectory datasets show performance improvements over the state-of-the-art methods and
reveal many interesting properties of trajectories in different datasets.


novel points:

(1) joint optimization of point preference and route plan;
(2) feature-driven, incorporates information about time, location, POI categories and behavior history
(3) strong performance compared to IJCAI 2015. we can also quantify the contribution from both point ranking and routing.
(4) F1 on pairs.
