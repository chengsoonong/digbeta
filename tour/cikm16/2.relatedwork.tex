\section{Related Work}
\label{sec:relatedwork}

We summarise recent work most related to formulating and solving learning problems on assembling routes from POIs.
There are three settings of recommendation problems for locations and routes, as illustrated in Figure~\ref{fig:threesettings}.
The first setting can be called POI recommendation (Figure~\ref{fig:threesettings}(a)). Each location (A to E) is scored with geographic and behavioural information such as category, reviews, popularity, spatial information such as distance, and temporal information such as travel time uncertainty, time of the day or day of the week.
This can be in discovery mode, such as identifying points-of-interest~\cite{zheng2009mining,li2015instagram} and includes efficient querying of geographic objects for trips ~\cite{hashem2015efficient}.
A popular approach is based on the collaborative filtering model
for learning user-location affinity~\cite{shi2011personalized}, with additional ways to incorporate spatial~\cite{lian2014geomf,liu2014exploiting}, temporal~\cite{yuan2013timeaware,hsieh2014mining,gao2013temporal}, or spatial-temporal~\cite{yuan2014graph} information.

Figure~\ref{fig:threesettings}(b) illustrates the second setting: next location recommendation.
Here the input is a partial trajectory (e.g. started at point A and currently at point B), the task of the algorithm is to score the next candidate location (e.g, C, D and E) based on the perceived POI score and transition compatibility with input $A\rightarrow B$.
It is a variant of POI recommendation except given both the user and locations travelled to date. The solutions to this problem include incorporating Markov chains into collaborative filtering~\cite{fpmc10,ijcai13,zhang2015location},
quantifying tourist traffic flow between points-of-interest~\cite{zheng2012patterns},
formulating a binary decision or ranking problem~\cite{baraglia2013learnext}, and with sequence models such as recurrent neural networks~\cite{aaai16}.


This paper considers the final setting: tour recommendation, as illustrated in Figure~\ref{fig:threesettings}(c). Here the input are some factors about the desired route, e.g. starting point A and end point C, along with auxiliary information such as the desired length of trip. The algorithm needs to take into account location desirability (as indicated by node size) and transition compatibility (as indicated by edge width), and compare route hypotheses such as A-D-B-C and A-E-D-C. Existing work in this area either uses heuristic combination of locations and routes~\cite{lu2010photo2trip,ijcai15,lu2012personalized}, or formulates an optimisation problem that is not informed or evaluated by behaviour history~\cite{gioniswsdm14,chen2015tripplanner}. Joint learning of location preferences and routes remains an open problem.
Our work is in this category. We formulate a learning problem to score the whole trajectory, taking into account individual POI properties and relationships among different POIs.
